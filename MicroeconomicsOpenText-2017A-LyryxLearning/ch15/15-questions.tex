\newpage
\section*{Exercises for Chapter~\ref{chap:internationaltrade}}

\begin{enumialphparenastyle}

% Solutions file for exercises opened
\Opensolutionfile{solutions}[solutions/ch15ex]

\begin{ex}\label{ex:ch15ex1}
The following table shows the labour input requirements to produce a bushel of wheat and a litre of wine in two countries, Northland and Southland, on the assumption of constant cost production technology -- meaning that the production possibility curves in each are straight lines.
\begin{center}
\begin{tabu} to 25em {|X[3,c]X[1,c]X[1,c]|} \hline 
\multicolumn{3}{|c|}{\cellcolor{rowcolour}\textbf{Labour requirements per unit produced}} \\
	& Northland & Southland \\
\rowcolor{rowcolour}Per bushel of wheat & 1 & 3 \\ 
Per litre of wine & 2 & 4 \\ \hline 
\end{tabu}
\end{center}
\begin{enumerate}
\item  Which country has an absolute advantage in the production of both wheat and wine?
\item  What is the opportunity cost of wheat in each economy? Of wine?
\item  What is the pattern of comparative advantage here?
\item  Suppose the country with a comparative advantage in wine reduces wheat production by one bushel and reallocates the labour involved to wine production. How much additional wine does it produce?  
\item  Which country, if either, gains from this change in production and trade, and what is the gain?
\item  If the country with the comparative advantage in wheat reduced wine production enough to increase wheat production by one bushel, how much wine could it get by selling the additional bushel of wheat to the other country at that economy's opportunity cost?
\end{enumerate}
\begin{sol}
\begin{enumerate}
	\item	Northland has an absolute advantage in the production of both goods, as it has lower labour requirements for each.
	\item	The opportunity cost of 1 bushel of wheat is 1/2 litre of wine in Northland and 3/4 litre of wine in Southland.
	\item	Northland has a comparative advantage in wheat while Southland does in wine.
	\item	By reducing wheat production by 1 bushel, Southland can produce an additional 3/4 litre of wine.
	\item	Both countries can gain if Northland shifts production from wine to wheat and the countries trade wine for wheat at a rate between 1/2 litre of wine for 1 bushel of wheat and 3/4 litre of wine for one bushel of wheat.
	\item	By reducing wine production by 1/2 litre, Northland can increase wheat production by 1 bushel, which, at Southland's opportunity cost, exchanges for 3/4 litre of wine, giving Northland a gain of 1/4 litre of wine.
\end{enumerate}
\end{sol}
\end{ex}

\begin{ex}\label{ex:ch15ex2}
Canada and the United States can produce two goods, xylophones and yogourt. Each good can be produced with labour alone. Canada requires 60 hours to produce a ton of yogourt and 6 hours to produce a xylophone. The United States requires 40 hours to produce the ton of yogourt and 5 hours to produce a xylophone.
\begin{enumerate}
\item  Describe the state of absolute advantage between these economies in producing goods.
\item  In which good does Canada have a comparative advantage? Does this mean the United States has a comparative advantage in the other good?
\item  Draw the production possibility frontier for each economy to scale on a diagram, assuming that each economy has an endowment of 240 hours of labour.
\item  On the same diagram, draw Canada's consumption possibility frontier on the assumption that it can trade with the United States at the United States rate of transformation. 
\item  Draw the US consumption possibility frontier under the assumption that it can trade at Canada's rate of transformation.
\end{enumerate}
\begin{sol}
\begin{enumerate}
	\item	The US has an absolute advantage in both goods.
	\item	Canada has a comparative advantage in xylophones. The US has a comparative advantage in yogourt.
	\item	See diagram below.
	\item	See diagram below.
\end{enumerate}
\begin{center}
\begin{tikzpicture}[background color=figurebkgdcolour,use background,xscale=0.2,yscale=0.5]
	\draw [dashed,ultra thick,name path=cdnconpos] (0,5.8333) --(35,0);
	\draw [dashed,ultra thick,name path=usconpos] (0,8) -- (48,0);
	\draw [ppfcolourthree,ultra thick,name path=ppfcan] (0,5) node [black,mynode,left] {4} -- (35,0) node [black,mynode,below] {40};
	\draw [ppfcolourthree,ultra thick,name path=ppfus] (0,8) node [black,mynode,left] {6} -- (40,0) node [black,mynode,below] {48};
	\draw [thick, -] (0,10) node [mynode1,above] {Yogourt} |- (50,0) node [mynode1,right] {Xylophone};
	\path [name path=arrowline] (0,4.5) -- +(50,-3);
	\draw [name intersections={of=arrowline and ppfcan, by=i1}]
		[<-,thick,shorten <=1mm] (i1) -- +(-1,-2) node [mynode,below] {$PPF_{\text{CAN}}$};
	\draw [name intersections={of=arrowline and ppfus, by=i2}]
		[<-,thick,shorten <=1mm] (i2) -- +(10,3) node [mynode,right] {$PPF_{\text{US}}$};
	\draw [name intersections={of=arrowline and cdnconpos, by=i3}]
		[<-,thick,shorten <=1mm] (i3) -- +(-1,-2) node [mynode,below] {$CPF_{\text{CAN}}$};
	\draw [name intersections={of=arrowline and usconpos, by=i4}]
		[<-,thick,shorten <=1mm] (i4) -- +(10,3) node [mynode,right] {$CPF_{\text{US}}$};
\end{tikzpicture}
\end{center}
\end{sol}
\end{ex}

\begin{ex}\label{ex:ch15ex3}
The domestic demand for bicycles is given by $P=36-0.3Q$. The foreign supply is given by $P=18$ and domestic supply by $P=16+0.4Q$.
\begin{enumerate}
\item  Illustrate the market equilibrium on a diagram, and compute the amounts supplied by domestic and foreign suppliers.
\item  If the government now imposes a tariff of \$6 per unit on the foreign good, illustrate the impact geometrically, and compute the new quantities supplied by domestic and foreign producers.
\item  In the diagram, illustrate the area representing tariff revenue and compute its value.
\end{enumerate}
\begin{sol}
\begin{enumerate}
	\item	The diagram shows that the amount traded is 60 units; of which domestic producers supply 5 and 55 are imported.
	\item	In this case, the foreign supply curve SW shifts up from a price of \$18 to \$24. The amount traded is now 40 units, 20 of which are supplied domestically.
	\item	Tariff revenue is EFHI$=\$120$.
\end{enumerate}
\begin{center}
\begin{tikzpicture}[background color=figurebkgdcolour,use background,xscale=0.25,yscale=0.25]
	% supply
	\draw [supplycolour,dashed,ultra thick,domain=0:17,name path=DomSup] plot (\x, {3+\x}) node [black,mynode,above] {$S$};
	% demand
	\draw [demandcolour,ultra thick,name path=DomDem] (7,20) node [mynode,above,black] {$D$} -- (25,2);
	% supply with and without tariff
	\draw [supplycolour,ultra thick,name path=WorldSup] (0,10) node [black,mynode,left] {\$18} -- (24,10) node [black,mynode,right] {$S_W$};
	\draw [supplycolour,ultra thick,name path=WorldSupTariff] (0,12) node [black,mynode,left] {\$24} -- (24,12) node [black,mynode,right] {$S_W+\text{tariff}$};
	% axes
	\draw [thick, -] (0,23) node (yaxis) [mynode1,above] {Price} |- (25,0) node (xaxis) [mynode1,right] {Quantity};
	% intersection of lines
	\draw [name intersections={of=WorldSup and DomSup, by=C},name intersections={of=WorldSupTariff and DomSup, by=E},name intersections={of=WorldSup and DomDem, by=G},name intersections={of=WorldSupTariff and DomDem, by=F}];
	
	% paths to create points I and H on WorldSup line
	\path [name path=Iline] (xaxis -| E) -- +(0,23);
	\path [name path=Hline] (xaxis -| F) -- +(0,23);
	% intersection of Iline and Hline with WorldSup and dotted lines
	\draw [name intersections={of=Iline and WorldSup, by=I},name intersections={of=Hline and WorldSup, by=H}]
		[dotted,thick] (E) node [mynode,above] {E} -- (I) node [mynode,below left] {I} -- (xaxis -| I) node [mynode,below] {20}
		[dotted,thick] (F) node [mynode,above] {F} -- (H) node [mynode,below right] {H} -- (xaxis -| H) node [mynode,below] {40};
	% coloured square
	\draw [fill=demandcolour!25,demandcolour!25] ([xshift=0.5mm,yshift=-0.75mm]E) -- ([xshift=-0.5mm,yshift=-0.75mm]F) -- ([xshift=-0.5mm,yshift=0.75mm]H) -- ([xshift=0.5mm,yshift=0.75mm]I);
	% Coloured triangles A (region enclosed by C-E-I) and B (region enclosed by F-G-H)
	\draw [fill=supplycolour!25,supplycolour!25] ([xshift=2.5mm,yshift=0.75mm]C) -- coordinate [midway] (Aarrow) ([xshift=-0.5mm,yshift=-2.5mm]E) -- ([xshift=-0.5mm,yshift=0.75mm]I);
	\draw [fill=supplycolour!25,supplycolour!25] ([xshift=0.5mm,yshift=-2.5mm]F) -- coordinate [midway] (Barrow) ([xshift=-2.5mm,yshift=0.75mm]G) -- ([xshift=0.5mm,yshift=0.75mm]H);
	% arrows to Aarrow and Barrow
	\draw [<-,thick,shorten <=-1.5mm] (Aarrow) -- +(-3,3) node [mynode,above] {A};
	\draw [<-,thick,shorten <=-1.5mm] (Barrow) -- +(3,3) node [mynode,above] {B};
	% dotted line from 5Q
	\path [name path=5Qline] (2,0) -- +(0,23);
	\draw [name intersections={of=5Qline and WorldSup, by=5Q}]
		[dotted,thick] (5Q) -- (xaxis -| 5Q) node [mynode,below] {5};
	% dotted line from 60Q
	\path [name path=60Qline] (21,0) -- +(0,23);
	\draw [name intersections={of=60Qline and WorldSup, by=60Q}]
		[dotted,thick] (60Q) -- (xaxis -| 60Q) node [mynode,below] {60};
	% arrow between SW and SW+tariff
	\path [name path=arrowpath] (23,0) -- +(0,23);
	\draw [name intersections={of=arrowpath and WorldSup, by=s1},name intersections={of=arrowpath and WorldSupTariff, by=s2}]
		[->,thick,shorten <=0.5mm,shorten >=0.5mm] (s1) -- (s2);
\end{tikzpicture}
\end{center}
\end{sol}
\end{ex}
	
\begin{ex}\label{ex:ch15ex4}
In Exercise~\ref{ex:ch15ex3}, illustrate the deadweight losses associated with the imposition of the tariff, and compute the amounts.
\begin{enumerate}
\item  Compute the additional amount of profit made by the domestic producer as a result of the tariff. [Hint: refer to Figure~\ref{fig:tarifftrade} in the text.]
\end{enumerate}
\begin{sol}
\begin{enumerate}
	\item	The deadweight losses correspond to the two triangles, A and B, in the diagram, and amount to \$105.
	\item	The amount of additional profit for domestic producers is \$75.
\end{enumerate}
\end{sol}
\end{ex}

\begin{ex}\label{ex:ch15ex5}
The domestic demand for office printers is given by $P=40-0.2Q$. The supply of domestic producers is given by $P=12+0.1Q$, and international supply by $P=20$. 
\begin{enumerate}
\item  Illustrate this market geometrically.
\item  Compute total demand and the amounts supplied by domestic and foreign suppliers.
\item  If the government gives a production subsidy of \$2 per unit to domestic suppliers in order to increase their competitiveness, calculate the new amounts supplied by domestic and foreign producers. [Hint: The domestic supply curve becomes $P=10+0.1Q$].
\item  Compute the cost to the government of this scheme.
\end{enumerate}
\begin{sol}
\begin{enumerate}
	\item	See figure below.
	\item	The total quantity of trade is 100 units, of which 80 are supplied domestically.
	\item	The subsidy shifts the domestic supply curve down by \$2 at each quantity. This supply intersects the demand curve at $Q=100$. Foreign producers are squeezed out of the market completely.
	\item	Cost to the government is \$200.
\end{enumerate}
\begin{center}
\begin{tikzpicture}[background color=figurebkgdcolour,use background,xscale=0.03,yscale=0.15]
	\draw [thick] (0,45) node (yaxis) [mynode1,above] {Price} |- (220,0) node (xaxis) [mynode1,right] {Quantity};
	\draw [ultra thick,demandcolour,name path=D] (0,40) node [mynode,left,black] {40} -- node [mynode,above right,black,pos=0.2] {$D$} (200,0) node [mynode,below,black] {200};
	\draw [ultra thick,supplycolour,name path=SW] (0,20) node [mynode,left,black] {20} -- +(210,0) node [mynode,right,black] {$S_W$};
	\draw [ultra thick,dashed,supplycolour,name path=SD] (0,12) node [mynode,left,black] {12} -- (210,33) node [mynode,right,black] {$S_{\text{DOM}}$};
	\draw [ultra thick,supplycolour,name path=SDsub] (0,10) node [mynode,left,black] {10} -- (210,31) node [mynode,right,black] {$S_{\text{DOM}}$ with subsidy};
	\draw [name intersections={of=D and SD, by=80Q},name intersections={of=D and SDsub, by=100Q}]
		[dotted,thick] (80Q) -- (xaxis -| 80Q) node [mynode,below left=0cm and -0.2cm] {80}
		[dotted,thick] (100Q) -- (xaxis -| 100Q) node [mynode,below right=0cm and -0.2cm] {100};
\end{tikzpicture}
\end{center}
\end{sol}
\end{ex}
	
\begin{ex}\label{ex:ch15ex6}
The domestic demand for turnips is given by $P=128-(1/2)Q$. The market supply of domestic suppliers is given by $P=12+(1/4)Q$, and the world price is \$32 per bushel. 
\begin{enumerate}
\item  First graph this market and then solve for the equilibrium quantity purchased.
\item  How much of the quantity traded will be produced domestically and how much will be imported?
\item  Assume now that a quota of 76 units is put in place. Illustrate the resulting market equilibrium graphically.
\item  Compute the domestic price of turnips and the associated quantity traded with the quota in place. [Hint: you could shrink the demand curve in towards the origin by the amount of the quota and equate the result with the domestic supply curve].
\end{enumerate}
\begin{sol}
\begin{enumerate}
	\item	See diagram below.
	\item	Domestic producers will supply 80 and imports will be 112.
	\item	The equilibrium with the quota is point A in the diagram with imports equal to the quota of 76.
	\item	The equilibrium quantity with the quota is 180, with 76 imported and 104 supplied by domestic producers. The equilibrium market price is \$38.
\end{enumerate}
\begin{center}
\begin{tikzpicture}[background color=figurebkgdcolour,use background,xscale=0.027,yscale=0.05]
	\draw [thick] (0,130) node (yaxis) [mynode1,above] {Price} |- (260,0) node (xaxis) [mynode1,right] {Quantity};
	\draw [ultra thick,demandcolour,name path=D] (0,128) node [mynode,left,black] {128} -- node [mynode,above right,black,pos=0.2] {$D_{\text{DOM}}$} (256,0) node [mynode,below,black] {256};
	\draw [ultra thick,supplycolour,name path=SW] (0,32) coordinate (P32) node [mynode,left,black] {32} -- +(260,0) node [mynode,right,black] {$S_W$};
	\draw [ultra thick,supplycolour,name path=SD] (0,12) node [mynode,left,black] {12} -- (260,77) node [mynode,right,black] {$S_{\text{DOM}}$};
	\draw [ultra thick,dashed,supplycolour,name path=SDquota] (156,0) node [mynode,below,black] {156} -- (156,32) -- (260,58) node [mynode,right,black] {$S_{\text{DOM(quota)}}$};
	\draw [name intersections={of=D and SDquota, by=A}]
		[dotted,thick] (yaxis |- A) node [mynode,left] {38} -- (A) node [mynode,above] {A} -- (xaxis -| A) node [mynode,below left=0cm and -0.25cm] {180};
	\draw [name intersections={of=SD and SW, by=Q80}]
		[dotted,thick] (Q80) -- (xaxis -| Q80) node [mynode,below] {80};
	\path [name path=Q192line] (192,0) -- +(0,128);
	\draw [name intersections={of=Q192line and SW, by=Q192}]
		[dotted,thick] (Q192) -- (xaxis -| Q192) node [mynode,below right=0cm and -0.25cm] {192};
\end{tikzpicture}
\end{center}
\end{sol}
\end{ex}
	
\begin{ex}\label{ex:ch15ex7}
The domestic market for cheese is given by $P=108-2Q$ and $P=16+1/4Q$. These are the demand and supply conditions. The good can be supplied internationally at a constant price $P=20$.
\begin{enumerate}
\item  Illustrate the domestic market in the absence of trade and solve for the equilibrium price and quantity.
\item  With free trade illustrate the market graphically and compute the total amount purchased, and the amounts supplied by domestic and international suppliers. 
\item  Suppose now that the government implements a price floor in the domestic market equal to \$28. Illustrate the market outcome graphically.
\item  For the outcome with a price floor, compute the quantity supplied by domestic and international suppliers respectively. 
\end{enumerate}
\begin{sol}
\begin{enumerate}
	\item	See diagram below.
	\item	See diagram below.
	\item	The quantity permitted to be brought to market would be 40 units, even though the supply side would be willing to supply more at this price, buyers will demand just 40 at a price of \$28.
\end{enumerate}
\begin{center}
\begin{tikzpicture}[background color=figurebkgdcolour,use background,xscale=0.12,yscale=0.18]
	\draw [thick] (0,35) node (yaxis) [mynode1,above] {Price} |- (60,0) node (xaxis) [mynode1,right] {Quantity};
	\draw [ultra thick,demandcolour,name path=D] (36.5,35) -- node [mynode,above right,black,pos=0.2] {$D_{\text{DOM}}$} (54,0) node [mynode,below,black] {54};
	\draw [ultra thick,supplycolour,name path=SW] (0,20) node [mynode,left,black] {20} -- +(60,0) node [mynode,right,black] {$S_W$};
	\draw [ultra thick,dashed,supplycolour,name path=PF] (0,28) node [mynode,above left=-0.2cm and 0cm,black] {28} -- +(60,0);
	\draw [ultra thick,supplycolour,name path=SD] (0,16) node [mynode,left,black] {16} -- (60,31) node [mynode,right,black] {$S_{\text{DOM}}$};
	\draw [name intersections={of=D and PF, by=i1},name intersections={of=SD and SW, by=i2},name intersections={of=D and SW, by=i3},name intersections={of=D and SD, by=i4}]
		[dotted,thick] (i1) -- (xaxis -| i1) node [mynode,below left=0cm and -0.2cm] {40}
		[dotted,thick] (i2) -- (xaxis -| i2) node [mynode,below] {16}
		[dotted,thick] (i3) -- (xaxis -| i3) node [mynode,below] {44}
		[dotted,thick] ([xshift=-4cm]yaxis |- i4) node [mynode,left] {26} -| ([yshift=-2cm]xaxis -| i4) node [mynode,below] {41};
\end{tikzpicture}
\end{center}
\end{sol}
\end{ex}
	
\begin{ex}\label{ex:ch15ex8}
The following are hypothetical production possibilities tables for Canada and the United States. For each line required, plot any two or more points on the line.
\begin{center}
\begin{tabu} to \linewidth {|X[1,c]X[0.5,c]X[0.5,c]X[0.5,c]X[0.5,c]X[1,c]X[0.5,c]X[0.5,c]X[0.5,c]X[0.5,c]|} \hline 
\multicolumn{5}{|c}{\cellcolor{rowcolour}\textbf{Canada}} & \multicolumn{5}{c|}{\cellcolor{rowcolour}\textbf{United States}} \\
	& A & B & C & D &  & A & B & C & D \\
\rowcolor{rowcolour}	\textbf{Peaches} & 0 & 5 & 10 & 15 & \textbf{Peaches} & 0 & 10 & 20 & 30 \\
\textbf{Apples} & 30 & 20 & 10 & 0 & \textbf{Apples} & 15 & 10 & 5 & 0 \\ \hline 
\end{tabu}
\end{center}
\begin{enumerate}
\item  Plot Canada's production possibilities curve by plotting at least 2 points on the curve. 
\item  Plot the United States' production possibilities curve by plotting at least 2 points on the curve on the graph above. 
\item  What is each country's cost ratio of producing Peaches and Apples?
\item  Which economy should specialize in which product? 
\item  Plot the United States' trading possibilities curve (by plotting at least 2 points on the curve) if the actual terms of the trade are 1 apple for 1 peach. 
\item  Plot the Canada' trading possibilities curve (by plotting at least 2 points on the curve) if the actual terms of the trade are 1 apple for 1 peach.
\item  Suppose that the optimum product mixes before specialization and trade were B in the United States and C in Canada. What are the gains from specialization and trade? 
\end{enumerate}
\begin{sol}
	The figure below illustrates parts (a) through (f). Since the total production before trade was 20 of each, and after specialization it is 30 of each, the gain is 10 of each good.
	\begin{center}
	\begin{tikzpicture}[background color=figurebkgdcolour,use background,xscale=0.2,yscale=0.2]
		\draw [thick] (0,35) node (yaxis) [mynode1,above] {Apples} |- (35,0) node (xaxis) [mynode1,right] {Peaches};
		\draw [ultra thick,dashed,name path=ConPos] (0,30) node [mynode,left,black] {30} -- coordinate[midway] (ConPosCoord) (30,0) node [mynode,below,black] {30};
		\draw [ultra thick,ppfcolourone,name path=PPFCan] (0,30) coordinate (CanSpec) -- node [mynode,below left,black,pos=0.4] {$PPF_{\text{CAN}}$} (15,0) node [mynode,below,black] {15};
		\draw [ultra thick,ppfcolourtwo,name path=PPFUS] (0,15) node [mynode,left,black] {15} -- node [mynode,below left,black,pos=0.7] {$PPF_{\text{US}}$} (30,0) coordinate (USSpec);
		\draw [<-,thick,shorten <=1mm] (CanSpec) -- +(5,0) node [mynode,right] {Canada should\\specialize in apples};
		\draw [<-,thick,shorten <=1mm] (USSpec) -- +(0,5) node [mynode,above] {US should\\specialize\\in peaches};
		\draw [<-,thick,shorten <=1mm] (ConPosCoord) -- +(5,5) node [mynode,right] {Consumption\\possibilities for each\\economy with an\\exchange rate of 1:1};
	\end{tikzpicture}
	\end{center}
\end{sol}
\end{ex}

% Closes solutions file for this chapter
\Closesolutionfile{solutions}

\end{enumialphparenastyle}