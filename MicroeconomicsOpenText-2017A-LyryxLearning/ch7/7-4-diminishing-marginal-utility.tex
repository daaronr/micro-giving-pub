\section{Diminishing marginal utility and risk}\label{sec:ch7sec4}

We proposed in Chapter~\ref{chap:individualchoice} that people have diminishing marginal utility: Successive equal increments in consumption yield progressively less additional utility. This concept is central to understanding behaviour towards risk.

Suppose you have a specific sum of money and you are offered a fair gamble in the sense we described above -- you have an equal chance of winning or losing a fixed amount -- suppose \$1,000. While this game is by definition fair in the probabilistic sense it might not be fair in the utility sense. Diminishing marginal utility means that the additional utility associated with the gain of \$1,000 is less than the utility change associated with losing \$1,000: The more money we have, the less is the utility value placed on it at the margin; the less we have, the greater is the value placed on an additional unit. In consequence, risk-averse individuals devote resources to reducing risk; this is why insurance forms such a sizable sector in the economy. Furthermore, those who take on risk (insurers) have to be rewarded for doing so, and many economic activities consist of the more risk-averse individuals paying insurers to bear risk.

Insurance companies always insure a multitude of individuals or households. Hence, while these companies pay out claims to many customers each year the companies have so many customers that it is rare that they encounter a year when the insurance claims and the insurance premiums paid to the company do not balance out. We sometimes call this the law of large numbers.

\subsection*{Risk pooling}

Consider the following personal relationship that has a strong economic component: an information technology (IT) consultant and his partner, a musician, have risky incomes\footnote{This example is motivated by a somewhat similar example in the economics text book "Economics" by David Begg, Stanley Fischer and Rudiger Dornbusch, 2003, Mc Graw Hill UK.}. Each earns \$5,000 in a good month and nothing in a bad month. But their risks are independent: Whether the musician has a good or bad month has no bearing on the IT consultant's income. Individually, their incomes are quite risky, but together they are less so. Let us examine Table~\ref{table:riskyincome} to see why.

\begin{table}[H]
\begin{center}
\begin{tabu} to \linewidth {X[1,c]X[1,c]X[1,c]|X[1,c]X[1,c]|X[1,c]}	\hhline{~~~---}
		&		&		&	\multicolumn{3}{c|}{\cellcolor{rowcolour}\textbf{IT Consultant}}	\\	\hhline{~~~---}
		&		&		&	\multicolumn{2}{c|}{\textbf{Together}}	&	\multicolumn{1}{c|}{\multirow{2}{*}{\textbf{Alone}}}	\\
		&		&		&	\multicolumn{1}{c}{\cellcolor{rowcolour}\textit{Good}}	&	\cellcolor{rowcolour}\textit{Bad}	&	\multicolumn{1}{c|}{}	\\	\hline
\multicolumn{1}{|c|}{\cellcolor{rowcolour}}	&	\multicolumn{1}{c}{\multirow{2}{*}{\textbf{Together}}}	&	\cellcolor{rowcolour}\textit{Good}	&	\$5,000; $p$=0.5	&	\$2,500; $p$=0.5	&	\multicolumn{1}{c|}{\cellcolor{rowcolour}\$5,000; $p$=0.5}	\\[-0.05em]
\multicolumn{1}{|c|}{\cellcolor{rowcolour}}	&	\multicolumn{1}{c}{}	&	\cellcolor{rowcolour}\textit{Bad}	&	\$2,500; $p$=0.5	&	\$0; $p$=0.5	&	\multicolumn{1}{c|}{\cellcolor{rowcolour}\$0; $p$=0.5}	\\[-0.1em]	\hhline{~-----}
\multicolumn{1}{|c|}{\cellcolor{rowcolour}\multirow{-3}{*}{\textbf{Musician}}}	&	\multicolumn{2}{c|}{\textbf{Alone}}	&	\cellcolor{rowcolour}\$5,000; $p$=0.5	&	\cellcolor{rowcolour}\$0; $p$=0.5	&		\\	\hhline{-----~}
\end{tabu}
\end{center}
\caption{Pooling risky incomes \label{table:riskyincome}}
\end{table}

When the two individuals pool and subsequently split their incomes, the outcome is given in the shaded region of the table. A good month for each yields \$5,000 each as before. But a good month for one, coupled with a bad month for the other, now yields \$2,500 each rather than zero or \$5,000. A bad month for each still yields a zero income to each. 

Together, their income is more stable than when alone. To confirm this, look at the lowest row in the table. When the incomes are shared, each partner will get \$5,000 one-quarter of the time (denoted by the probability $p$ equaling 0.25), zero one-quarter of the time, and \$2,500 half of the time. In contrast, when alone, they each get \$5,000 and zero half of the time. The latter outcomes are more dispersed, because they are more extreme than when income is shared; the probability of extreme outcomes is greater. Another way of saying this is that the variance is greater.

As a consequence of less dispersion, \textit{the shared outcome yields more utility on average}. This is because of diminishing marginal utility. In particular, if, each time one of the individuals got \$2,500, they were presented with the option of getting zero half of the time and \$5,000 half of the time, this latter option would make them worse off---for the simple reason that the gain in utility in going from \$2,500 to \$5,000 is less than the loss in utility in going to zero from \$2,500. This is the consequence of diminishing $MU$. In general, the more stable income stream is preferred and, since the sharing option gets them closer to a more stable income stream, it yields more utility on average. 

Figure~\ref{fig:utilextavg} develops this argument in graphical form. It relates total utility ($TU$) to income. In the risky scenario, the individual gets either zero dollars or \$5,000 with equal probability, and therefore gets utility of zero or OB with equal probability. Therefore, the average utility is OC (which is one half of OB, and sometimes called the `expected' utility). In contrast, where no risk is involved, and the individual always gets the average payout of \$2,500, utility is OA, which is greater than OC. In summary, where diminishing marginal utility exists, \textit{the average or expected utility of the event is less than the utility associated with the average or expected dollar outcome}.

% Figure 7.1
\begin{FigureBox}{0.12}{0.5}{25em}{Utility from extremes and averages \label{fig:utilextavg}}{If an individual gets \$0 half of the time and \$5,000 half the time, her average utility is OC -- an average of zero and OB. But if she gets \$2,500 every time then her utility is OA, which is greater than OC.}
\draw [dotted,thick]
	(0,5) node [mynode,left] {A} -- (25,5) -- (25,0) node [mynode,below] {\$2,500}
	(0,7.07107) node [mynode,left] {B} -- (50,7.07107) -- (50,0) node [mynode,below] {\$5,000};
% Total utility curve
\draw [tucolour,ultra thick,domain=0:54,name path=TU] plot (\x, {sqrt(\x)}) node [black,mynode,right] {$TU$};
% axes
\draw [thick, -] (0,10) node (yaxis) [mynode1,above] {Total\\utility} -- (0,0) node [mynode,left] {O} -- (55,0) node (xaxis) [right] {\$};
% point of TU axis
\draw [thick,-] (-1,3.535534) -- (0,3.535534) node [mynode,left] {C=1/2(OB)};
% paths to intersect with TU curve
\path [name path=Aline] (0,5) -- +(55,0);
\path [name path=Bline] (0,7.07107) -- +(55,0);
% intersection of A and B lines with TU curve
\draw [name intersections={of=TU and Aline, by=A},name intersections={of=TU and Bline, by=B}]
	[dotted,thick] (yaxis |- A) node [mynode,left] {A} -| (xaxis -| A) node [mynode,below] {\$2,500}
	[dotted,thick] (yaxis |- B) node [mynode,left] {B} -| (xaxis -| B) node [mynode,below] {\$5,000};
\end{FigureBox}

In this example, individuals pool their risk. But while \terminology{pooling independent risks }is the key to insurance, such pooling does not work when all individuals face the same risk: If the IT consultant always had a good month whenever the musician had a good month, and conversely in bad months, no change in outcomes would result from their deciding to pool their incomes.

\begin{DefBox}
\textbf{Risk pooling}: a means of reducing risk and increasing utility by aggregating or pooling multiple independent risks.
\end{DefBox}

\subsection*{Risk spreading}

Risk spreading is a different form of risk management. Imagine that an oil supertanker has to be insured and the owner approaches one insurer. In the event of the tanker being ship- wrecked the damage caused by the resulting oil spill would be catastrophic -- both to the environment and the insurance company, which would have to pay for the damage. In this instance the insurance company does not benefit from the law of large numbers -- it is not insuring thousands of such tankers and therefore would find it difficult to balance the potential claims with the annual insurance premiums. As a consequence such insurers generally spread the potential cost among other insurers, and this is what is called \terminology{risk spreading}. 

\begin{DefBox}
\textbf{Risk spreading}: spreads the risk of a venture among multiple sub insurers.
\end{DefBox}

The logic is similar to participating in an office Super-Bowl lottery. Most people are willing to risk the loss of \$5 or \$10, but very few would participate if the required bet were \$100. The world's major insurers -- such as Lloyd's of London has hundreds of syndicates who each take on a small proportion of a big risk. These syndicates may again choose to subdivide their share among others, until the big risk becomes widely spread. In this way it is possible to insure against almost any eventuality, no matter how large. 

\subsection*{Risk, the firm and the investor}

Let us return to our starting point: individuals like to get a high return on their investment, but dislike an excess of risk -- that is to say they prefer as little variation in their return as possible, other things equal. 

Evidently, a firm is a risky operation, and consequently investing in a firm bears risks. Risk management of the type we have just explored is critical to inducing risk-averse individuals to invest in corporations whose returns are uncertain. Let us see how.