\section{Profit, ownership and corporate goals}\label{sec:ch7sec2}

As economists, we believe that profit maximization accurately describes a typical firm's objective. However, since large firms are not run by their owners but by their executives or agents, it is frequently hard for the shareholders to know exactly what happens within a company. Even the board of directors---the guiding managerial group---may not be fully aware of the decisions, strategies, and practices of their executives and managers. Occasionally things go wrong, and managers decide to follow their own interests rather than the interests of the company. In technical terms, the interests of the corporation and its shareholders might not be aligned with the interests of its managers. For example managers might have a short horizon and take steps to increase their own income in the short term, knowing that they will move job before the long-term impacts of their decisions impact the firm. 

At the same time, the marketplace for the ownership of corporations exerts a certain discipline: If firms are not as productive or profitable as possible, they may become \textit{subject to takeover} by other firms. Fear of such takeover can induce executives and boards to maximize profits. 

The shareholder-manager relationship is sometimes called a \terminology{principal-agent relationship}, and it can give rise to a principal-agent problem. If it is costly or difficult to monitor the behaviour of an agent because the agent has additional information about his own performance, the principal may not know if the agent is working to achieve the firm's goals. This is the \terminology{principal-agent problem}. 

\begin{DefBox}
\textbf{Principal or owner}: delegates decisions to an agent, or manager.

\textbf{Agent}: usually a manager who works in a corporation and is directed to follow the corporation's interests.

\textbf{Principal-agent problem}: arises when the principal cannot easily monitor the actions of the agent, who therefore may not act in the best interests of the principal.
\end{DefBox}

In an effort to deal with such a challenge, corporate executives frequently get bonuses or \terminology{stock options} that are related to the overall profitability of their firm. Stock options usually take the form of an executive being allowed to purchase the company's stock in the future -- but at a price that is predetermined. If the company's profits do increase, then the price of the company's stock will reflect this and increase likewise. Hence the executive has an incentive to work with the objective of increasing profits because that will enable him to buy the company stock in the future at a lower price than it will be worth.

\begin{DefBox}
\textbf{Stock option}: an option to buy the stock of the company at a future date for a fixed, predetermined price.
\end{DefBox}

The threat of takeover and the structure of rewards, together, imply that the assumption of profit maximization is a reasonably robust one.