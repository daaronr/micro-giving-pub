\section{Cartels: acting like a monopolist} \label{sec:cartel}

A \terminology{cartel} is a group of suppliers that colludes to operate like a monopolist. The most famous cartel in modern times is certainly the oil cartel formed by the members of the Organization of Oil Exporting Countries (OPEC). This cartel first flexed its muscles seriously in 1973, by increasing the world price of oil from \$3 per barrel to \$10 per barrel. The result was to transfer billions of dollars from the energy-importing nations in Europe and the US (and partly North America) to OPEC members -- the demand for oil is relatively inelastic, hence an increase in price increases total expenditures.

\begin{DefBox}
A \textbf{cartel} is a group of suppliers that colludes to operate like a monopolist.
\end{DefBox}

A second renowned cartel is managed by De Beers, which controls a large part of the world's diamond supply.  A third is Major League Baseball in the US, which is excluded from competition law. Lesser-known cartels, but in some cases very effective, are those formed by the holders of taxi licenses in many cities throughout the world: entry is extremely costly. By limiting the right to enter the taxi industry, the incumbents can charge a higher price than if entry to the industry were free. 

Some cartels are sustained through violence, and frequently wars break out between competing cartels or groups who want to sustain their market power. Drug gangs in large urban areas frequently fight for hegemony over distribution. And the recent drug wars in Mexico between rival cartels had seen tens of thousands on individuals killed as of 2012. 

Some cartels are successful and stable, others not so, in the sense that individual members of the cartel may decide to `cheat' -- by undercutting the agreed price.  To understand the dynamics of cartels consider Figure~\ref{fig:cartelindustry}. Several producers, with given production capacities, come together and agree to restrict output with a view to increasing price and therefore profit. Each firm has a $MC$ curve, and the industry supply is defined as the sum of these marginal cost curves, as illustrated in Figure~\ref{fig:industrysupply}. We could think of the resulting cartel as one in which there is a single supplier with many different plants -- a multi-plant monopolist. To maximize profits this organization will choose an output level $Q_m$ where the $MR$ equals the $MC$. In contrast, if these firms act competitively the output chosen will be $Q_c$. The competitive output yields no supernormal profit, whereas the monopoly/cartel output does.

% Figure 10.13 (called 10.9 in original text)
\input{figures/ch10/ch10fig13}

The cartel results in a deadweight loss equal to the area ABF, just as in the standard monopoly model.

\subsection*{Cartel instability}

However, cartels tend to be unstable, and the degree of instability depends on the authority that the governing body of the cartel can exercise over its members, and also the degree of information it has on the operations of its members. The reason for instability lies in the fact that each individual member of the cartel has an incentive to increase its output, because the monopoly price that the cartel attempts to sustain exceeds the cost of producing a marginal unit of output. In Figure~\ref{fig:cartelindustry} each firm has a $MC$ of output equal to \$F when the group collectively produces the output $Q_m$. Yet any firm that brings its output to market at the price $P_m$ will make a profit of AF on an additional unit of output -- \textit{provided the other members of the cartel agree to restrict their output}. Since each firm faces the same incentive to increase output, it is difficult to restrain all members from doing so.

Individual members are more likely to abide by the cartel rules if the organization can sanction them for breaking the supply-restriction agreement. Alternatively, if the actions of individual members are not observable by the organization, then the incentive to break ranks may be too strong for the cartel to sustain its monopoly power. 

Cartels within individual economies are almost universally illegal. Yet at the international level there exists no governing authority to limit such behaviour. In practice, governments are unwilling to see their own citizens and consumers being `gouged', but are relatively unconcerned if their national or multinational corporations are willing and successful in gouging the consumers of other economies! We will see in Chapter~\ref{chap:government} that Canada's Competition Act forbids the formation of cartels, as it forbids many other anti-competitive practices.

\begin{ApplicationBox}{The taxi cartel \label{app:taxicartel}}
While we tend to think of cartels as small groups of suppliers, in fact they often contain thousands of members and are maintained by legal entry barriers. City taxis are an example of such a formation: Entry is limited by the city or province or state, and fares are maintained at a higher level as a consequence of the resulting lower supply. A secondary market then develops for licenses -- medallions, in which the city may offer new medallions through auction, or existing owners may exit and sell their medallions. Restricted entry characterizes most of Canada's major cities, with the result that new medallions frequently generate in excess of \$100,000 from the buyer. New York and Boston medallions traded at close to one million dollars in 2012.

\bigskip
But a recent start-up company is aiming to change all of that: \textit{Uber} arrived on the international scene. This company developed a smart-phone app that links demanders for taxis with drivers who are not part of the traditional taxi companies. \textit{Uber} set up operations in many of the world's major cities and has succeeded in taking a small part of the taxi business away from the traditional operators. One result is that the price of taxi medallions on the open market in large US cities has fallen by about 20\%. Not surprisingly, the traditional taxi companies are charging that \textit{Uber} operators are violating the accepted rules governing the taxi business, and have launched legal suits against \textit{Uber}.

\bigskip 
\textit{Uber} is part of what is called the `sharing economy'. Participants operate with very little traditional capital. For example, suppliers may use an online site to rent a spare bedroom in their house to visitors to their city (Airbnb), and thus compete with hotels. The main capital in this business is in the form of the information technology that links potential buyers to potential sellers.

\bigskip
See \href{www.fcpp.org}{www.fcpp.org} for Canada, and for New York:

\bigskip
\href{www.nyc.gov/html/tlc/downloads/pdf/press_release_medallion_auction.pdf}{www.nyc.gov/html/tlc/downloads/pdf/press\_release\_medallion\_auction.pdf}
\end{ApplicationBox}