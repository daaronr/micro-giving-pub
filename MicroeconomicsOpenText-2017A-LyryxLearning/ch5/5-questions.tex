\newpage
\section*{Exercises for Chapter~\ref{chap:welfare}}

\begin{enumialphparenastyle}

% Solutions file for exercises opened
\Opensolutionfile{solutions}[solutions/ch5ex]

\begin{ex}\label{ex:ch5ex1}
Four teenagers live on your street. Each is willing to shovel snow from one driveway each day. Their ``willingness to shovel'' valuations (supply) are: Jean, \$10; Kevin, \$9; Liam, \$7; Margaret, \$5. Several households are interested in having their driveways shoveled, and their willingness to pay values (demand) are: Jones, \$8; Kirpinsky, \$4; Lafleur, \$7.50; Murray, \$6.
\begin{enumerate}
	\item	Draw the implied supply and demand curves as step functions.
	\item	How many driveways will be shoveled in equilibrium?
	\item	Compute the maximum possible sum for the consumer and supplier surpluses.
	\item	If a new (wealthy) family arrives on the block, that is willing to pay \$12 to have their driveway cleared, recompute the answers to parts (a), (b), and (c).
\end{enumerate}
\begin{sol}
\begin{enumerate}
	\item	The step functions are similar to those in Figure~\ref{fig:apartmentmarket}. In ascending order, Margaret is the first supplier, Liam the second, etc. You must also order the demanders in descending order.
	\item	Two: Margaret and Liam will supply, while Jones and Lafleur will purchase. The third highest demander (Murray) is willing to pay \$6, while the third supplier is willing to supply only if the price is \$9. Hence there is no third unit supplied.
	\item	The equilibrium price will lie in the range \$7.0-\$7.5. So	let us say it is \$7. The consumer surplus of each buyer is therefore \$1 and \$0.5. The supplier surpluses are zero and \$2.
	\item	Two driveways will still be cleared. The highest value buyers are now willing to pay \$12 and \$8. The third highest value buyer is willing to pay \$7.0. But on the supply side the third supplier still supplies only if he gets \$9. Therefore two units will be supplied. If the price remains at \$7 (it could fall in the range between \$7 and \$8) the consumer surpluses are now \$5 and \$1, and the	supplier surpluses remain the same.
\end{enumerate}
\end{sol}
\end{ex}

\begin{ex}\label{ex:ch5ex2}
Consider a market where supply and demand are given by $P=10$ and $P=34-Q$ respectively.
\begin{enumerate}
	\item	Illustrate the market geometrically, and compute the equilibrium quantity.
	\item	Impose a tax of \$2 per unit on the good so that the supply curve is now $P=12$. Calculate the new equilibrium quantity, and illustrate it in your diagram.
	\item	Calculate the tax revenue generated, and also the deadweight loss.
\end{enumerate}
\begin{sol}
\begin{enumerate}
	\item	The supply curve is horizontal at a price of \$10. The demand curve price intercept is \$34 and the quantity intercept is 34. The	equilibrium quantity is 24.
	\item	The new supply curve is $P=12$. Substituting this price into the demand curve yields $Q=22$.
	\item	Tax revenue is \$44: each of the 22 units sold yields \$2. The deadweight loss is the standard triangular area in Figure~\ref{fig:taxationlaboursupply}. It is \$2.
\end{enumerate}
\end{sol}
\end{ex}

\begin{ex}\label{ex:ch5ex3}
Redo Exercise~\ref{ex:ch5ex2} with the demand curve replaced by $P=26-(2/3)Q$.
\begin{enumerate}
	\item	Is this new demand curve more or less elastic than the original at the equilibrium?
	\item	What do you note about the relative magnitudes of the DWL and tax revenue estimates here, relative to the previous question?
\end{enumerate}
\begin{sol}
\begin{enumerate}
	\item	With a supply curve given by $P=10$, the new demand curve yields an equilibrium quantity of 24 once again. The new demand curve is `flatter' at the equilibrium than the original, indicating that it is more elastic.
	\item	With a tax of \$2 imposed the new equilibrium quantity is 21. Hence tax revenue is \$42. The DWL is \$3.
\end{enumerate}
\end{sol}
\end{ex}

\begin{ex}\label{ex:ch5ex4}
Next, consider an example of DWL in the labour market. Suppose the demand for labour is given by the fixed gross wage $W=\$16$. The supply is given by $W=0.8L$.
\begin{enumerate}
	\item	Illustrate the market geometrically.
	\item	Calculate the equilibrium amount of labour supplied, and the supplier surplus.
	\item	Suppose a wage tax that reduces the wage to $W=\$12$ is imposed. By how much is the supplier's surplus reduced at the new equilibrium?
\end{enumerate}
\begin{sol}
\begin{enumerate}
	\item	The supply curve goes through the origin and the demand curve is horizontal at $W=\$16$ -- see diagram below.
	\item	The equilibrium amount of labour supplied is 20 units. The supplier surplus is the area above the supply curve below the
	equilibrium price$=\$160$.
	\item	At a net wage of \$12, labour supplied falls to 15. The downward shift in the wage reduces the quantity supplied. The new supplier surplus is the triangular area bounded by $W=12$ and $L=15$. Its value is therefore \$90.
\end{enumerate}
\begin{center}
	\begin{tikzpicture}[background color=figurebkgdcolour,use background,xscale=0.4,yscale=0.5]
	\draw [thick] (0,10) node (yaxis) [mynode1,above] {Wage} |- (15,0) node (xaxis) [mynode1,right] {$L$};
	\draw [demandcolour,ultra thick,name path=W16] (0,5) -- +(14.5,0) node [black,mynode,right] {$W=16$};
	\draw [supplycolour,ultra thick,domain=0:14,name path=W08L] plot (\x, {0.666*\x+0.333}) node [black,mynode,right] {$W=0.8L$};
	\draw [name intersections={of=W16 and W08L, by=E}]
	[dotted,thick] (E) node [mynode,above] {E} node [mynode,below left=0.75em and 2.75em] {Supplier's\\surplus area} -- (xaxis -| E) node [mynode,below] {20};
	\end{tikzpicture}
\end{center}
\end{sol}
\end{ex}

\begin{ex}\label{ex:ch5ex5}
Governments are in the business of providing information to potential buyers. The first serious provision of information on the health consequences of tobacco use appeared in the United States Report of the Surgeon General in 1964.
\begin{enumerate}
	\item	How would you represent this intervention in a supply and demand for tobacco diagram?
	\item	Did this intervention ``correct'' the existing market demand?
\end{enumerate}
\begin{sol}
\begin{enumerate}
	\item	The demand curve shifts inwards.
	\item	Yes, because consumers previously did not have full information about the product.
\end{enumerate}
\end{sol}
\end{ex}

\begin{ex}\label{ex:ch5ex6}
In deciding to drive a car in the rush hour, you think about the cost of gas and the time of the trip.
\begin{enumerate}
	\item	Do you slow down other people by driving?
	\item	Is this an externality, given that you yourself are suffering from slow traffic?
\end{enumerate}
\begin{sol}
\begin{enumerate}
	\item	Yes.
	\item	Yes, because the congestion effect is not incorporated into the price of driving.
\end{enumerate}
\end{sol}
\end{ex}

\begin{ex}\label{ex:ch5ex7}
Suppose that our local power station burns coal to generate electricity. The demand and supply functions for electricity are given by $P=12-0.5Q$ and $P=2+0.5Q$, respectively. However, for each unit of electricity generated, there is an externality. When we factor this into the supply side of the market, the real social cost is increased, and the supply curve is $P=3+0.5Q$.
\begin{enumerate}
	\item	Find the free market equilibrium and illustrate it geometrically.
	\item	Calculate the efficient (i.e. socially optimal) level of production.
\end{enumerate}
\begin{sol}
\begin{enumerate}
	\item	The free market equilibrium is obtained by equating demand and private-cost supply curves: $Q=10$, $P=\$7$.
	\item	Using the social supply curve yields and equilibrium of $Q=6$. These answers are illustrated graphically in Figure~\ref{fig:negextineff}.
\end{enumerate}
\end{sol}
\end{ex}

\begin{ex}\label{ex:ch5ex8}
Evan rides his mountain bike down Whistler each summer weekend. The utility value he places on each kilometre ridden is given by $P=4-0.02Q$, where $Q$ is the number of kilometres. He incurs a cost of \$2 per kilometre in lift fees and bike depreciation.
\begin{enumerate}
	\item	How many kilometres will he ride each weekend? [Hint: Think of this ``value'' equation as demand, and this ``cost'' equation as a (horizontal) supply.]
	\item	But Evan frequently ends up in the local hospital with pulled muscles and broken bones. On average, this cost to the Canadian taxpayer is \$0.50 per kilometre ridden. From a societal viewpoint, what is the efficient number of kilometres that Evan should ride each weekend?
\end{enumerate}
\begin{sol}
\begin{enumerate}
	\item	Equating demand to price yields $Q=100$km. See figure below.
	\item	Using a price of \$2.5 rather than \$2.0 yields a quantity of 75km.
\end{enumerate}
\begin{center}
	\begin{tikzpicture}[background color=figurebkgdcolour,use background,xscale=0.3,yscale=0.25]
	\draw [thick] (0,20) node (yaxis) [mynode1,above] {$P$} |- (25,0) node (xaxis) [mynode1,right] {$Q$};
	\draw [ultra thick,demandcolour,name path=D] (0,15) node [mynode,left,black] {4} -- (20,0) node [mynode,below,black] {200};
	\draw [ultra thick,supplycolour,name path=P2] (0,7.5) -- +(20,0) node [mynode,right,black] {$P=2$};
	\draw [ultra thick,dashed,supplycolour,name path=P25] (0,9) -- +(20,0) node [mynode,right,black] {$P=2.5$ (Social cost)};
	\draw [name intersections={of=D and P2, by=E},name intersections={of=P25 and D, by=SO}]
	[dotted,thick] (E) -- (xaxis -| E) node [mynode,below] {100}
	[dotted,thick] (SO) -- (xaxis -| SO) node [mynode,below] {75};
	\draw [<-,thick,shorten <=1mm] (E) -- +(3,3) node [mynode,right] {Equilibrium};
	\draw [<-,thick,shorten <=1mm] (SO) -- +(3,3) node [mynode,right] {Social optimum};
	\end{tikzpicture}
\end{center}
\end{sol}
\end{ex}

\begin{ex}\label{ex:ch5ex9}
Your local dry cleaner, Bleached Brite, is willing to launder shirts at its cost of \$1.00 per shirt. The neighbourhood demand for this service is $P=5-0.005Q$.
\begin{enumerate}
	\item	Illustrate and compute the market equilibrium.
	\item	Suppose that, for each shirt, Bleached Brite emits chemicals into the local environment that cause \$0.25 damage per shirt. This means the full cost of each shirt is \$1.25. Calculate the socially optimal number of shirts to be cleaned.
\end{enumerate}
\begin{sol}
\begin{enumerate}
	\item	The supply curve is horizontal at $P=\$1$. The demand curve has	a price intercept of 5 and a quantity intercept of 1000. The	equilibrium quantity is 800.
	\item	The socially optimal quantity is obtained by recognizing that the social cost is \$1.25 rather than \$1.0. Here $Q^*=750$.
\end{enumerate}
\end{sol}
\end{ex}

\begin{ex}\label{ex:ch5ex10}
The supply curve for agricultural labour is given by $W=6+0.1L$, where $W$ is the wage (price per unit) and $L$ the quantity traded. Employers are willing to pay a wage of \$12 to all workers who are willing to work at that wage; hence the demand curve is $W=12$.
\begin{enumerate}
	\item	Illustrate the market equilibrium, and compute the equilibrium wage (price) and quantity of labour employed.
	\item	Compute the supplier surplus at this equilibrium.
\end{enumerate}
\begin{sol}
\begin{enumerate}
	\item	The demand curve is horizontal at $P=\$12$. The supply curve slopes upwards with a price intercept of \$6. Equilibrium is $L=60$.
	\item	Surplus is the area beneath the demand curve above the supply curve$=\$180$.
\end{enumerate}
\end{sol}
\end{ex}

\begin{ex}\label{ex:ch5ex11}
The demand for ice cream is given by $P=24-Q$ and the supply curve by $P=4$.
\begin{enumerate}
	\item	Illustrate the market equilibrium, and compute the equilibrium price and quantity.
	\item	Calculate the consumer surplus at the equilibrium.
	\item	As a result of higher milk prices to dairy farmers the supply conditions change to $P=6$. Compute the new quantity traded, and calculate the loss in consumer surplus.
\end{enumerate}
\begin{sol}
\begin{enumerate}
	\item	The equilibrium here is $Q=20$, $P=\$4$.
	\item	Consumer surplus is \$200.
	\item	The new quantity is $Q=18$ and $CS=\$162$.
\end{enumerate}
\end{sol}
\end{ex}

\begin{ex}\label{ex:ch5ex12}
Two firms A and B, making up a sector of the economy, emit pollution (pol) and have marginal abatement costs: $MA_A=24-pol$ and $MA_B=24-(1/2)pol$. So the total abatement curve for this sector is given by $MA=24-(1/3)pol$. The marginal damage function is constant at a value of \$12 per unit of pollution emitted: $MD=\$12$.
\begin{enumerate}
	\item	Draw the $MD$ and market-level $MA$ curves and establish the efficient level of pollution for this economy.
\end{enumerate}
\begin{sol}
	The marginal abatement curves are essentially the demand for pollution rights on the part of the producers. If we sum these curves horizontally it is easy to see that the price intercept	remains at \$24 and the horizontal intercept becomes 72 ($=24+48$). Hence the total demand for abatement becomes $MA=24-(1/3)pol$. The $MD$ function is $MD=\$12$. This is the `supply' function because firms are	able to buy the pollution rights at this price. The efficient level of pollution is 36 units.
	
\end{sol}
\end{ex}

\begin{ex}\label{ex:ch5ex13}
In Exercise~\ref{ex:ch5ex12}, if each firm is permitted to emit half of the efficient level of pollution, illustrate your answer in a diagram which contains the $MA_A$ and $MA_B$ curves.
\begin{enumerate}
	\item	With each firm producing this amount of pollution, how much would it cost each one to reduce pollution by one unit?
	\item	If these two firms can freely trade the right to pollute, how many units will they (profitably) trade?
\end{enumerate}
\begin{sol}
\begin{enumerate}
	\item	See diagram below. The answers are \$15 and \$6.
	\item	As long as the abatement costs are different it is profitable to trade. With a total number of permits available of 36 units, the amount they trade will depend upon the  price they agree upon. Provided the price lies between \$6 and \$15 they have an incentive to trade.
\end{enumerate}
\begin{center}
	\begin{tikzpicture}[background color=figurebkgdcolour,use background,xscale=0.18,yscale=0.2]
	\draw [thick] (0,25) node (yaxis) [mynode1,above] {\$} |- (50,0) node (xaxis) [mynode1,right] {$pol$};
	\draw [demandcolour,ultra thick,name path=MAA] (0,24) node [mynode,left,black] {24} -- node [mynode,above right,black,pos=0.9] {$MA_A$} (24,0);
	\draw [demandcolour,ultra thick,name path=MAB] (0,24) -- node [mynode,above right,black,pos=0.9] {$MA_B$} (48,0);
	\draw [supplycolour,ultra thick,name path=Pol] (18,0) -- +(0,24) node [mynode,above,black] {$pol=18$};
	\draw [name intersections={of=MAA and Pol, by=A},name intersections={of=MAB and Pol, by=B}]
	[dotted,thick] (yaxis |- A) node [mynode,left] {6} -- (A)
	[dotted,thick] (yaxis |- B) node [mynode,left] {15} -- (B);
	\end{tikzpicture}
\end{center}
\end{sol}
\end{ex}

\begin{ex}\label{ex:ch5ex14}
Once again, in Exercise~\ref{ex:ch5ex13}, suppose that the government's policy is to allow firms to pollute provided that they purchase a permit valued at \$10 per unit emitted (rather than allocating a pollution quota to each firm).
\begin{enumerate}
	\item	How many units of pollution rights would be purchased and by the two participants in this market?
\end{enumerate}
\begin{sol}
	Firm A would purchase 14 units and Firm B would purchase 28. Clearly the lower price price means that the total amount of pollution emitted is greater.
	
\end{sol}
\end{ex}

\begin{ex}\label{ex:ch5ex15}
The market demand for vaccine XYZ is given by $P=36-Q$ and the supply conditions are $P=20$. There is a positive externality associated with being vaccinated, and the real societal value is known and given by $P=36-(1/2)Q$.
\begin{enumerate}
	\item	What is the market solution to this supply and demand problem?
	\item	What is the socially optimal number of vaccinations?
	\item	If we decide to give the supplier a given dollar amount per vaccination supplied in order to reduce price and therefore increase the number of vaccinations to the social optimum, what would be the dollar value of that per-unit subsidy?
\end{enumerate}
\begin{sol}
\begin{enumerate}
	\item	The market solution is obtained by equating the market demand and supply. This yields $Q=16$ and $P=\$20$.
	\item	The socially optimal amount takes account of the fact that there are positive externalities. The demand curve that reflects these externalities is above the private demand curve. Hence the socially optimal equilibrium is at a greater output $Q=32$.
	\item	To induce a demand of 32 units in the private marketplace the price would have to be \$4. Hence the subsidy per unit would be \$16.
\end{enumerate}
\end{sol}
\end{ex}

\begin{ex}\label{ex:ch5ex16}
In Exercise~\ref{ex:ch5ex15}, suppose that we give buyers the subsidy instead of giving it to the suppliers. By how much would the demand curve have to shift upward in order that the socially optimal quantity is realized?
\begin{sol}
	The demand curve would have to be shifted upwards to the point where it intersects the supply curve at 32 units. The new price intercept would have to be \$52. Hence the subsidy would again be \$16.
	
\end{sol}
\end{ex}

\begin{ex}\label{ex:ch5ex17}
The demand and supply curves in a regular market (no externalities) are given by $P=42-Q$ and $P=0.2Q$.
\begin{enumerate}
	\item	Solve for the equilibrium price and quantity.
	\item	A percentage tax of 100\% is now levied on each unit supplied. Hence the form of the new supply curve $P=0.4Q$. Find the new market price and quantity.
	\item	How much per unit is the supplier paid?
	\item	Compute the producer and consumer surpluses after the imposition of the tax and also the DWL.
\end{enumerate}
\begin{sol}
\begin{enumerate}
	\item	Equating the functions yields $Q=35$, $P=\$7$.
	\item	The solutions becomes $Q=30$, $P=\$12$.
	\item	The consumer pays \$12, the supplier gets half of this.
	\item	Using the customary triangle formulas yields $CS=\$450$; $PS=\$90$; $DWL=\$15$.
\end{enumerate}
\end{sol}
\end{ex}

% Closes solutions file for this chapter
\Closesolutionfile{solutions}

\end{enumialphparenastyle}