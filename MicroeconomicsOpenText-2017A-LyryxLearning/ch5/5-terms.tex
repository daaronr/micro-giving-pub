\newpage
\markboth{Key Terms}{Key Terms}
	\addcontentsline{toc}{section}{Key terms}
	\section*{\textsc{Key Terms}}
\begin{keyterms}
\textbf{Efficiency} addresses the question of how well the economy's resources are used and allocated.

\textbf{Equity} deals with how society's goods and rewards are, and should be, distributed among its different members, and how the associated costs should be apportioned.

\textbf{Consumer surplus} is the excess of consumer willingness to pay over the market price.

\textbf{Supplier or producer surplus} is the excess of market price over the reservation price of the supplier.

\textbf{Efficient market}: maximizes the sum of producer and consumer surpluses.

\textbf{Tax wedge} is the difference between the consumer and producer prices.

\textbf{Revenue burden} is the amount of tax revenue raised by a tax.

\textbf{Excess burden} of a tax is the component of consumer and producer surpluses forming a net loss to the whole economy.

\textbf{Deadweight loss} of a tax is the component of consumer and producer surpluses forming a net loss to the whole economy.

\textbf{Distortion} in resource allocation means that production is not at an efficient output, or a given output is not efficiently allocated.

\textbf{Externality} is a benefit or cost falling on people other than those involved in the activity's market. It can create a difference between private costs or values and social costs or values.

\textbf{Corrective tax} seeks to direct the market towards a more efficient output.

\textbf{Greenhouse gases} that accumulate excessively in the earth's atmosphere prevent heat from escaping and lead to global warming.

\textbf{Marginal damage curve} reflects the cost to society of an additional unit of pollution.

\textbf{Marginal abatement curve} reflects the cost to society of reducing the quantity of pollution by one unit.

\textbf{Tradable permits} and \textbf{corrective/carbon taxes} are market-based systems aimed at reducing GHGs.

\textbf{Horizontal equity} is the equal treatment of similar individuals.

\textbf{Vertical equity} is the different treatment of different people in order to reduce the consequences of these innate differences.

\textbf{Intergenerational equity} requires a balancing of the interests and well-being of different generations and cohorts.
\end{keyterms}