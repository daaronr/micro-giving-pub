\section{Efficient market outcomes}\label{sec:ch5sec3}

The definition of the surplus measures is straightforward: Once we have the demand and supply curves, the area between each one and the equilibrium price can be calculated. With straight-line functions, these areas involve triangles. But where does the notion of market efficiency enter? Let us pursue the example.

In addition to these city apartments, there are many others in the suburbs that do not have the desirable ``proximity to downtown'' characteristic. There are also many more demanders in the market for living space than the number who rented at \$500 in the city. Who are these other individuals? Clearly they place a lower value on city apartments than the individuals who are willing to pay at least \$500.

The equilibrium price of \$500 in Figure~\ref{fig:apartmentmarket} has two implications. First, individuals who place a lower value on a city apartment must seek accommodation elsewhere. Second, suppliers who have a reservation price above the equilibrium price will not participate. This implies that an \terminology{efficient market} maximizes the sum of producer and consumer surpluses. Here is why.

\begin{DefBox}
An \textbf{efficient market} maximizes the sum of producer and consumer surpluses.
\end{DefBox}

Instead of a freely functioning market, imagine that the city government rents all apartments from suppliers at the price of \$500 per unit, but decides to allocate the apartments to tenants in a lottery (we can imagine the government getting the money to pay for the apartments from tax revenue). By doing this, many demanders who place a low value on a city apartment would end up living in one, and other individuals, who were not so fortunate in the lottery, would not obtain an apartment, even if they valued one highly. Suppose, then, that Frank gets an apartment in the lottery and Cathy does not. This outcome would not be efficient, because there are further gains in surplus to be had. Frank and Cathy can now strike a private deal so that \textit{both} gain.

If Frank agrees to sublet to Cathy at a price between their respective valuations of \$400 and \$700---say \$600---he will gain \$200 and she will gain \$100. This is because Frank values the apartment only at \$400, but now obtains \$600. Cathy values it at \$700 but pays only \$600. The random allocation of apartments, therefore, is not efficient, because further gains from trade are possible. In contrast, \textit{the market mechanism, in which suppliers and demanders freely trade, leaves no scope for additional trades that would improve the well-being of participants}.

It is frequently useful to characterize market equilibrium in terms of the behaviour of \textit{marginal} participants---the very last buyer and the very last supplier, or the very last unit supplied and demanded. In addition, we will continue with the assumption that the supply curve represents the full cost of each unit of production. It follows that, at the equilibrium, the value placed on the last unit purchased (as reflected in the demand curve) equals the cost of supplying that unit. If one more unit were traded, we can see from Figure~\ref{fig:measuringsurplus} that the value placed on that additional unit (as represented by the demand curve) would be less than its cost of production. This would be a poor use of society's resources. Phrased another way, resources would not be used efficiently unless the cost of the last unit equaled the value placed on it.

Before applying the concept of efficiency, and the surpluses it embodies, students should note that we have invoked some assumptions. For example, if individual incomes change, the corresponding market demand curve changes, and any market equilibrium will then depend on the new distribution of incomes. 