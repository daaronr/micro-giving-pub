\section{Equity, justice, and efficiency}\label{sec:ch5sec8}

Our discussion of environmental challenges in the modern era illustrates starkly the tradeoffs that we face \textit{inter-generationally}: disregarding the impacts of today's behaviour can impact future generations. Clearly there is a question here of equity. 

Economists use several separate notions of equity in formulating policy: \terminology{horizontal equity}, \terminology{vertical equity}, and \terminology{inter-generational equity}. Horizontal equity dictates, for example, that people who have the same income should pay the same tax, while the principle of vertical equity dictates that people with more income should pay more tax, and perhaps a higher rate of tax. Inter-generational equity requires that the interests of different cohorts of individuals---both those alive today and those not yet born---should be balanced by ethical principles.

\begin{DefBox}
\textbf{Horizontal equity} is the equal treatment of similar individuals.

\textbf{Vertical equity} is the different treatment of different people in order to reduce the consequences of these innate differences.

\textbf{Intergenerational equity} requires a balancing of the interests and well-being of different generations and cohorts.
\end{DefBox}

Horizontal equity rules out discrimination between people whose economic characteristics and performance are similar. Vertical equity is more strongly normative. Most people agree that horizontal equity is a good thing. In contrast, the \textit{extent} to which resources should be redistributed from the ``haves'' to the ``have-nots'' to increase vertical equity is an issue on which it would be difficult to find a high degree of agreement. 

People have different innate abilities, different capacities, and different wealth. These differences mean people earn different incomes in a market economy. They also affect the pattern of consumer demand. Brazil, with a very unequal distribution of income and wealth, has a high demand for luxuries such as domestic help. In more egalitarian Denmark, few can afford servants. Different endowments of ability, capital, and wealth thus imply different demand curves and determine different equilibrium prices and quantities. In principle, \textit{by varying the distribution of earnings, we could influence the outcomes in many of the economy's markets}.

This is an important observation, because it means that \textit{we can have many different efficient outcomes in each of the economy's markets} when considered in isolation. The position of a demand curve in any market may depend upon how incomes and resources are distributed in the economy. Accordingly, when it is proposed that the demand curve represents the ``value'' placed on a good or service, we should really think of this value as a measure of willingness to pay, \textit{given the current distribution of income}.

For example, the demand curve for luxury autos would shift downward if a higher tax rate were imposed on those individuals at the top end of the income distribution. Yet the auto market could be efficient with either a low or high set of income taxes. Let us pursue this example further in order to understand more fully that the implementation of a degree of redistribution from rich to poor involves an equity--efficiency trade-off.

\begin{ApplicationBox}{Equity, ability, luck, and taxes \label{app:equityabilitylucktax}}
John Rawls, a Harvard philosopher who died recently, has been one of the most influential proponents of redistribution in modern times. He argued that much of the income difference we observe between individuals arises on account of their inherited abilities, social status, or good fortune. Only secondarily, he proposes, are income differences due to similar individuals making different work choices.

\bigskip
If this view is accurate, he challenges us to think today of a set of societal rules we would adopt, not knowing our economic status or ability in a world that would begin tomorrow! He proposes that, in such an experiment, we could collectively adopt a set of rules favouring the less fortunate, in particular those at the very bottom of the income heap.
\end{ApplicationBox}

\subsection*{Equity versus efficiency}

Figure~\ref{fig:equityefflabourmarket} describes the market for high-skill labour. With no income taxes, the equilibrium labour supply and wage rate are given by $(L_0,W_0)$. If a tax is now imposed that reduces the gross wage $W_0$ to $W_{t1}$, the consequence is that less labour is supplied and there is a net loss in surplus equal to the dollar amount $E_0E_1$A. This is the efficiency loss associated with raising government revenue equal to $W_0$A$E_1W_{t1}$. Depending on how this money is spent, society may be willing to trade off some efficiency losses in return for redistributive gains. 

% Figure 5.8
\begin{FigureBox}{0.4}{0.5}{25em}{Equity versus efficiency in the labour market \label{fig:equityefflabourmarket}}{Doubling the wage tax on labour from $(W_0-W_{t1})$ to $(W_0-W_{t2})$ increases the DWL from A$E_0E_1$ to B$E_0E_2$. The DWL more than doubles -- in this case it quadruples when the tax doubles.}
% demand lines
\draw [demandcolour,ultra thick,name path=Dt2] (0,3) node [black,mynode,left] {$W_{t2}$} -- (13,3) node [black,mynode,right] {$D_{t2}$};
\draw [demandcolour,ultra thick,name path=Dt1] (0,5) node [black,mynode,left] {$W_{t1}$} -- (13,5) node [black,mynode,right] {$D_{t1}$};
\draw [demandcolour,ultra thick,name path=D0] (0,7) node [black,mynode,left] {$W_0$} -- (13,7) node [black,mynode,right] {$D_0$};
% supply line
\draw [supplycolour,ultra thick,domain=4:14,name path=S] plot (\x, {\x-4}) node [black,mynode,right] {$S$};
% axes
\draw [thick, -] (0,10) node (yaxis) [above] {Wage} -- (0,0) -- (15,0) node (xaxis) [right] {Labour};
% intersection of demand lines and supply
\draw [name intersections={of=S and Dt2, by=E2},name intersections={of=S and Dt1, by=E1},name intersections={of=S and D0, by=E0}]
	[dotted,thick] (E2) node [mynode,below right] {$E_2$} -- (xaxis -| E2) node [mynode,below] {$L_2$}
	[dotted,thick] (E1) node [mynode,below right] {$E_1$} -- (xaxis -| E1) node [mynode,below] {$L_1$}
	[dotted,thick] (E0) node [mynode,below right] {$E_0$} -- (xaxis -| E0) node [mynode,below] {$L_0$};
% paths to create dotted lines to B and A
\path [name path=Aline] (E1) -- +(0,4);
\path [name path=Bline] (E2) -- +(0,5);
% intersection of Aline and Bline with D0 line
\draw [name intersections={of=D0 and Aline, by=A},name intersections={of=D0 and Bline, by=B}]
	[dotted,thick] (E1) -- (A) node [mynode,below right] {A}
	[dotted,thick] (E2) -- (B) node [mynode,below right] {B};
% path to create arrow for wage tax arrows
\path [name path=IWTline] (5.5,0) -- +(0,10);
% intersection of IWTline with D0 and Dt1 lines
\draw [name intersections={of=IWTline and D0, by=notax},name intersections={of=IWTline and Dt1, by=firsttax}]
	[<->,thick,shorten >=0.5mm,shorten <=0.5mm] (notax) -- node [mynode,left,midway] {Initial wage tax} (firsttax);
% intersection of IWTline with Dt2 line, and arrow for Final wage tax
\draw [name intersections={of=IWTline and Dt2, by=finaltax}]
	[<->,thick,shorten >=0.5mm,shorten <=0.5mm] ([xshift=22em]finaltax) -- node [mynode,right,midway] {Final\\wage\\tax} ([xshift=22em]notax);
\end{FigureBox}

Let us continue with the illustration: suppose the tax is increased further so as to reduce the net wage to $W_{t2}$. The DWL is now B$E_0E_2$, much larger than before. Whether we should take this extra step in sacrificing more efficiency for redistributive gains is an ethical or normative issue. The citizens of some economies, most notably in Scandinavia, appear more willing than the citizens of the United States to make efficiency sacrifices in return for other objectives. Canada lies between these extremes, and our major political parties can be placed clearly on a spectrum of willingness to trade equity and efficiency. A vital role for the economist, therefore, is to clarify the nature and extent of the trade-offs. The field of public economics views this as a centrepiece in its investigations.
