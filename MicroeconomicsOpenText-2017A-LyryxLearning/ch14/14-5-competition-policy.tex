\section{Regulation and competition policy} \label{sec:regulation}
 
\subsection*{Goals and objectives}

The goals of competition policy are relatively uniform across developed economies: the promotion of domestic competition; the development of new ideas, new products and new enterprises; the promotion of efficiency in the resource-allocation sense; the development of manufacturing and service industries that can compete internationally.

In addition to these economic objectives, governments dislike monopolies or monopoly practices if they lead to an undue \textit{concentration of political power}. Such power can lead to a concentration of wealth and influence in the hands of an elite.

Canada's regulatory body is the \textit{Competition Bureau}, whose activity is governed primarily by the \textit{Competition Act} of 1986. This act replaced the \textit{Combines Investigation Act}. The \textit{Competition Tribunal} acts as an adjudication body, and is composed of judges and non- judicial members. This tribunal can issue orders on the maintenance of competition in the marketplace. Canada has had anti-combines legislation since 1889, and the act of 1986 is the most recent form of such legislation and policy. The Competition Act does not forbid monopolies, but it does rule as unlawful the \textit{abuse} of monopoly. Canada's competition legislation is aimed at anti-competitive practices, and a full description of its activities is to be found on its website at \href{www.competitionbureau.gc.ca}{www.competitionbureau.gc.ca}. Let us examine some of these proscribed policies.

\subsection*{Anti-competitive practices}

\terminology{Mergers} may turn competitive firms into a single organization with excessive market power. The customary justification for mergers is that they permit the merged firms to achieve scale economies that would otherwise be impossible. Such scale economies may in turn result in lower prices in the domestic or international market to the benefit of the consumer, but may alternatively reduce competition and result in higher prices. Equally important in this era of global competition is the impact of a merger on a firm's ability to compete internationally

Mergers can be of the horizontal type (e.g. two manufacturers of pre-mixed concrete merge) or vertical type (a concrete manufacturer merges with a cement manufacturer). While mergers have indeed the potential to reduce domestic competition, this is less true today than in the past, when international trade was more restricted.

\terminology{Cartels} aim to restrict output and thereby increase profits. These formations are almost universally illegal in individual national economies. At the same time governments are rarely worried if their own national firms participate in international cartels.

While cartels are one means of increasing prices, \terminology{price discrimination} is another. For example, if a concrete manufacturer makes her product available to large builders at a lower price than to small-scale builders -- perhaps because the large builder has more bargaining power, then the small builder is at a competitive disadvantage in the construction business. If the small firm is forced out of the construction business as a consequence, then competition in this sector is reduced. 

\terminology{Predatory pricing} is a practice that is aimed at driving out competition by artificially reducing the price of one product sold by a supplier. For example, a dominant nationwide transporter could reduce price on a particular route where competition comes from a strictly local competitor. By `subsidizing' this route from profits on other routes, the dominant firm could undercut the local firm and drive it out of the market.

\begin{DefBox}
\textbf{Predatory pricing} is a practice that is aimed at driving out competition by artificially reducing the price of one product sold by a supplier.
\end{DefBox}

Suppliers may also \terminology{refuse to deal}. If the local supplier of pre-mixed concrete refuses to sell the product to a local construction firm, then the ability of such a downstream firm to operate and compete may be compromised. This practice is similar to that of \terminology{exclusive sales} and \terminology{tied sales}. An exclusive sale might involve a large vegetable wholesaler forcing her retail clients to buy \textit{only} from this supplier. Such a practice might hurt the local grower of aubergines or zucchini, and also may prevent the retailer from obtaining \textit{some} of her vegetables at a lower price or at a higher quality elsewhere. A tied sale is one where the purchaser must agree to purchase a \textit{bundle} of goods from the one supplier. 

\begin{DefBox}
\textbf{Refusal to deal}: an illegal practice where a supplier refuses to sell to a purchaser.

\textbf{Exclusive sale}: where a retailer is obliged (perhaps illegally) to purchase all wholesale products from a single supplier only.

\textbf{Tied sale}: one where the purchaser must agree to purchase a bundle of goods from the one supplier.
\end{DefBox}

\terminology{Resale price maintenance} involves the producer requiring a retailer to sell a product at a specified price. This practice can hurt consumers since they cannot `shop around'.  In Canada, we frequently encounter a `manufacturer's suggested retail price' for autos and durable goods. But since these prices are not \textit{required}, the practice conforms to the law.

\begin{DefBox}
\textbf{Resale price maintenance} is an illegal practice wherein a producer requires sellers to maintain a specified price.
\end{DefBox}

\terminology{Bid rigging} is an illegal practice in which normally competitive bidders conspire to fix the awarding of contracts or sales. For example, two builders, who consider bidding on construction projects, may decide that one will bid seriously for project X and the other will bid seriously on project Y.  In this way they conspire to reduce competition in order to make more profit.

\begin{DefBox}
\textbf{Bid rigging} is an illegal practice in which bidders (buyers) conspire to set prices in their own interest.
\end{DefBox}

\terminology{Deception and dishonesty in promoting products} can either short-change the consumer or give one supplier an unfair advantage over other suppliers.

\subsection*{Enforcement}

The Competition Act is enforced through the Competition Bureau in a variety of ways. Decisions on acceptable business practices are frequently reached through study and letters of agreement between the Bureau and businesses. In some cases, where laws appear to have been violated, criminal proceedings may follow. 

\subsection*{Regulation, deregulation and privatization}

The last two decades have seen a discernible trend towards privatization and deregulation in Canada, most notably in the transportation and energy sectors. Modern deregulation in the US began with the passage of the \textit{Airline deregulation Act} of 1978, and was pursued with great energy under the Reagan administration in the eighties. The Economic Council of Canada produced an influential report in 1981, entitled ``Reforming Regulation'', on the impact of regulation and possible deregulation of specific sectors.

Telecommunications provision, in the era when the telephone was the main form of such communication, was traditionally viewed as a natural monopoly. The Canadian Radio and Telecommunications Commission (CRTC) regulated its rates. The industry has developed dramatically in the last decade with the introduction of satellite-facilitated communication, the internet, multi-purpose cable networks, cell phones and service integration. With such rapid development in new technologies and modes of delivery the CRTC has been playing catch-up to these new realities.

Transportation, in virtually all forms, has been deregulated in Canada since the nineteen eighties. Railways were originally required to subsidize the transportation of grain under the \textit{Crow's Nest Pass} rate structure. But the subsidization of particular markets requires an excessive rate elsewhere, and if the latter markets become subject to competition then a competitive system cannot function. This structure, along with many other anomalies, was changed with the passage of the \textit{Canada Transportation Act} in 1996. 

Trucking, historically, has been regulated by individual provinces. Entry was heavily controlled prior to the federal \textit{National Transportation Act} of 1987, and subsequent legislation introduced by a number of provinces, have made for easier entry and a more competitive rate structure. 

Deregulation of the airline industry in the US in the late seventies had a considerable influence on thinking and practice in Canada. The Economic Council report of 1981 recommended in favour of easier entry and greater fare competition. These policies were reflected in the 1987 National Transportation Act. Most economists are favourable to deregulation and freedom to enter, and the US experience indicated that cost reductions and increased efficiency could follow. In 1995 an agreement was reached between the US and Canada that provided full freedom for Canadian carriers to move passengers to any US city, and freedom for US carriers to do likewise, subject to a phase-in provision.

The National Energy Board regulates the development and transmission of oil and natural gas. But earlier powers of the Board, involving the regulation of product prices, were eliminated in 1986, and controls on oil exports were also eliminated. 

Agriculture remains a highly controlled area of the economy. Supply `management', which is really supply restriction, and therefore `price maintenance', characterizes grain, dairy, poultry and other products. Management is primarily through provincial marketing boards. 

\subsection*{Price regulation}

Regulating monopolistic sectors of the economy is one means of reducing their market power. In Chapter~\ref{chap:imperfectcompetition} it was proposed that indefinitely decreasing production costs in an industry means that the industry might be considered as a `natural' monopoly: higher output can be produced at lower cost with fewer firms. Hence, a single supplier has the potential to supply the market at a lower unit cost; unless, that is, such a single supplier uses his monopoly power. To illustrate how the consumer side may benefit from this production structure through regulation, consider Figure~\ref{fig:deccostsupplier}. For simplicity suppose that long-run marginal costs are constant and that average costs are downward sloping due to an initial fixed cost. The profit-maximizing (monopoly) output is where $MR=MC$ at $Q_m$ and is sold at the price $P_m$. This output is inefficient because the willingness of buyers to pay for additional units of output exceeds the additional cost. On this criterion the efficient output is $Q^*$. But $LATC$ exceeds price at $Q^*$, and therefore it is not feasible for a producer.

% Figure 14.4
\input{figures/ch14/ch14fig4}

One solution is for the regulating body to set a price-quantity combination of $P_r$, and $Q_r$, where price equals average cost and therefore generates a normal rate of profit. This output level is still lower than the efficient output level $Q^*$, but is more efficient than the profit-maximizing output $Q_m$. It is more efficient in the sense that it is closer to the efficient output $Q^*$. A problem with such a strategy is that it may induce lax management or overpayment to the factors of production: if producers are allowed to charge an average cost price, then there is a reduced incentive for them to keep strict control of their costs in the absence of competition in the marketplace.

A second solution to the declining average cost phenomenon is to implement what is called a \terminology{two-part tariff}. This means that customers pay an `entry fee' in order to be able to purchase the good. For example, telephone subscribers may pay a fixed charge per month for their phone and then may pay an additional charge that varies with use. In this way it is possible for the supplier to charge a price per unit of output that is closer to marginal cost and still make a profit, than under an average cost pricing formula. In terms of Figure~\ref{fig:deccostsupplier}, the total value of entry fees, or fixed components of the pricing, would have to cover the difference between $MC$ and $LATC$ times the output supplied. In Figure~\ref{fig:deccostsupplier} this implies that if the efficient output $Q^*$ is purchased at a price equal to the $MC$ the producer loses the amount (c-$MC$) on each unit sold. The access fees would therefore have to cover at least this value.

Such a solution is appropriate when fixed costs are high and marginal costs are low. This situation is particularly relevant in the modern market for telecommunications: the cost to suppliers of marginal access to their networks, whether it be for internet, phone or TV, is negligible compared to the cost of maintaining the network and installing capacity.

\begin{DefBox}
\textbf{Two-part tariff}: involves an access fee and a per unit of quantity fee.
\end{DefBox}

Finally, a word of caution: Nobel Laureate George Stigler has argued that there is a danger of regulators becoming too close to the regulated, and that the relationship can evolve to a point where the regulator may protect the regulated firms. A likely such case was Canada's airline industry in the seventies.  