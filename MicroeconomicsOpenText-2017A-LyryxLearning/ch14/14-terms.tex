\newpage
\markboth{Key Terms}{Key Terms}
	\addcontentsline{toc}{section}{Key terms}
	\section*{\textsc{Key Terms}}
\begin{keyterms}
\textbf{Market failure} defines outcomes in which the allocation of resources is not efficient.

\textbf{Public goods} are non-rivalrous, in that they can be consumed simultaneously by more than one individual; additionally they may have a non-excludability characteristic.

\textbf{Efficient supply of public goods} is where the marginal cost equals the sum of individual marginal valuations, and each individual consumes the same quantity.

\textbf{Asymmetric information} is where at least one party in an economic relationship has less than full information and has a different amount of information from another party.

\textbf{Adverse selection} occurs when incomplete or asymmetric information describes an economic relationship.

\textbf{Moral hazard} may characterize behaviour where the costs of certain activities are not incurred by those undertaking them.

\textbf{Predatory pricing} is a practice that is aimed at driving out competition by artificially reducing the price of one product sold by a supplier.

\textbf{Refusal to deal}: an illegal practice where a supplier refuses to sell to a purchaser.

\textbf{Exclusive sale}: where a retailer is obliged (perhaps illegally) to purchase all wholesale products from a single supplier only.

\textbf{Tied sale}: one where the purchaser must agree to purchase a bundle of goods from the one supplier.

\textbf{Two-part tariff}: involves an access fee and a per unit of quantity fee.
\end{keyterms}