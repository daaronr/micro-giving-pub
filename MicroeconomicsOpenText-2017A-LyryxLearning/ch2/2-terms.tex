\newpage
\markboth{Key Terms}{Key Terms}
	\addcontentsline{toc}{section}{Key terms}
	\section*{\textsc{Key Terms}}
\begin{keyterms}
\textbf{Variables}: measures that can take on different sizes.

\textbf{Data}: recorded values of variables.

\textbf{Time series data}: a set of measurements made sequentially at different points in time.

\textbf{High (low) frequency data} series have short (long) intervals between observations.

\textbf{Cross-section data}: values for different variables recorded at a point in time.

\textbf{Longitudinal data} follow the same units of observation through time.

\textbf{Index number}: value for a variable, or an average of a set of variables, expressed relative to a given base value.
\begin{equation*}
\text{Value of index}=\frac{\text{Absolute value in year }t}{\text{Absolute value in base year}}\times 100. 
\end{equation*}

\textbf{Percentage change}$=\displaystyle\frac{(\text{change in values})}{\text{original value}}\times 100$.

\textbf{Consumer price index}: the average price level for consumer goods and services.

\textbf{Inflation rate}: the annual percentage increase in the consumer price index.

\textbf{Deflation rate}: the annual percentage decrease in the consumer price index.

\textbf{Real price index}: a nominal price index divided by the consumer price index, scaled by 100.

\textbf{Nominal price index}: the current dollar price of a good or service.

\textbf{Nominal earnings}: earnings measured in current dollars.

\textbf{Real earnings}: earnings measure in constant dollars to adjust for changes in the general price level.

\textbf{Scatter diagram} plots pairs of values simultaneously observed for two variables.

\textbf{Econometrics} is the science of examining and quantifying relationships between economic variables.

\textbf{Regression line} represents the average relationship between two variables in a scatter diagram.

\textbf{Positive economics} studies objective or scientific explanations of how the economy functions.

\textbf{Normative economics} offers recommendations that incorporate value judgments.

\textbf{Economic equity} is concerned with the distribution of well-being among members of the economy.
\end{keyterms}