\section{Variables, data and index numbers}\label{sec:ch2sec2}

Economic theories and models are concerned with economic variables. \terminology{Variables} are measures that can take on different sizes. The interest rate on a student loan, for example, is a variable with a certain value at a point in time but perhaps a different value at an earlier or later date. Economic theories and models explain the causal relationships between variables.

\begin{DefBox}
\textbf{Variables}: measures that can take on different values.
\end{DefBox}

\terminology{Data} are the recorded values of variables. Sets of data provide specific values for the variables we want to study and analyze. Knowing that the Don Valley Parkway is congested does not tell us how slow our trip to downtown Toronto will be. To choose the best route downtown we need to ascertain the \textit{degree of congestion}---the data on traffic density and flow on alternative routes. A model is useful because it defines the variables that are most important and to the analysis of travel time and the data that are required for that analysis.

\begin{DefBox}
\textbf{Data}: recorded values of variables.
\end{DefBox}

Sets of data also help us to test our models or theories, but first we need to pay attention to the economic logic involved in observations and modelling. For example, if sunspots or baggy pants were found to be correlated with economic expansion, would we consider these events a coincidence or a key to understanding economic growth? The observation is based on facts or data but it need not have any economic content. The economist's task is to distinguish between coincidence and economic causation. 

While the more frequent wearing of loose clothing in the past may have been associated with economic growth because they both occurred at the same time (correlation), one could not argue on a logical basis that this behaviour causes good economic times. Therefore, the past association of these variables should be considered as no more than a coincidence.

Once specified on the basis of economic logic, a model must be tested to determine its usefulness in explaining observed economic events. The earlier example of a model of house prices and mortgage rates was based on the economics of the effect of financing cost on expenditure and prices. But we did not test that model by confronting it with the data. It may be that effects of mortgages rates are insignificant compared to other influences on house prices.  

\subsection*{Time-series data}

Data come in several forms. One form is \terminology{time-series}, which reflects a set of measurements made in sequence at different points in time. Table~\ref{table:housepriceindex} reports the annual time series values for several price series. Such information may also be presented in charts or graphs. Figure~\ref{fig:housemortgageprice} plots the data from column 2, and each point represents the data observed for a specific time period. The horizontal axis reflects time in years, the vertical axis price in dollars.

\begin{DefBox}
\textbf{Time-series}: a set of measurements made sequentially at different points in time.
\end{DefBox}

%%% Table 2.1
\begin{Table}{25em}{House prices and price indexes \label{table:housepriceindex}}{\textit{Source}: Prices for North Vancouver houses come from Royal Le Page; CPI from Statistics Canada, CANSIM II, V41692930 and author's calculations.}
\begin{tabu} to \linewidth {|X[0.75,c]X[1.25,c]X[1,c]X[0.5,c]X[1.25,c]|}	\hline
\rowcolor{rowcolour}	\textbf{1}	&	\textbf{2}	&	\textbf{3}	&	\textbf{4}	&	\textbf{5}	\\
\textbf{Date}	&	\textbf{Price of detached}	&	\textbf{House price}	&	\textbf{CPI}	&	\textbf{Real house price}	\\[-0.5em]
	&	\textbf{bungalows}	&	\textbf{index}	&	&	\textbf{index}	\\[-0.5em]
	&	\textbf{N. Vancouver}	&	&	&	\\
\rowcolor{rowcolour}	\textbf{1999Q1} & 330,000 & 100.0 & 100.00 & 100.00 \\
													\textbf{2000Q1} & 345,000 & 104.55 & 101.29 & 103.21 \\
\rowcolor{rowcolour}	\textbf{2001Q1} & 350,000 & 106.06 & 104.63 & 101.37 \\
													\textbf{2002Q1} & 360,000 & 109.09 & 105.49 & 103.41 \\
\rowcolor{rowcolour}	\textbf{2003Q1} & 395,000 & 119.70 & 108.61 & 110.21 \\
													\textbf{2004Q1} & 434,000 & 131.52 & 110.01 & 119.55 \\
\rowcolor{rowcolour}	\textbf{2005Q1} & 477,000 & 144.55 & 112.81 & 128.13 \\ 
													\textbf{2006Q1} & 580,000 & 175.76 & 114.32 & 153.75 \\
\rowcolor{rowcolour}	\textbf{2007Q1} & 630,000 & 190.91 & 117.33 & 162.71 \\ 
													\textbf{2008Q1} & 710,000 & 215.15 & 118.62 & 181.38 \\
\rowcolor{rowcolour}	\textbf{2009Q1} & 605,000 & 183.33 & 120.56 & 152.07 \\
													\textbf{2010Q1} & 740,000 & 224.24 & 125.40 & 178.96 \\
\rowcolor{rowcolour}	\textbf{2011Q1} & 800,000 & 242.42 & 129.06 & 187.83 \\ 
													\textbf{2012Q1} & 870,000 & 263.33 & 131.00 & 210.02 \\ \hline 
\end{tabu}
\end{Table}

Annual data report one observation per year. We could, alternatively, have presented them in quarterly, monthly, or even weekly form. The frequency we use depends on the purpose: If we are interested in the longer-term trend in house prices, then the annual form suffices. In contrast, financial economists, who study the behaviour of stock prices, might not be content with daily or even hourly prices; they may need prices minute-by-minute. Such data are called \terminology{high-frequency} data, whereas annual data are \terminology{low-frequency} data.

\begin{DefBox}
\textbf{High (low) frequency data}: series with short (long) intervals between observations.
\end{DefBox}

When data are presented in charts or when using diagrams the scales on the axes have important visual effects. Different scales on either or both axes alter the perception of patterns in the data. To illustrate this, the data from columns 1 and 2 of Table~\ref{table:housepriceindex} are plotted in Figures~\ref{fig:housevancouver1} and \ref{fig:housevancouver2}, but with a change in the scale of the vertical axis.

%%% Figure 2.2 and 2.3 (known in word document as 2.2a and 2.2b, respectively)
\input{figures/ch2/ch2fig2}
\input{figures/ch2/ch2fig3}

The greater \textit{apparent} slope in Figure~\ref{fig:housevancouver1} might easily be interpreted to mean that prices increased more steeply than suggested in Figure~\ref{fig:housevancouver2}. But a careful reading of the axes reveals that this is not so; using different scales when plotting data or constructing diagrams can mislead the unaware viewer. 

\subsection*{Cross-section data}

In contrast to time-series data, \terminology{cross-section} data record the values of different variables at a point in time. Table~\ref{table:unemprate2012} contains a cross-section of unemployment rates for Canada and Canadian provinces economies. For January 2012 we have a snapshot of the provincial economies at that point in time, likewise for the months until June. This table therefore contains \terminology{repeated cross-sections}.

When the unit of observation is the same over time such repeated cross sections are called longitudinal data. For example, a health survey that followed and interviewed the same individuals over time would yield longitudinal data. If the individuals differ each time the survey is conducted, the data are repeated cross sections. \terminology{Longitudinal data} therefore follow the same units of observation through time.

%%% Table 2.2
\begin{Table}{25em}{Unemployment rates, Canada and Provinces, monthly 2012, seasonally adjusted \label{table:unemprate2012}}{\centering \textit{Source}: Statistics Canada CANSIM Table 282-0087}
\begin{tabu} to 35em {|X[2.5,l]X[1,c]X[1,c]X[1,c]X[1,c]X[1,c]X[1,c]|}
\hline 
\rowcolor{rowcolour}					& \textbf{Jan} & \textbf{Feb} & \textbf{Mar} & \textbf{Apr} & \textbf{May}  & \textbf{Jun} \\
						\textbf{CANADA}	& 7.6 & 7.4 & 7.2 & 7.3 & 7.3 & 7.2 \\
\rowcolor{rowcolour}	\textbf{NFLD}	& 13.5 & 12.9 & 13.0 & 12.3 & 12.0 & 13.0 \\
						\textbf{PEI} 	& 12.2 & 10.5 & 11.3 & 11.0 & 11.3 & 11.3 \\ 
\rowcolor{rowcolour}	\textbf{NS} 	& 8.4 & 8.2 & 8.3 & 9.0 & 9.2 & 9.6 \\
						\textbf{NB} 	& 9.5 & 10.1 & 12.2 & 9.8 & 9.4 & 9.5 \\
\rowcolor{rowcolour}	\textbf{QUE} 	& 8.4 & 8.4 & 7.9 & 8.0 & 7.8 & 7.7 \\ 
						\textbf{ONT} 	& 8.1 & 7.6 & 7.4 & 7.8 & 7.8 & 7.8 \\ 
\rowcolor{rowcolour}	\textbf{MAN} 	& 5.4 & 5.6 & 5.3 & 5.3 & 5.1 & 5.2 \\
						\textbf{SASK} 	& 5.0 & 5.0 & 4.8 & 4.9 & 4.5 & 4.9 \\ 
\rowcolor{rowcolour}	\textbf{ALTA}	& 4.9 & 5.0 & 5.3 & 4.9 & 4.5 & 4.6 \\ 
						\textbf{BC}		& 6.9 & 6.9 & 7.0 & 6.2 & 7.4 & 6.6 \\ \hline 
\end{tabu}
\end{Table}

\begin{DefBox}
\textbf{Cross-section data}: values for different variables recorded at a point in time.

\textbf{Repeated cross-section data}: cross-section data recorded at regular or irregular intervals.

\textbf{Longitudinal data}: follow the same units of observation through time.
\end{DefBox}

\subsection*{Index numbers}

It is important in economic analysis to discuss and interpret data in a meaningful manner. \terminology{Index numbers} help us greatly in doing this. They are values of a given variable, or an average of a set of variables expressed relative to a base value.  The key characteristics of indexes are that they are \textit{not dependent upon the units of measurement of the data in question, and they are interpretable easily with reference to a given base value}. To illustrate, let us change the price data in column 2 of Table~\ref{table:housepriceindex} into index number form.

\begin{DefBox}
\textbf{Index number}: value for a variable, or an average of a set of variables, expressed relative to a given base value.
\end{DefBox}

The first step is to choose a base year as a reference point. This could be any one of the periods. We will simply take the first period as the year and \textit{set the price index value equal to 100 in that year.} The value of 100 is usually chosen in order to make comparisons simple, but in some cases a base year value of 1.0 is used. If the base year value of 100 is used, the value of index in any year $t$ is:

\begin{equation} \label{eq:valueofindex}
\text{Value of index}=\frac{\text{Absolute value in year }t}{\text{Absolute value in base year}}\times 100
\end{equation}

Suppose we choose 1999 as the base year for constructing an index of the house prices given in Table~\ref{table:housepriceindex} House prices in that year were \$330,000. Then the \textit{index for the base year} has a value:

\begin{equation*}
\text{Index in 1999}=\frac{\$330,000}{\$330,000}\times 100=100
\end{equation*}

Applying the method to each value in column 2 yields column 3, which is now in index number form. For example, the January 2003 value is:

\begin{equation*}
\text{Index in 2003}=\frac{\$395,000}{\$330,000}\times 100=119.7
\end{equation*}

Each value in the index is interpreted relative to the value of 100, the base price in January 1999. The beauty of this column lies first in its \textit{ease of interpretation}. For example, by 2003 the price increased to 119.7 points relative to a value of 100. This yields an immediate interpretation: The index has increased by 19.7 points \textit{per hundred} or \textit{percent}. While it is particularly easy to compute a percentage change in a data series when the base value is 100, it is not necessary that the reference point have a value of 100. By definition, a \terminology{percentage change} is given by the change in values relative to the initial value, multiplied by 100. For example, the percentage change in the price from 2006 to 2007, using the price index is: $(190.91-175.76)/175.76\times 100=8.6$ percent.

\begin{DefBox}
$\text{\textbf{Percentage change}}=(\text{change in values})/\text{original value}\times 100$.
\end{DefBox}

Furthermore, index numbers enable us to make \textit{comparisons with the price patterns for other goods} much more easily. If we had constructed a price index for wireless phones, which also had a base value of 100 in 1999, we could make immediate comparisons without having to compare one set of numbers defined in dollars with another defined in tens of thousands of dollars. In short, index numbers simplify the interpretation of data.

\subsection*{Composite index numbers}

Index numbers have even wider uses than those we have just described. Suppose you are interested in the price trends for all fuels as a group in Canada during the last decade. You know that this group includes coal, natural gas, and oil, but you suspect that these components have not all been rising in price at the same rate. You also know that, while these fuels are critical to the economy, some play a bigger role than others, and therefore should be given more importance, or weight, in a general fuel price index. In fact, the official price index for these fuels is a \textit{weighted average of the component price indexes}. The fuels that are more important get a heavier weighting in the overall index. For example, if oil accounts for 60 percent of fuel use, natural gas for 25 percent, and coal for 15 percent, the price index for fuel could be computed as follows:

\begin{equation} \label{eq:fuelpriceindex}
\text{Fuel price index}=(\text{oil index}\times 0.6)+(\text{natural gas index}\times 0.25)+(\text{coal index}\times 0.15)
\end{equation}

To illustrate this, Figure~\ref{fig:compositefuel} presents the price trends for these three fuels. The data come from Statistics Canada's CANSIM database. In addition, the overall fuel price index is plotted. It is frequently the case that components do not display similar patterns, and in this instance the composite index follows oil most closely, reflecting the fact that oil has the largest weight.

%%% Figure 2.4 (called 2.3 in original text)
\begin{FigureBox}{1.15}{1}{25em}{Composite fuel price index \label{fig:compositefuel}}{\textit{Source}: Statistics Canada. Table 330-0007 -- Raw materials price indexes, monthly (index, 2002=100), CANSIM.}
\begin{axis}[
	%axis lines=left,
	axis line style=thick,
	every tick label/.append style={font=\footnotesize},
	every node near coord/.append style={font=\scriptsize},
	xticklabel style={rotate=90,anchor=east,/pgf/number format/1000 sep=},
	scaled y ticks=false,
	yticklabel style={/pgf/number format/fixed,/pgf/number format/1000 sep = \thinspace},
	xmin=2000,xmax=2012,ymin=0,ymax=375,
	y=1cm/50,
	x=0.8cm/1,
	x label style={at={(axis description cs:0.5,-0.05)},anchor=north},
	xlabel={Year},
	ylabel={Index value},
	legend entries={All fuels,Coal,Oil,Gas},
	legend style={at={(axis cs:2001,225)},anchor=south west},
]
\addplot[datasetcolourone,ultra thick,mark=none] table {	% all fuels
x							y
2000					89.1
%2000.083333		90.3
%2000.166667		96.5
%2000.25				99.6
%2000.333333		85.4
%2000.416667		97.5
2000.5				105.3
%2000.583333		102.3
%2000.666667		106.6
%2000.75				114.9
%2000.833333		116.8
%2000.916667		122.4
2001					107.1
%2001.083333		111.7
%2001.166667		112
%2001.25				108.1
%2001.333333		108.6
%2001.416667		110.6
2001.5				109.1
%2001.583333		104.6
%2001.666667		106.1
%2001.75				99.8
%2001.833333		89.3
%2001.916667		82.1
2002					81.2
%2002.083333		81.5
%2002.166667		84.9
%2002.25				96.9
%2002.333333		102.1
%2002.416667		103.5
2002.5				97.5
%2002.583333		101
%2002.666667		104.5
%2002.75				110
%2002.833333		108.3
%2002.916667		99.5
2003					110.2
%2003.083333		121.5
%2003.166667		129.7
%2003.25				120.7
%2003.333333		104.9
%2003.416667		99.5
2003.5				105.3
%2003.583333		105.2
%2003.666667		106.7
%2003.75				95.6
%2003.833333		97.3
%2003.916667		98.9
2004					102.4
%2004.083333		107.8
%2004.166667		110.6
%2004.25				118.4
%2004.333333		118.7
%2004.416667		126.4
2004.5				123.4
%2004.583333		126.1
%2004.666667		137.4
%2004.75				139.1
%2004.833333		151.2
%2004.916667		137
2005					123.6
%2005.083333		134.8
%2005.166667		142.1
%2005.25				157.6
%2005.333333		155
%2005.416667		145.5
2005.5				161.8
%2005.583333		167.7
%2005.666667		181.6
%2005.75				182
%2005.833333		176.3
%2005.916667		167.8
2006					160.5
%2006.083333		172.5
%2006.166667		161.9
%2006.25				162.2
%2006.333333		174.1
%2006.416667		186.1
2006.5				183.9
%2006.583333		195.4
%2006.666667		185
%2006.75				166.1
%2006.833333		147.9
%2006.916667		149
2007					159.4
%2007.083333		150.7
%2007.166667		159.3
%2007.25				159.7
%2007.333333		166
%2007.416667		163.9
2007.5				167.5
%2007.583333		184.3
%2007.666667		179.9
%2007.75				182
%2007.833333		183.2
%2007.916667		201.3
2008					202.4
%2008.083333		211.8
%2008.166667		209.5
%2008.25				234.1
%2008.333333		258.8
%2008.416667		276.7
2008.5				300.3
%2008.583333		306.3
%2008.666667		271
%2008.75				240.7
%2008.833333		195.2
%2008.916667		149.3
2009					103.1
%2009.083333		106.2
%2009.166667		111.6
%2009.25				142.1
%2009.333333		138.8
%2009.416667		147.7
2009.5				166.6
%2009.583333		155.2
%2009.666667		166.3
%2009.75				163.7
%2009.833333		172.1
%2009.916667		175.7
2010					167.8
%2010.083333		176.9
%2010.166667		180.5
%2010.25				182.6
%2010.333333		185.2
%2010.416667		161
2010.5				164.5
%2010.583333		167.9
%2010.666667		170.1
%2010.75				164.5
%2010.833333		167.8
%2010.916667		179.6
2011					189.5
%2011.083333		184.6
%2011.166667		184.7
%2011.25				212.7
%2011.333333		240.7
%2011.416667		220.6
2011.5				209.9
%2011.583333		205.1
%2011.666667		190.7
%2011.75				198.6
%2011.833333		200.1
%2011.916667		216.4
2012					209.6
};
\addplot[datasetcolourtwo,ultra thick,mark=none,densely dotted] table {	% coal
x							y
2000					96.5
%2000.083333		97.1
%2000.166667		98.6
%2000.25				98.9
%2000.333333		98.7
%2000.416667		99.9
2000.5				100.7
%2000.583333		99.4
%2000.666667		98.3
%2000.75				97.9
%2000.833333		98.2
%2000.916667		99
2001					96
%2001.083333		98.5
%2001.166667		103.1
%2001.25				104.2
%2001.333333		102.1
%2001.416667		101.2
2001.5				98.3
%2001.583333		98.9
%2001.666667		97.5
%2001.75				98.7
%2001.833333		98.6
%2001.916667		99.2
2002					98.8
%2002.083333		99
%2002.166667		99
%2002.25				99
%2002.333333		99
%2002.416667		101.1
2002.5				101
%2002.583333		100.4
%2002.666667		100.6
%2002.75				101
%2002.833333		101.2
%2002.916667		98.8
2003					99.9
%2003.083333		99.5
%2003.166667		99.1
%2003.25				99.5
%2003.333333		99.2
%2003.416667		97.6
2003.5				97.9
%2003.583333		99
%2003.666667		98.9
%2003.75				98.7
%2003.833333		98.2
%2003.916667		97.9
2004					98
%2004.083333		97.8
%2004.166667		98.2
%2004.25				98.2
%2004.333333		98.4
%2004.416667		98.8
2004.5				98.6
%2004.583333		98.1
%2004.666667		98
%2004.75				97.7
%2004.833333		97.2
%2004.916667		97.2
2005					97.5
%2005.083333		98
%2005.166667		99.2
%2005.25				97
%2005.333333		98.1
%2005.416667		107.5
2005.5				102.6
%2005.583333		100.6
%2005.666667		101.6
%2005.75				101.8
%2005.833333		100.6
%2005.916667		96.7
2006					98.3
%2006.083333		99.9
%2006.166667		98.8
%2006.25				99.4
%2006.333333		100.5
%2006.416667		101.8
2006.5				108.4
%2006.583333		103.2
%2006.666667		101.4
%2006.75				101.1
%2006.833333		108.2
%2006.916667		99.1
2007					96.9
%2007.083333		98.3
%2007.166667		93
%2007.25				96.4
%2007.333333		97.3
%2007.416667		96.8
2007.5				95
%2007.583333		100.3
%2007.666667		99
%2007.75				96.9
%2007.833333		95.9
%2007.916667		95.8
2008					94.6
%2008.083333		95.7
%2008.166667		97.8
%2008.25				97.1
%2008.333333		92.6
%2008.416667		101
2008.5				106.5
%2008.583333		101.8
%2008.666667		99.5
%2008.75				102.7
%2008.833333		105.4
%2008.916667		104.5
2009					102.4
%2009.083333		107.6
%2009.166667		110.2
%2009.25				108.5
%2009.333333		107.8
%2009.416667		109.1
2009.5				107.8
%2009.583333		106.7
%2009.666667		105.8
%2009.75				103.9
%2009.833333		98.1
%2009.916667		97.2
2010					95
%2010.083333		103.6
%2010.166667		108.4
%2010.25				109.5
%2010.333333		108.7
%2010.416667		107.2
2010.5				112.6
%2010.583333		105.7
%2010.666667		103.4
%2010.75				104
%2010.833333		103
%2010.916667		100.5
2011					99.5
%2011.083333		108
%2011.166667		101.8
%2011.25				107.2
%2011.333333		106.8
%2011.416667		106.2
2011.5				112.9
%2011.583333		111.2
%2011.666667		108.5
%2011.75				111.2
%2011.833333		117.6
%2011.916667		112.1
2012					105.9
};
\addplot[datasetcolourthree,ultra thick,mark=none,densely dashed] table {	% oil
x							y
2000					95.8
%2000.083333		97
%2000.166667		104.8
%2000.25				108.9
%2000.333333		91.1
%2000.416667		105.9
2000.5				114.6
%2000.583333		108.4
%2000.666667		114.3
%2000.75				123
%2000.833333		124.2
%2000.916667		131.3
2001					107
%2001.083333		109.2
%2001.166667		111
%2001.25				102.8
%2001.333333		104.1
%2001.416667		107.1
2001.5				105.8
%2001.583333		101.5
%2001.666667		103.5
%2001.75				98.2
%2001.833333		84.6
%2001.916667		75.4
2002					73.2
%2002.083333		76.2
%2002.166667		80.5
%2002.25				95.1
%2002.333333		101.4
%2002.416667		102.9
2002.5				97.5
%2002.583333		103.4
%2002.666667		107.8
%2002.75				113.3
%2002.833333		110.2
%2002.916667		99.5
2003					112.2
%2003.083333		125
%2003.166667		134.8
%2003.25				123.5
%2003.333333		104
%2003.416667		98.9
2003.5				105.4
%2003.583333		105.4
%2003.666667		108.3
%2003.75				93.8
%2003.833333		96.5
%2003.916667		98.4
2004					102.6
%2004.083333		107.9
%2004.166667		111.6
%2004.25				121.8
%2004.333333		122.3
%2004.416667		131.8
2004.5				126.6
%2004.583333		131
%2004.666667		144.9
%2004.75				147
%2004.833333		162.1
%2004.916667		144.5
2005					128
%2005.083333		141.8
%2005.166667		150.8
%2005.25				169.8
%2005.333333		166.5
%2005.416667		154.6
2005.5				174.7
%2005.583333		181
%2005.666667		197.9
%2005.75				197.7
%2005.833333		188.6
%2005.916667		177.8
2006					169.7
%2006.083333		180.8
%2006.166667		169.2
%2006.25				170.3
%2006.333333		188
%2006.416667		202.6
2006.5				200.1
%2006.583333		215
%2006.666667		202.4
%2006.75				179
%2006.833333		156.9
%2006.916667		158.2
2007					170.8
%2007.083333		161.3
%2007.166667		172.2
%2007.25				172.3
%2007.333333		179.8
%2007.416667		177.2
2007.5				181.7
%2007.583333		203.3
%2007.666667		198.1
%2007.75				200.7
%2007.833333		202.4
%2007.916667		224.2
2008					225.4
%2008.083333		237.2
%2008.166667		234.2
%2008.25				264.2
%2008.333333		293.9
%2008.416667		314.8
2008.5				343.3
%2008.583333		347.8
%2008.666667		305.1
%2008.75				268.6
%2008.833333		215.3
%2008.916667		159
2009					101.9
%2009.083333		107.2
%2009.166667		114.1
%2009.25				151.8
%2009.333333		150.6
%2009.416667		161.5
2009.5				184.7
%2009.583333		171.5
%2009.666667		185.3
%2009.75				182.3
%2009.833333		193.1
%2009.916667		196.8
2010					187.1
%2010.083333		197.5
%2010.166667		201.8
%2010.25				204.6
%2010.333333		207.5
%2010.416667		177.9
2010.5				182.1
%2010.583333		187.5
%2010.666667		190.5
%2010.75				183.4
%2010.833333		187.9
%2010.916667		202.4
2011					214.6
%2011.083333		209.3
%2011.166667		209.7
%2011.25				243.6
%2011.333333		277.9
%2011.416667		253.3
2011.5				240
%2011.583333		234
%2011.666667		216.5
%2011.75				226.2
%2011.833333		228.2
%2011.916667		248.4
2012					240.4
};
\addplot[datasetcolourfour,ultra thick,mark=none,dashdotdotted] table {	% gas
x							y
2000					63.7
%2000.083333		65.2
%2000.166667		65.8
%2000.25				65.8
%2000.333333		63.6
%2000.416667		66.6
2000.5				71.3
%2000.583333		80.4
%2000.666667		79.5
%2000.75				86.7
%2000.833333		91
%2000.916667		92.1
2001					108.5
%2001.083333		122.3
%2001.166667		116.1
%2001.25				128.2
%2001.333333		125.6
%2001.416667		124.4
2001.5				121.9
%2001.583333		116.6
%2001.666667		116.4
%2001.75				105.6
%2001.833333		105.7
%2001.916667		105.5
2002					108.9
%2002.083333		105.4
%2002.166667		104.9
%2002.25				105
%2002.333333		106.3
%2002.416667		106.7
2002.5				96.9
%2002.583333		89.4
%2002.666667		88.8
%2002.75				95.1
%2002.833333		99.8
%2002.916667		99.9
2003					101.7
%2003.083333		107.1
%2003.166667		108.3
%2003.25				109.2
%2003.333333		109.8
%2003.416667		102.9
2003.5				105.8
%2003.583333		105.1
%2003.666667		99.7
%2003.75				104.3
%2003.833333		101.2
%2003.916667		101.6
2004					101.7
%2004.083333		108.8
%2004.166667		107.2
%2004.25				104.1
%2004.333333		103.7
%2004.416667		103.5
2004.5				110.4
%2004.583333		105.3
%2004.666667		104.9
%2004.75				104.8
%2004.833333		104.2
%2004.916667		105.1
2005					105
%2005.083333		104.6
%2005.166667		104.4
%2005.25				104.5
%2005.333333		105.2
%2005.416667		105.6
2005.5				105.1
%2005.583333		109.9
%2005.666667		111
%2005.75				114
%2005.833333		124.8
%2005.916667		126.8
2006					122.5
%2006.083333		139.9
%2006.166667		133.5
%2006.25				129.4
%2006.333333		114.2
%2006.416667		115
2006.5				113.2
%2006.583333		109.6
%2006.666667		109
%2006.75				110.4
%2006.833333		108.9
%2006.916667		109.8
2007					111
%2007.083333		104.8
%2007.166667		103.8
%2007.25				105.4
%2007.333333		106.1
%2007.416667		106.6
2007.5				106.3
%2007.583333		101.1
%2007.666667		99.7
%2007.75				100
%2007.833333		99.3
%2007.916667		101.7
2008					102.3
%2008.083333		100.5
%2008.166667		101.2
%2008.25				102.1
%2008.333333		105.9
%2008.416667		110.3
2008.5				112
%2008.583333		126.6
%2008.666667		123.5
%2008.75				119.5
%2008.833333		106.9
%2008.916667		107
2009					108.9
%2009.083333		100.9
%2009.166667		99.6
%2009.25				98.1
%2009.333333		85
%2009.416667		84.5
2009.5				84.6
%2009.583333		80.8
%2009.666667		80.1
%2009.75				79.5
%2009.833333		78
%2009.916667		81.2
2010					81.3
%2010.083333		84.2
%2010.166667		84.4
%2010.25				83.6
%2010.333333		84.8
%2010.416667		84.1
2010.5				84
%2010.583333		78.7
%2010.666667		78.1
%2010.75				78.9
%2010.833333		76.7
%2010.916667		76.9
2011					76.6
%2011.083333		72.5
%2011.166667		72.1
%2011.25				73.1
%2011.333333		73.8
%2011.416667		73.9
2011.5				73.5
%2011.583333		74.3
%2011.666667		73.7
%2011.75				73.5
%2011.833333		71.8
%2011.916667		71.3
2012					70.9
};
\end{axis}
\end{FigureBox}

\subsection*{Other composite price indexes}

The fuels price index is just one of many indexes constructed to measure a composite group of economic variables. There are also published indexes of commodity prices, including and excluding fuels, agricultural prices, average hourly earnings, industrial production, unit labour costs, Canadian dollar effective exchange rate (CERI), the S\&P/TSX stock market prices and consumer prices to list just a few. All these indexes are designed to reduce the complexity of the data on key sectors of the economy and important economic conditions.

The \terminology{consumer price index (CPI)} is the most widely quoted price index in the economy. It measures the average price level in the economy and changes in the CPI provide measures of the rate at which consumer goods and services change in price---\terminology{inflation} if prices increase, \terminology{deflation} if prices decline. 

The CPI is constructed in two stages. First, a consumer expenditure survey is use to establish the importance or weight of each of eight categories in a `basket' of goods and services. Then the cost of this basket of services in a particular year is compared to its cost in the chosen base year. With this base year cost of the basket set at 100 the ratio of the cost of the same basket in any other year to its cost in the base year multiplied my 100 gives the CPI for that year. The CPI for any given year is:

\begin{equation} \label{eq:cpi} 
\text{CPI}_t=\frac{\text{Cost of basket in year }t}{\text{Cost of basket in base year}}\times 100
\end{equation} 

\begin{DefBox}
\textbf{Consumer price index}: the average price level for consumer goods and services

\textbf{Inflation rate}: the annual percentage increase in the consumer price index

\textbf{Deflation rate}: the annual percentage decrease in the consumer price index
\end{DefBox}

\subsection*{Using price indexes}

The CPI is useful both as an indicator of how much prices change in the aggregate, and also as an indicator of \textit{relative price} changes. Column 4 of Table~\ref{table:housepriceindex} provides the Vancouver CPI with the same base year as the North Vancouver house price index. Note how the two indexes move very differently over time. The price of housing has increased considerably \textit{relative to the overall level of prices} in the local economy, as measured by the CPI: Housing has experienced a \textit{relative price increase}, or a \terminology{real price index}. This real increase is to be distinguished from the \terminology{nominal price index}, which is measured without reference to overall prices. The real price index for housing (or any other specific product) is obtained by dividing its specific price index by the CPI.

\begin{equation*}
\text{Real house price index}=\frac{\text{nominal house price index}}{\text{CPI}}\times 100
\end{equation*}

\begin{DefBox}
\textbf{Real price index}: a nominal price index divided by the consumer price index, scaled by 100.

\textbf{Nominal price index}: the current dollar price of a good or service.
\end{DefBox}

The resulting index is given in column 5 of Table~\ref{table:housepriceindex}. This index has a simple interpretation: It tells us by how much the price of Vancouver houses has changed relative to the general level of prices for goods and services. For example, between 1999 and 2004 the number 119.55 in column 5 for the year 2004 indicates that housing increased in price by 19.55 percent \textit{relative to prices in general}.

Here is a further simple example. Table~\ref{table:nominalrealearnings} reports recent annual data on indexes of \terminology{nominal earnings}, measured in current dollars, both average weekly and hourly rates, over the 2003-2011 time period. The table also reports the consumer price index for the same time period. To simplify the illustration all indexes have been \textit{re-based} to 2003=100 by dividing the reported value of the index in each year by its value in 2003 and multiplying by 100.

The table shows the difference between changes in nominal and real earnings. \terminology{Real earnings} are measured in constant dollars adjusted for changes in the general price level. The adjustment is made by dividing the indexes of nominal earnings in each year by the consumer price index in that year and multiplying by 100. 

\begin{DefBox}
\textbf{Nominal earnings}: earnings measured in current dollars.

\textbf{Real earnings}: earnings measure in constant dollars to adjust for changes in the general price level.
\end{DefBox}

As measured by the nominal weekly and hourly indexes, nominal earnings increased by 26 to 27 percent over the eight year period 2003-2011. However, the general price level as measured by the consumer price index (CPI) increased by close to 17 percent over the same period. As a result, real earnings, measured in terms of the purchasing power of nominal earnings increased by only about 9 percent, notable less than in 26 percent increase in nominal earnings.

\begin{Table}{25em}{Nominal and real earnings in Canada 2003-2011 \label{table:nominalrealearnings}}{\textit{Source}: Statistics Canada, CANSIM Series V1558664, V1606080 and V41690914 and author's calculations}
\begin{tabu} to 35em {|X[0.5,c]X[1,c]X[1,c]|X[0.5,c]X[1,c]X[1,c]|}	\hline 
\multicolumn{3}{|c|}{\cellcolor{rowcolour}\textbf{Nominal earnings}} & \multicolumn{3}{c|}{\cellcolor{rowcolour}\textbf{Real earnings}} \\ \hline 
\textbf{Year} & \textbf{Average} & \textbf{Average} & \textbf{CPI} & \textbf{Average} & \textbf{Average} \\[-0.5em]
	&	\textbf{weekly}	&	\textbf{hourly}	&	&	\textbf{weekly}	&	\textbf{hourly}	\\[-0.5em]
	&	\textbf{earnings}	&	\textbf{earnings}	&	&	\textbf{earnings}	&	\textbf{earnings}	\\	\hline
\rowcolor{rowcolour}	\textbf{2003} & 100.0 & 100.0 & 100.0 & 100.0 & 100.0 \\
						\textbf{2004} & 102.7 & 102.7 & 101.8 & 100.8 & 100.9 \\
\rowcolor{rowcolour}	\textbf{2005} & 106.7 & 106.2 & 104.2 & 102.4 & 101.9 \\ 
						\textbf{2006} & 109.4 & 108.8 & 106.1 & 103.0 & 102.5 \\ 
\rowcolor{rowcolour}	\textbf{2007} & 114.1 & 113.8 & 108.5 & 105.2 & 104.9 \\ 
						\textbf{2008} & 117.4 & 117.7 & 111.0 & 105.7 & 106.1 \\ 
\rowcolor{rowcolour}	\textbf{2009} & 119.2 & 121.3 & 111.3 & 107.1 & 109.0 \\ 
						\textbf{2010} & 123.5 & 125.0 & 113.3 & 109.0 & 110.3 \\
\rowcolor{rowcolour}	\textbf{2011} & 126.6 & 127.5 & 116.6 & 108.6 & 109.4 \\ \hline  
\end{tabu}
\end{Table}

These observations illustrate two important points. First the distinction between real and nominal values is very important. If the general price level is changing, changes in real values will differ from changes in nominal values. Real values change by either less or more than changes in nominal values. Second, in addition to tracking change over time, index numbers used in combination simplify the adjustment from nominal to real values, as shown in both Table~\ref{table:housepriceindex} and ~\ref{table:nominalrealearnings}.

However, a word of caution is necessary. Index numbers can be used to track both nominal and real values over time but they do not automatically adjust for change in the quality of products and services or the changing patterns of output and use in the cases of composite indexes. Index number bases and weights need constant adjustment to deal with these issues. 

