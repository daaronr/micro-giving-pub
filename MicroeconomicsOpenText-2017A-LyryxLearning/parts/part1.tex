\cleardoublepage
\thispagestyle{empty}
\vspace{30mm}
\addcontentsline{toc}{part}{\normalfont \textbf{Part One: The Building Blocks}}
{\color{parttextcolour}\fontsize{1.25cm}{3em}\selectfont\textbf{Part One}} \\ \\
{\color{parttextcolour}\huge The Building Blocks}

\vspace{10mm}
{\color{partlinecolour}\rule{25em}{2pt}}
\vspace{10mm}

{\large\color{parttextcolour}
~\ref{chap:intro}. Introduction to key ideas

~\ref{chap:tmd}. Theories, models and data

~\ref{chap:classical}. The classical marketplace -- demand and supply}

\vspace{10mm}

{\normalfont Economics is a social science; it analyzes human interactions in a scientific manner. We begin by defining the central aspects of this social science -- trading, the marketplace, opportunity cost and resources. We explore how producers and consumers interact in society. Trade is central to improving the living standards of individuals. This material forms the subject matter of Chapter~\ref{chap:intro}.

Methods of analysis are central to any science. Consequently we explore how data can be displayed and analyzed in order to better understand the economy around us in Chapter~\ref{chap:tmd}. Understanding the world is facilitated by the development of theories and models and then testing such theories with the use of data-driven models.

Trade is critical to individual well-being, whether domestically or internationally. To understand this trading process we analyze the behaviour of suppliers and buyers in the marketplace. Markets are formed by suppliers and demanders coming together for the purpose of trading. Thus, demand and supply are examined in Chapter~\ref{chap:classical} in tabular, graphical and mathematical form.}

