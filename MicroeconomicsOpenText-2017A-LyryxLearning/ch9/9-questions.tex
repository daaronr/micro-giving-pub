\newpage
\section*{Exercises for Chapter~\ref{chap:perfectcompetition}}

\begin{enumialphparenastyle}

% Solutions file for exercises opened
\Opensolutionfile{solutions}[solutions/ch9ex]

\begin{ex}\label{ex:ch9ex1}
Wendy's Window Cleaning is a small local operation. Winnie presently cleans the outside windows in her neighbours' houses for \$36 per house. She does ten houses per day. She is incurring total costs of \$420, and of this amount \$100 is fixed. The cost per house is constant.
\begin{enumerate}
	\item	What is the marginal cost associated with cleaning the windows of one house -- we know it is constant?
	\item	At a price of \$36, what is her break-even level of output (number of houses)?
	\item	If the fixed cost is `sunk' and she cannot increase her output in the short run, should she shut down?
\end{enumerate}
\begin{sol}
\begin{enumerate}
	\item	The $MC$ is \$32.
	\item	Her break-even level of output is 25 units.
	\item	No, because she can cover her variable costs. $TVC=\$320$; $TR=\$360$.
\end{enumerate}
\end{sol}
\end{ex}

\begin{ex}\label{ex:ch9ex2}
A manufacturer of vacuum cleaners incurs a constant variable cost of production equal to \$80. She can sell the appliances to a wholesaler for \$130. Her annual fixed costs are \$200,000.	How many vacuums must she sell in order to cover her total costs?
\begin{sol}
	For total revenue to equal total cost it must be the case that $130\times Q=200,000+80\times Q$. Therefore $Q=4,000$.
	
\end{sol}
\end{ex}

\begin{ex}\label{ex:ch9ex3}
For the vacuum cleaner producer in Exercise~\ref{ex:ch9ex2}:
\begin{enumerate}
	\item	Draw the $MC$ curve. 
	\item	Next, draw her $AFC$ and her $AVC$ curves. 
	\item	Finally, draw her $ATC$ curve. 
	\item	In order for this cost structure to be compatible with a perfectly competitive industry, what must happen to her $MC$ curve at some output level?
\end{enumerate}
\begin{sol}
\begin{enumerate}
	\item	The $MC$ is horizontal at \$80.
	\item	See diagram below.
	\item	See diagram below.
	\item	The $MC$ would have to increase at some point.
\end{enumerate}
\begin{center}
	\begin{tikzpicture}[background color=figurebkgdcolour,use background]
	\begin{axis}[
	axis line style=thick,
	every tick label/.append style={font=\footnotesize},
	ymajorgrids,
	grid style={dotted},
	every node near coord/.append style={font=\scriptsize},
	xticklabel style={rotate=90,anchor=east,/pgf/number format/1000 sep=},
	scaled y ticks=false,
	yticklabel style={/pgf/number format/fixed,/pgf/number format/1000 sep = \thinspace},
	xmin=1000,xmax=6000,ymin=0,ymax=300,
	y=1cm/40,
	x=1cm/600,
	x label style={at={(axis description cs:0.5,-0.05)},anchor=north},
	xlabel={Quantity},
	ylabel={Price},
	]
	\addplot[mccolour,dotted,ultra thick] table {
		X	Y
		1000	80
		6000	80
	};\addlegendentry {$AVC$ and $MC$}
	\addplot[afccolour,ultra thick,domain=1000:6000,samples=100] {200 / x * 1000};\addlegendentry {Average fixed cost}
	\addplot[datasetcolourtwo,dashed,ultra thick,domain=1000:6000,samples=100] {280 / x * 1000};\addlegendentry {Average total cost}
	\end{axis}
	\end{tikzpicture}
\end{center}
\end{sol}
\end{ex}

\begin{ex}\label{ex:ch9ex4}
Consider the supply curves of two firms in a competitive industry: $P=q_A$ and $P=2q_B$.
\begin{enumerate}
	\item	On a diagram, draw these two supply curves, marking their intercepts and slopes numerically (remember that they are really $MC$ curves). 
	\item	Now draw a supply curve that represents the combined supply of these two firms.
\end{enumerate}
\begin{sol}
	The market supply curve goes through the origin with a slope of	2/3. This follows from the fact that we can write the supply curves as $q_A=P$ and $q_B=0.5P$. Hence $Q=q_A+q_B=1.5P$; or $P=(2/3)Q$.
	
\end{sol}
\end{ex}

\begin{ex}\label{ex:ch9ex5}
Amanda's Apple Orchard Productions Limited produces 10,000 kilograms of apples per month. Her total production costs at this output level are \$8,000. Two of her many competitors have larger-scale operations and produce 12,000 and 15,000 kilos at total costs of \$9,500 and \$11,000 respectively. If this industry is competitive, on what segment of the $LAC$ curve are these producers producing? 
\begin{sol}
	Since the costs per unit are declining with output, they are producing on the downward-sloping segment of the $LATC$. To see this we need just calculate $ATC$ at each output.
	
\end{sol}
\end{ex}

\begin{ex}\label{ex:ch9ex6}
Consider the data in the table below. $TC$ is total cost, $TR$ is total revenue, and $Q$ is output.
\begin{center}
\begin{tabu} to \linewidth {|X[1,c]X[1,c]X[1,c]X[1,c]X[1,c]X[1,c]X[1,c]X[1,c]X[1,c]X[1,c]X[1,c]X[1,c]|}	\hline
\rowcolor{rowcolour}	\textbf{Q}	&	0	&	1	&	2	&	3	&	4	&	5	&	6	&	7	&	8	&	9	&	10	\\
						\textbf{TC}	&	10	&	18	&	24	&	31	&	39	&	48	&	58	&	69	&	82	&	100	&	120	\\
\rowcolor{rowcolour}	\textbf{TR}	&	0	&	11	&	22	&	33	&	44	&	55	&	66	&	77	&	88	&	99	&	110	\\	\hline
\end{tabu}
\end{center}
\begin{enumerate}
	\item	Add some extra rows to the table below and for each level of output calculate the $MR$, the $MC$ and total profit.
	\item	Next, compute $AFC$, $AVC$, and $ATC$ for each output level, and draw these three cost curves on a diagram.
	\item	What is the profit-maximizing output?
	\item	How can you tell that this firm is in a competitive industry?
\end{enumerate}
\begin{sol}
\begin{enumerate}
	\item	See the table.
	\item	See the figure below.
	\item	$Q=7$. At this output $MC=MR$.
	\item	Price is fixed.
\end{enumerate}
\begin{center}\footnotesize 
	\begin{tabu} to \linewidth {|X[0.8,c]X[1,c]X[1,c]X[1,c]X[1,c]X[1,c]X[1,c]X[1,c]X[1,c]X[1,c]X[1,c]X[1,c]|}	\hline
		\rowcolor{rowcolour}	\textbf{Q}	&	0	&	1	&	2	&	3	&	4	&	5	&	6	&	7	&	8	&	9	&	10	\\
		\textbf{TC}	&	10	&	18	&	24	&	31	&	39	&	48	&	58	&	69	&	82	&	100	&	120	\\
		\rowcolor{rowcolour}	\textbf{TR}	&	0	&	11	&	22	&	33	&	44	&	55	&	66	&	77	&	88	&	99	&	110	\\
		\textbf{Profit}	& $-10.00$ & $-7.00$ & $-2.00$ & 2.00 & 5.00 & 7.00 & 8.00 & 8.00 & 6.00 & $-1.00$ & $-10.00$ \\
		\rowcolor{rowcolour}\textbf{MR} &  & 11.00 & 11.00 & 11.00 & 11.00 & 11.00 & 11.00 & 11.00 & 11.00 & 11.00 & 11.00 \\
		\textbf{MC} &  & 8.00 & 6.00 & 7.00 & 8.00 & 9.00 & 10.00 & 11.00 & 13.00 & 18.00 & 20.00 \\
		\rowcolor{rowcolour}\textbf{AFC} &  & 10.00 & 5.00 & 3.33 & 2.50 & 2.00 & 1.67 & 1.43 & 1.25 & 1.11 & 1.00 \\
		\textbf{AVC} &  & 8.00 & 7.00 & 7.00 & 7.25 & 7.60 & 8.00 & 8.43 & 9.00 & 10.00 & 11.00 \\
		\rowcolor{rowcolour}\textbf{ATC} &  & 18.00 & 12.00 & 10.33 & 9.75 & 9.60 & 9.67 & 9.86 & 10.25 & 11.11 & 12.00 \\ \hline
	\end{tabu}
\end{center}
\begin{center}
	\begin{tikzpicture}[background color=figurebkgdcolour,use background]
	\begin{axis}[
	axis line style=thick,
	every tick label/.append style={font=\footnotesize},
	ymajorgrids,
	grid style={dotted},
	every node near coord/.append style={font=\scriptsize},
	xticklabel style={rotate=90,anchor=east,/pgf/number format/1000 sep=},
	scaled y ticks=false,
	yticklabel style={/pgf/number format/fixed,/pgf/number format/1000 sep = \thinspace},
	xmin=0,xmax=12,ymin=0,ymax=20,
	y=1cm/3.75,
	x=1cm/1.75,
	x label style={at={(axis description cs:0.5,-0.05)},anchor=north},
	xlabel={Quantity},
	ylabel={\$},
	]
	\addplot[afccolour,ultra thick,mark=*] table {
		X	Y
		1	10
		2	5
		3	3.33
		4	2.5
		5	2.0
		6	1.67
		7	1.43
		8	1.25
		9	1.11
		10	1
	};\addlegendentry {$AFC$}
	\addplot[avccolour,ultra thick,mark=square*] table {
		X	Y
		1	8
		2	7
		3	7
		4	7.25
		5	7.6
		6	8
		7	8.43
		8	9
		9	10
		10	11
	};\addlegendentry {$AVC$}
	\addplot[atccolour,ultra thick,mark=triangle*] table {
		X	Y
		1	18
		2	12
		3	10.33
		4	9.75
		5	9.6
		6	9.67
		7	9.86
		8	10.25
		9	11.11
		10	12
	};\addlegendentry {$ATC$}
	\end{axis}
	\end{tikzpicture}
\end{center}
\end{sol}
\end{ex}

\begin{ex}\label{ex:ch9ex7}
The market demand and supply curves in a perfectly competitive industry are given by: $Q_d=30,000-600P$ and $Q_s=200P-2000$.
\begin{enumerate}
	\item	Draw these functions on a diagram, and calculate the equilibrium price of output in this industry.
	\item	Now assume that an additional firm is considering entering. This firm has a short-run $MC$ curve defined by $MC=10+0.5q$, where $q$ is the firm's output. If this firm enters the industry and it knows the equilibrium price in the industry, what output should it produce?
\end{enumerate}
\begin{sol}
\begin{enumerate}
	\item	The equilibrium price is \$40 and the equilibrium quantity is 6,000. The price intercept for the demand equation 50, the quantity intercept 30,000. The price intercept for the supply equation is 10 and the quantity intercept 2,000.
	\item	With a perfectly competitive structure, this new firm cannot influence the price. Therefore it maximizes profit by setting $P=MC$. That is $40=10+0.5q$. Solving this equation yields a quantity value $q=60$.
\end{enumerate}
\end{sol}
\end{ex}

\begin{ex}\label{ex:ch9ex8}
Consider Exercise~\ref{ex:ch9ex7} again.
\begin{enumerate}
	\item	Suppose all of the existing firms have the same cost structure as the new entrant, how many firms are there in the industry?
	\item	Each firm in the industry has a total cost curve of the form $TC=400+10q+(1/4)q^2$. There is no distinction between the long run and short run -- there is only one possible size of firm. Derive the $ATC$ by dividing each term in the $TC$ curve by $q$, and calculate the cost per unit at the output being produced by each firm at the existing equilibrium price. 
	\item	Since you now know the price and cost per unit, calculate the profit that each firm is making.
	\item	Is the $ATC$ sloping up or down at the current equilibrium? [Hint: is the $MC$ above or below the $ATC$ at the chosen output?]
\end{enumerate}
\begin{sol}
\begin{enumerate}
	\item	If each firm produces 60 units then there must be 100 firms.
	\item	The $ATC$ is of the form $ATC=400/q+10+q/4$. Thus at an output of 60, $ATC=400/60+10+60/4=31.67$.
	\item	Profit is $(P-ATC)\times q=60\times (40-31.67)=\$500$.
	\item	$ATC$ must slope upwards because $MC$ is greater than $ATC$ here. 
\end{enumerate}
\end{sol}
\end{ex}

\begin{ex}\label{ex:ch9ex9}
Consider the long-run for the industry described in Exercise~\ref{ex:ch9ex8}.
\begin{enumerate}
	\item	What will happen to the number of firms in this industry in the long run?
	\item	The minimum of the $ATC$ curve for this firm occurs at a value of \$30. Given that you know this, what output will be produced in the industry in the long run?
	\item	Once you know the output produced in the industry, and the minimum of the $ATC$ curve, calculate the number of firms that will produce in the long run `normal profit' equilibrium.
\end{enumerate}
\begin{sol}
\begin{enumerate}
	\item	Entry will take place in view of supernormal profits.
	\item	Since this is a competitive industry the price in the LR equilibrium must equal the minimum of the LR $ATC$. Hence $P=30$. From the demand curve it follows that $Q=12,000$.
	\item	At a $MC=ATC=\$30$, it follows that $q=40$ for each firm. Hence there will be 300 firms.
\end{enumerate}
\end{sol}
\end{ex}

\begin{ex}\label{ex:ch9ex10}
Now consider what will happen in this industry in the very long run -- with technological change. The total cost curve becomes $TC=225+10q+(1/4)q^2$. The $MC$ remains unchanged, and the minimum of the $ATC$ now occurs at a value of \$25.
\begin{enumerate}
	\item	What is the price in the market in this new long run, and what quantity is traded? 
	\item	What quantity will each firm produce at this price?  
	\item	How many firms will there be in the industry?
\end{enumerate}
\begin{sol}
\begin{enumerate}
	\item	In a competitive industry the LR price equals the minimum of the LR $ATC$ curve. Hence $P=\$25$. From the demand curve above it follows that quantity demanded at this price is 15,000.
	\item	Each firm will produce where its $MC=P$. Hence equating $MC=P$ yields $10+0.5q=25$ implying $q=30$.
	\item	With a total quantity demand of 15,000 at this price there will be 500 firms. The reason we have more firms is that, with lower fixed costs, each firm attains the minimum of its $ATC$ at a lower level of output. 
\end{enumerate}
\end{sol}
\end{ex}

\begin{ex}\label{ex:ch9ex11}
Consider two firms in a perfectly competitive industry. They have the same $MC$ curves and differ only in having higher and lower fixed costs. Suppose the $ATC$ curves are of the form: $400/q+10+(1/4)q$ and $225/q+10+(1/4)q$. The $MC$ for each is a straight line: $MC=10+(1/2)q$.
\begin{enumerate}
	\item	Each $ATC$ curve is U shaped and has a minimum at the quantities 40 and 30 respectively. Draw two $ATC$ curve on the same diagram as the $MC$ curve that reflect this information.
	\item	Compute the break-even price for each firm.
	\item	Explain why both of these firms cannot continue to produce in the long run in a perfectly competitive market.
	\item	Could you derive an expression for the total variable cost curve for these firms, given the total cost curves are: $400+10q+(1/4)q^2$, and $225+10q+(1/4)q^2$?
\end{enumerate}
\begin{sol}
\begin{enumerate}
	\item	Each $ATC$ curve must intersect the $MC$ at the minimum of the $ATC$.
	\item	The breakeven price for each firm is the minimum of the firm's $ATC$.
	\item	The price in the market will be forced down to the level at which the most efficient producers can supply the market. Consequently the producer with the higher fixed cost will either have to adopt the technology of the lower-cost producer or exit the industry.
	\item	The total variable cost is the part of the total cost excluding the fixed component. Since the terms `400' and `225' are independent of output then the total variable cost curves for these firms are $10q+(1/4)q^2$. 
\end{enumerate}
\end{sol}
\end{ex}

% Closes solutions file for this chapter
\Closesolutionfile{solutions}

\end{enumialphparenastyle}