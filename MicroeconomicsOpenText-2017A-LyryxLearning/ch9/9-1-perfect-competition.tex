\section{The perfect competition paradigm}\label{sec:ch9sec1}

The essence of a competitive market is that it permits competition on the part of a large number of suppliers. Each supplier produces an output that forms a small part of the total market, and the sum of all of these individual outputs represents the production of that sector of the economy. Right off we might think of florists, barber shops or corner stores as enterprises that live in a competitive environment. 

At the other extreme, a market that has just a single supplier is a monopolist. For example, Via Rail is the only supplier of passenger rail services between Windsor, Ontario and the city of Quebec. 

We have used the word `paradigm' in the title to this section, and for good reason: we will develop once again a \textit{model} of supply behaviour for a market in which there are many small suppliers, producing essentially the same product, competing with one-another to meet the demands of consumers. 

The structures that we call perfect competition and monopoly are extremes in the market place. Most sectors of the economy lie somewhere between these limiting cases. For example, the market for internet services usually contains several providers in any area -- some provide using a fibre cable, others by satellite. The market for smart-phones is dominated by two major players -- Apple and Samsung. Hence, while these markets that have a limited number of suppliers are competitive in that they freely and fiercely compete for the buyer's expenditure, these are not perfectly competitive markets, because they do not have a very large number of suppliers.

\begin{DefBox}
A \textbf{perfectly competitive} industry is one in which many suppliers, producing an identical product, face many buyers, and no one participant can influence the market.
\end{DefBox}

In all of the models we develop in this chapter we will assume that the objective of firms is to maximize profit -- the difference between revenues and costs.

\begin{DefBox}
\textbf{Profit maximization} is the goal of competitive suppliers -- they seek to maximize the difference between revenues and costs.
\end{DefBox}

The presence of so many sellers in perfect competition means that each firm recognizes its own small size in relation to the total market, and that its actions have no perceptible impact on the market price for the good or service being traded. Each firm is therefore a \textit{price taker}---in contrast to a monopolist, who is a \textit{price setter}. 

The same ``smallness'' characteristic was assumed when we examined the demands of individuals earlier. Each buyer takes the price as given. He or she is not big enough to be able to influence the price. So while customers do not engage in bargaining when they go to the corner store to buy bread or milk, \textit{Emirate Airlines} engages in price negotiations when purchasing aircraft from \textit{Airbus}. The model underlying these types of transactions is examined in Chapter~\ref{chap:imperfectcompetition}. 

So when we describe a market as being perfectly competitive we do not mean that other forms of market are not competitive. All market structure are competitive in the sense that the suppliers wish to make profit and they produce as efficiently as possible in order to meet that goal.