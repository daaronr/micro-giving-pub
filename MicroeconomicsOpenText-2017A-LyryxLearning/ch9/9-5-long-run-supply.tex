\section{Long run industry supply}\label{sec:ch9sec5}

When aggregating the firm-level supply curves, as illustrated in Figure~\ref{fig:industrysupply}, we did not assume that all firms were identical. In that example, firm A has a cost structure with a lower $AVC$ curve, since its supply curve starts at a lower dollar value. This indicates that firm A may have a larger plant size than firm B -- one that puts A closer to the minimum efficient scale region of its long-run $ATC$ curve. 

Can firm B survive with his current scale of operation in the long run? Our industry dynamics indicate that it cannot. The reason is that, provided \textit{some} firms are making economic profits, new entrepreneurs will enter the industry and drive the price down to the minimum of the ATC curve \textit{of those firms who are operating with the lowest cost plant size}. B-type firms will therefore be forced either to leave the industry or to adjust to the least-cost plant size---corresponding to the lowest point on its \underbar{long-run} $ATC$ curve. Remember that the same technology is available to all firms; they each have the same long-run $ATC$ curve, and may choose different scales of operation in the short run, as illustrated in Figure~\ref{fig:firmdiffsize}. But in the long run they must all produce using the minimum-cost plant size, or else they will be driven from the market.

% Figure 9.7
\input{figures/ch9/ch9fig7}

This behaviour enables us to define a \terminology{long-run industry supply}. The long run involves the entry and exit of firms, and leads to a price corresponding to the minimum of the long-run $ATC$ curve. Therefore, if the long-run equilibrium price corresponds to this minimum, \textit{the long-run supply curve of the industry is defined by a particular price value---it is horizontal at that price}. More or less output is produced as a result of firms entering or leaving the industry, with those present always producing at the same unit cost in a long-run equilibrium.

\begin{DefBox}
\textbf{Industry supply in the long-run in perfect competition} is horizontal at a price corresponding to the minimum of the representative firm's long-run $ATC$ curve.
\end{DefBox}

This industry's long-run supply curve, $S_L$, and a particular short-run supply are illustrated in Figure~\ref{fig:lrdynamics}. Different points on $S_L$ are attained when demand shifts. Suppose that, from an initial equilibrium $Q_1$, defined by the intersection of $D_1$ and $S_1$, demand increases from $D_1$ to $D_2$ because of a growth in income. With a fixed number of firms, the additional demand can be met only at a higher price, where each existing firm produces more using their existing plant size. The economic profits that result induce new operators to produce. This addition to the industry's production capacity shifts the short-run supply outwards and price declines until normal profits are once again being made. The new long-run equilibrium is at $Q_2$, with more firms each producing at the minimum of their long-run $ATC$ curve, $P_E$.

% Figure 9.8
\input{figures/ch9/ch9fig8}

The same dynamic would describe the industry reaction to a decline in demand---price would fall, some firms would exit, and the resulting contraction in supply would force the price back up to the long-run equilibrium level. This is illustrated by a decline in demand from $D$ to $D_3$.

\subsection*{Increasing and decreasing cost industries}

While a horizontal long-run supply is the norm for perfect competition, in some industries costs increase with the scale of industry output; in others they decrease. This may be because all of the producers use a particular input that itself becomes more or less costly with the amount supplied.

For example, when the number of fruit and vegetable stands in a market becomes very large, the owners may find that their costs are reduced because their suppliers in turn can supply the inputs at a lower cost. Conversely, it is possible that costs may increase. Consider what frequently happens in a market such as large international airports: as more flights arrive and depart delays set in -- those intending to land may have to adopt a circling holding pattern while those departing encounter clearance delays. Costs increase with the scale of operation if we define the industry as the airport used by the many individual flight suppliers. In the case of declining costs, we have a \terminology{decreasing cost} industry; with rising costs we have an \terminology{increasing cost industry}. These conditions are reflected in the long-run industry supply curve by a downward-sloping segment or an upward sloping segment, as illustrated in Figure~\ref{fig:incdeccost}.

% Figure 9.9
\input{figures/ch9/ch9fig9}

\begin{DefBox}
\textbf{Increasing (decreasing) cost} industry is one where costs rise (fall) for each firm because of the scale of industry operation.
\end{DefBox}