\section{Efficient production}\label{sec:ch8sec1}

In Chapter~\ref{chap:firminvestorcapital}, we proposed that the firm is an organizational structure which is appropriate for producing those goods and services that the entrepreneur believes will yield her a profit. Efficient production is critical in any budget-driven organization. Just like private companies, public institutions are, and should be, concerned with costs and efficiency. We begin this chapter by exploring the concept of production efficiency, and progress to see how it translates into cost management.

The entrepreneur must employ factors of production in order to transform raw materials and other inputs into goods or services. She employs capital and Labour to attain this intermediate goal. The relationship between output and the inputs used in the production process is called a \terminology{production function}. It specifies how much output can be produced with given combinations of inputs. A production function is not restricted to profit-driven organizations. Municipal road repairs are carried out with labour and capital. Students are educated with teachers, classrooms, computers, and books. Each of these is a production process.

\begin{DefBox}
\textbf{Production function}: a technological relationship that specifies how much output can be produced with specific amounts of inputs.
\end{DefBox}

Producers strive to be efficient in their operations. Economists distinguish between two concepts of efficiency: One is \terminology{technological efficiency}; the other is \terminology{economic efficiency}. To illustrate the difference, consider the case of auto assembly in Oshawa Megamobile Inc., an auto manufacturer. Megamobile could assemble its vehicles either by using a large number of assembly workers and a plant that has a relatively small amount of machinery, or it could use fewer workers accompanied by more machinery in the form of robots. Each of these processes could be deemed technologically efficient, provided that there is no waste. If the workers without robots are combined with their capital to produce as much as possible, then that production process is technologically efficient. Likewise, in the scenario with robots, if the workers and capital are producing as much as possible, then that process too is efficient in the technological sense.

\begin{DefBox}
\textbf{Technological efficiency} means that the maximum output is produced with the given set of inputs.
\end{DefBox}

Economic efficiency is concerned with more than just technological efficiency. Since the entrepreneur's goal is to make profit, she must consider which technologically efficient process best achieves that objective. More broadly, any budget-driven process should focus on being economically efficient, whether in the public or private sector. An economically efficient production structure is the one that produces output at least cost. 

\begin{DefBox}
\textbf{Economic efficiency} defines a production structure that produces output at least cost.
\end{DefBox}

Auto-assembly plants the world over have moved to using robots during the last two decades. Why? The reason is not that robots were invented 20 years ago; they were invented long before that. The real reason is that, until recently, this technology was not economically efficient. Robots were too expensive; they were not capable of high-precision assembly. But once their cost declined and their accuracy increased they became economically efficient. 

The development of robots represented technological progress. When this progress reached a critical point, entrepreneurs embraced it.

To illustrate the point further, consider the case of garment assembly. There is no doubt that our engineers could make robots capable of joining the pieces of fabric that form garments. This is not beyond our technological abilities. Why, then, do we not have such capital-intensive production processes alongside Oshawa Megamobile? The answer is that, while such a concept could be technologically efficient, it would not be economically efficient. It is more profitable to use large amounts of labour and relatively traditional machines to assemble garments, particularly when labour in Asia is cheap and the garments can be shipped back to Canada inexpensively (in contrast to automobiles).

Efficiency in production is not limited to the manufacturing sector. Farmers must choose the optimal combination of labour, capital and fertilizer to use. In the health and education sectors, efficient supply involves choices on how many high- and low-skill workers to employ, how much traditional physical capital to use, how much information technology to use, based upon the productivity and cost of each.