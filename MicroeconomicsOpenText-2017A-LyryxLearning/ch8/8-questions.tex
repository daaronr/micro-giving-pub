\newpage
\section*{Exercises for Chapter~\ref{chap:prodcost}}

\begin{enumialphparenastyle}

% Solutions file for exercises opened
\Opensolutionfile{solutions}[solutions/ch8ex]

\begin{ex}\label{ex:ch8ex1}
Suppose you are told by a production engineer that the relationship between output $Q$ on the one hand and input, in the form of labour ($L$), on the other is $Q=5\sqrt{L}$. Capital is fixed, so we are operating in the short run.
\begin{enumerate}
	\item	Compute the output that can be produced in this firm using 1 through 9 units of labour by substituting these numbers into the production function.
	\item	Draw the resulting $TP$ curve to scale, relating output to labour.
	\item	Inspect your graph to see that it displays diminishing $MP$.
\end{enumerate}
\begin{sol}
\begin{enumerate}
	\item	For $1=1$ through 9 the output produced is 5.0, 7.07, 8.66, 10.0, 11.18, 12.25, 13.23, 14.14, 15.0.
	\item	See the figure below.
	\item	Note that total output increases at a diminishing rate -- the $MP$ is declining.
\end{enumerate}
\begin{center}
	\begin{tikzpicture}[background color=figurebkgdcolour,use background]
	\begin{axis}[
	axis line style=thick,
	every tick label/.append style={font=\footnotesize},
	ymajorgrids,
	grid style={dotted},
	every node near coord/.append style={font=\scriptsize},
	xticklabel style={rotate=90,anchor=east,/pgf/number format/1000 sep=},
	scaled y ticks=false,
	yticklabel style={/pgf/number format/fixed,/pgf/number format/1000 sep = \thinspace},
	xmin=0,xmax=10,ymin=0,ymax=16,
	y=1cm/2.5,
	x=1cm/1.3,
	x label style={at={(axis description cs:0.5,-0.05)},anchor=north},
	xlabel={Labour},
	ylabel={Quantity},
	]
	\addplot[datasetcolourone,ultra thick,domain=1:10] table {
		X	Y
		1	5.0
		2	7.07
		3	8.66
		4	10.0
		5	11.18
		6	12.25
		7	13.23
		8	14.14
		9	15.0
	};
	\end{axis}
	\end{tikzpicture}
\end{center}
\end{sol}	
\end{ex}

\begin{ex}\label{ex:ch8ex2}
The total product schedule for \textit{Primitive Products} is given in the table below.
\begin{center}
\begin{tabu} to \linewidth {|X[2,c]X[1,c]X[1,c]X[1,c]X[1,c]X[1,c]X[1,c]X[1,c]X[1,c]X[1,c]X[1,c]|}	\hline
\rowcolor{rowcolour}	\textbf{Output}	&	1	&	6	&	12	&	20	&	30	&	42	&	53	&	60	&	66	&	70	\\
						\textbf{Labour}	&	1	&	2	&	3	&	4	&	5	&	6	&	7	&	8	&	9	&	10	\\	\hline
\end{tabu}
\end{center}
\begin{enumerate}
	\item	Draw the total product function for this firm to scale, or by using a spreadsheet.
	\item	Calculate the $AP$ and draw the resulting relationship on a separate graph.
	\item	Calculate the $MP$ and draw the schedule on the same graph as the $AP$.
	\item	By inspecting the $AP$ and $MP$ curves, can you tell if you have drawn them correctly? Why?
\end{enumerate}
\begin{sol}
	For each level of labour used, its $AP$ is: 1.0, 3.0, 4.0, 5.0, 6.0, 7.0, 7.57, 7.5, 7.33, 7.0. and the $MP$ is: 1, 5, 6, 8, 10, 12, 11, 7, 6, 4. The $AP$ and $MP$ are graphed below. If the $MP$ cuts the $AP$ at the latter's maximum, your graph is likely correct.
	\begin{center}
	\begin{tikzpicture}[background color=figurebkgdcolour,use background]
		\begin{axis}[
		axis line style=thick,
		every tick label/.append style={font=\footnotesize},
		ymajorgrids,
		grid style={dotted},
		every node near coord/.append style={font=\scriptsize},
		xticklabel style={rotate=90,anchor=east,/pgf/number format/1000 sep=},
		scaled y ticks=false,
		yticklabel style={/pgf/number format/fixed,/pgf/number format/1000 sep = \thinspace},
		xmin=0,xmax=10,ymin=0,ymax=80,
		y=1cm/12,
		x=1cm/1.3,
		x label style={at={(axis description cs:0.5,-0.05)},anchor=north},
		xlabel={Labour},
		ylabel={Quantity},
		]
		\addplot[datasetcolourone,ultra thick,domain=1:10] table {
			X	Y
			1	1
			2	6
			3	12
			4	20
			5	30
			6	42
			7	53
			8	60
			9	66
			10	70
		};
		\end{axis}
	\end{tikzpicture}
	\end{center}
	\begin{center}
	\begin{tikzpicture}[background color=figurebkgdcolour,use background]
		\begin{axis}[
		axis line style=thick,
		every tick label/.append style={font=\footnotesize},
		ymajorgrids,
		grid style={dotted},
		every node near coord/.append style={font=\scriptsize},
		xticklabel style={rotate=90,anchor=east,/pgf/number format/1000 sep=},
		scaled y ticks=false,
		yticklabel style={/pgf/number format/fixed,/pgf/number format/1000 sep = \thinspace},
		xmin=0,xmax=10,ymin=0,ymax=14,
		y=1cm/2.1,
		x=1cm/1.3,
		x label style={at={(axis description cs:0.5,-0.05)},anchor=north},
		xlabel={Labour},
		ylabel={Product},
		]
		\addplot[apcolour,ultra thick,domain=1:10] table {
			X	Y
			1	1.0
			2	3.0
			3	4.0
			4	5.0
			5	6.0
			6	7.0
			7	7.57
			8	7.5
			9	7.33
			10	7.0
		};
		\addlegendentry{$AP$}
		\addplot[mpcolour,ultra thick,domain=1:10] table {
			X	Y
			1	1
			2	5
			3	6
			4	8
			5	10
			6	12
			7	11
			8	7
			9	6
			10	4
		};
		\addlegendentry{$MP$}
		\end{axis}
	\end{tikzpicture}
	\end{center}
\end{sol}
\end{ex}

\begin{ex}\label{ex:ch8ex3}
Return to Exercise~\ref{ex:ch8ex1} above and now calculate and plot the $AP$ and $MP$ curves.
\begin{sol}
	The $AP$ schedule is 5.0, 3.54, 2.89, 2.5, 2.24, 2.04, 1.89, 1.77, 1.67. The $MP$ schedule is: 5.0, 2.07, 1.59, 1.34, 1.18, 1.07, 0.98, 0.91,	0.86.
	
\end{sol}
\end{ex}

\begin{ex}\label{ex:ch8ex4}
A short-run relationship between output and total cost is given in the table below.
\begin{center}
\begin{tabu} to \linewidth {|X[2.5,c]X[1,c]X[1,c]X[1,c]X[1,c]X[1,c]X[1,c]X[1,c]X[1,c]X[1,c]X[1,c]|}	\hline
\rowcolor{rowcolour}	\textbf{Output}		&	0	&	1	&	2	&	3	&	4	&	5	&	6	&	7	&	8	&	9	\\
						\textbf{Total Cost}	&	12	&	27	&	40	&	51	&	61	&	70	&	80	&	91	&	104	&	120	\\	\hline
\end{tabu}
\end{center}
\begin{enumerate}
	\item	What is the total fixed cost of production in this example?
	\item	Add some rows to your table and calculate the $AFC$, $AVC$, and $ATC$ curves for each level of output.
	\item	Calculate the $MC$ of producing additional levels of output.
	\item	Graph each of these four cost curves using the information you have developed.
\end{enumerate}
\begin{sol}
\begin{enumerate}
	\item	Fixed cost is \$12.
	\item	See below.
	\item	See below.
	\begin{center}
	\begin{tabu} to \linewidth {|X[1,c]X[1,c]X[1,c]X[1,c]X[1,c]X[1,c]|}	\hline
		\rowcolor{rowcolour}\textbf{$Q$} & \textbf{$TC$} & \textbf{$AFC$} & \textbf{$AVC$} & \textbf{$ATC$} & \textbf{$MC$}	\\
		0	&	12	&		&		&		&		\\
		\rowcolor{rowcolour}1	&	27	&	12.00	&	15.00	&	27.00	&	15	\\
		2	&	40	&	6.00	&	14.00	&	20.00	&	13	\\
		\rowcolor{rowcolour}3	&	51	&	4.00	&	13.00	&	17.00	&	11	\\
		4	&	61	&	3.00	&	12.25	&	15.25	&	10	\\
		\rowcolor{rowcolour}5	&	70	&	2.40	&	11.60	&	14.00	&	9	\\
		6	&	80	&	2.00	&	11.33	&	13.33	&	10	\\
		\rowcolor{rowcolour}7	&	91	&	1.71	&	11.29	&	13.00	&	11	\\
		8	&	104	&	1.50	&	11.50	&	13.00	&	13	\\
		\rowcolor{rowcolour}9	&	120	&	1.33	&	12.00	&	13.33	&	16 	\\	\hline
	\end{tabu}
	\end{center}
	\item	See below.
	\begin{center}
	\begin{tikzpicture}[background color=figurebkgdcolour,use background]
		\begin{axis}[
		axis line style=thick,
		every tick label/.append style={font=\footnotesize},
		ymajorgrids,
		grid style={dotted},
		every node near coord/.append style={font=\scriptsize},
		xticklabel style={rotate=90,anchor=east,/pgf/number format/1000 sep=},
		scaled y ticks=false,
		yticklabel style={/pgf/number format/fixed,/pgf/number format/1000 sep = \thinspace},
		xmin=0,xmax=10,ymin=0,ymax=30,
		y=1cm/4,
		x=1cm/1.2,
		x label style={at={(axis description cs:0.5,-0.05)},anchor=north},
		xlabel={Labour},
		ylabel={},
		]
		\addplot[afccolour,ultra thick,domain=1:9] table {
			X	Y
			1	12.00
			2	6.00
			3	4.00
			4	3.00
			5	2.40
			6	2.00
			7	1.71
			8	1.50
			9	1.33
		};
		\addlegendentry{$AFC$}
		\addplot[avccolour,ultra thick,domain=1:9] table {
			X	Y
			1	15.00
			2	14.00
			3	13.00
			4	12.25
			5	11.60
			6	11.33
			7	11.29
			8	11.50
			9	12.00
		};
		\addlegendentry{$AVC$}
		\addplot[atccolour,ultra thick,domain=1:9] table {
			X	Y
			1	27.00
			2	20.00
			3	17.00
			4	15.25
			5	14.00
			6	13.33
			7	13.00
			8	13.00
			9	13.33
		};
		\addlegendentry{$ATC$}
		\addplot[mccolour,ultra thick,domain=1:9] table {
			X	Y
			1	15
			2	13
			3	11
			4	10
			5	9
			6	10
			7	11
			8	13
			9	16
		};
		\addlegendentry{$MC$}
		\end{axis}
	\end{tikzpicture}
	\end{center}
\end{enumerate}
\end{sol}
\end{ex}

\begin{ex}\label{ex:ch8ex5}
Bernie's Bagels can function with three different oven capacities. For any given amount of labour, a larger output can be produced with a larger-capacity oven. The cost of a small oven is \$100; the cost of a medium-sized oven is \$140; and the cost of a large oven is \$180. Each worker is paid \$60 per shift. The production levels for each plant size are given in the table below.
\begin{center}
\begin{tabu} to \linewidth {|X[1,c]X[1,c]X[1,c]X[1,c]|}	\hline
\rowcolor{rowcolour}	\textbf{Labour}	&	\textbf{Small-oven output}	&	\textbf{Med-oven output}	&	\textbf{Large-oven output}	\\
						1	&	15	&	20	&	22	\\
\rowcolor{rowcolour}	2	&	30	&	40	&	46	\\
						3	&	48	&	64	&	70	\\
\rowcolor{rowcolour}	4	&	56	&	74	&	80	\\	\hline
\end{tabu}
\end{center}
\begin{enumerate}
	\item	Compute and graph $AP$ and $MP$ curves for each size of operation.
	\item	Verify that the relationship between the $AP$ and $MP$ curves is as it should be.
\end{enumerate}
\begin{sol}
	See table below. By this point you should be able to take these data and put them into Excel, or some spreadsheet tool, and plot.
	\begin{center}\small
	\begin{tabu} to \linewidth {|X[0.8,c]X[0.8,c]X[0.8,c]X[0.8,c]X[0.8,c]X[1,c]X[1,c]X[1,c]X[1,c]X[1,c]X[1,c]X[1,c]X[1,c]|}	\hline
		\rowcolor{rowcolour}$L$ & $Q_s$ & $Q_m$ & $Q_l$ & $AP_s$ & $AP_m$ & $AP_l$ & $MP_s$ & $MP_m$ & $MP_l$ & $ATC_s$ & $ATC_m$ & $ATC_l$ \\
		1 & 15 & 20 & 22 & 15 & 20.00 & 22.00 & 15.00 & 20.00 & 22.00 & 10.67 & 10.00 & 10.91 \\
		\rowcolor{rowcolour}2 & 30 & 40 & 46 & 15 & 20.00 & 23.00 & 15.00 & 20.00 & 24.00 & 7.33 & 6.50 & 6.52 \\
		3 & 48 & 64 & 70 & 16 & 21.33 & 23.33 & 18.00 & 24.00 & 24.00 & 5.83 & 5.00 & 5.14 \\
		\rowcolor{rowcolour}4 & 56 & 74 & 80 & 14 & 18.50 & 20.00 & 8.00 & 10.00 & 10.00 & 6.07 & 5.14 & 5.25 \\
		  & 100 & 140 & 180 & & & & & & & & & \\ \hline
	\end{tabu}
	\end{center}
\end{sol}
\end{ex}

\begin{ex}\label{ex:ch8ex6}
Now consider the cost curves associated with the production functions in Exercise~\ref{ex:ch8ex5}.
\begin{enumerate}
	\item	Compute the $ATC$ schedule for the medium plant size and verify that it is U shaped.
\end{enumerate}
\begin{sol}
	The $ATC$ for each plant size is given in the table accompanying the preceeding question. Each one has a U shape.
	
\end{sol}
\end{ex}

\begin{ex}\label{ex:ch8ex7}
Now consider Exercise~\ref{ex:ch8ex6} in the longer term -- with variable plant size.
\begin{enumerate}
	\item	Compute the $ATC$ for the small plant and large plant.
	\item	By inspecting all three $ATC$ curves, what can you say about scale economies over the range of output being considered?
\end{enumerate}
\begin{sol}
	Since the minimum point on the $ATC$ curve of the large plant size lies above the minimum point on the $ATC$ curve for the medium plant size in the table above, diminishing returns to scale eventually set in.
	
\end{sol}
\end{ex}

\begin{ex}\label{ex:ch8ex8}
The table below defines the inputs required in the long run to produce three different output levels using different combinations of capital and labour. The cost of labour is \$5 and the cost of capital is \$2 per unit.
\begin{center}
\begin{tabu} to \linewidth {|X[2,c]X[1,c]X[1,c]X[1,c]X[1,c]X[1,c]X[1,c]|}	\hline
\rowcolor{rowcolour}	\textbf{Capital used}	&	4	&	2	&	7	&	4	&	11	&	8	\\
						\textbf{Labour used}	&	5	&	6	&	10	&	12	&	15	&	16	\\
\rowcolor{rowcolour}	\textbf{Output}			&	4	&	4	&	8	&	8	&	12	&	12	\\	\hline
\end{tabu}
\end{center}
\begin{enumerate}
	\item	Calculate the least-cost method of producing each of the three levels of output defined in the bottom row.
	\item	On a graph with cost on the vertical axis and output on the horizontal axis, plot the relationship you have calculated in (a) between cost per unit and output. You have now plotted the long-run average cost.
\end{enumerate}
\begin{sol}
\begin{enumerate}
	\item	The least cost method can be ascertained from the table below.
	\item	The $TC$ will have the cost points 33, 64 and 96, corresponding to the output levels 4, 8, 12. The $ATC$ will have the values \$8.25,	8.0, 8.0. These values can be plotted easily.
\end{enumerate}
\begin{center}
	\begin{tabu} to \linewidth {|X[1,c]X[1,c]X[1,c]X[1,c]|}	\hline
		\rowcolor{rowcolour} $K$ & $L$ & $Q$ & $TC$ \\
		4	&	5	&	4	&	33	\\
		\rowcolor{rowcolour}2	&	6	&	4	&	34	\\
		7	&	10	&	8	&	64	\\
		\rowcolor{rowcolour}4	&	12	&	8	&	68	\\
		11	&	15	&	12	&	97	\\
		\rowcolor{rowcolour}8	&	16	&	12	&	96 \\	\hline
	\end{tabu}
\end{center}
\end{sol}
\end{ex}

\begin{ex}\label{ex:ch8ex9}
Suppose now that the cost of capital in Exercise~\ref{ex:ch8ex8} rises to \$3 per unit. Graph the new long-run average cost curve, and compare its position with the curve in that question.
\begin{sol}
	The total cost column is now 37, 36; 71, 72; 108, 104, and the relevant least-cost values are therefore 36, 71, 104. The new LR $ATC$ will lie everywhere above the LR $ATC$ defined for the lower price of capital. 
	
\end{sol}
\end{ex}

\begin{ex}\label{ex:ch8ex10}
Consider the long-run total cost structure for the following two firms.
\begin{center}
\begin{tabu} to \linewidth {|X[1,c]X[1,c]X[1,c]X[1,c]X[1,c]X[1,c]X[1,c]X[1,c]|}	\hline
\rowcolor{rowcolour}	\textbf{Output}	&	1	&	2	&	3	&	4	&	5	&	6	&	7	\\
						\textbf{Firm A}	&	\$40	&	\$52	&	\$65	&	\$80	&	\$97	&	\$119	&	\$144	\\
\rowcolor{rowcolour}	\textbf{Firm B}	&	\$30	&	\$40	&	\$50	&	\$60	&	\$70	&	\$80	&	\$90	\\	\hline
\end{tabu}
\end{center}
\begin{enumerate}
	\item	Compute the long-run $ATC$ curve for each firm.
	\item	Plot these curves and examine the type of scale economies each firm experiences at different output levels.
\end{enumerate}
\begin{sol}
\begin{enumerate}
	\item	The costs are given in the table below.
	\item	Firm A experiences decreasing returns to scale at high outputs, whereas B does not.
\end{enumerate}
\begin{center}
	\begin{tabu} to \linewidth {|X[1,c]X[1,c]X[1,c]X[1,c]X[1,c]|}	\hline
		\rowcolor{rowcolour} $Q$ & $TC_A$ & $LAC_A$ & $TC_B$ & $LAC_B$ \\
		1	&	40	&	40.00	&	30.00	&	30.00	\\
		\rowcolor{rowcolour}2	&	52	&	26.00	&	40.00	&	20.00	\\
		3	&	65	&	21.67	&	50.00	&	16.67	\\
		\rowcolor{rowcolour}4	&	80	&	20.00	&	60.00	&	15.00	\\
		5	&	97	&	19.40	&	70.00	&	14.00	\\
		\rowcolor{rowcolour}6	&	119	&	19.83	&	80.00	&	13.33	\\
		7	&	144	&	20.57	&	90.00	&	12.86 	\\	\hline
	\end{tabu}
\end{center}
\end{sol}
\end{ex}

\begin{ex}\label{ex:ch8ex11}
Use the data in Exercise~\ref{ex:ch8ex10} to establish the $LMC$ for each of these two firms.
\begin{enumerate}
	\item	Check that these $LMC$ curves intersect with the $LAC$ curves appropriately.
\end{enumerate}
\begin{sol}
	$MC$ curve data are given in the table below. Firm B has constant marginal costs in the LR; hence never encounters decreasing returns to scale. Firm A's LR $MC$ intersects its LR $ATC$ at an output between 5 and 6 units, where the $ATC$ is at a minimum. Firm A's $MC$ lies everywhere below its $ATC$.
	\begin{center}
	\begin{tabu} to \linewidth {|X[2,c]X[1,c]X[1,c]X[1,c]X[1,c]X[1,c]X[1,c]X[1,c]|}	\hline
		\rowcolor{rowcolour} \textbf{Output}	&	1	&	2	&	3	&	4	&	5	&	6	&	7	\\
		\textbf{$MC$ Firm A}	&	40	&	12	&	13	&	15	&	17	&	22	&	25	\\
		\rowcolor{rowcolour}\textbf{$MC$ Firm B}	&	30	&	10	&	10	&	10	&	10	&	10	&	10	\\	\hline
	\end{tabu}
	\end{center}
\end{sol}
\end{ex}

\begin{ex}\label{ex:ch8ex12}
Consider a firm whose $ATC$ (in the short run) is given by: $ATC=2/q+1+q/8$.
\begin{enumerate}
	\item	Using a spreadsheet plot this curve for values of output in the range 1\dots 20. You should first compute the cost values in a table.
	\item	Next tabulate the total cost for each level of output, using the fact that total cost is the product of quantity times $ATC$.
	\item	Now compute the marginal cost in the fourth column of your table by observing how total cost changes at each level of output.
	\item	Plot the $MC$ curve on the same graph as the $ATC$ curve, and verify that it cuts the $ATC$ at its minimum point.
\end{enumerate}
\begin{sol}
	The table and graphic that answer all parts of this question are given below.
\begin{center}
	\begin{tabu} to \linewidth {|X[1,c]X[1,c]X[1,c]X[1,c]|}	\hline
		\rowcolor{rowcolour} Output & $ATC$ & $TC$ & $MC$ \\
							1	&	3.125	&	3.125	&	3.125	\\
		\rowcolor{rowcolour}2	&	2.25	&	4.50	&	1.38	\\
							3	&	2.04	&	6.13	&	1.625	\\
		\rowcolor{rowcolour}4	&	2.00	&	8.00	&	1.875	\\
							5	&	2.03	&	10.13	&	2.125	\\
		\rowcolor{rowcolour}6	&	2.08	&	12.50	&	2.375	\\
							7	&	2.16	&	15.13	&	2.625	\\
		\rowcolor{rowcolour}8	&	2.25	&	18.00	&	2.875	\\
							9	&	2.35	&	21.13	&	3.125	\\
		\rowcolor{rowcolour}10	&	2.45	&	24.50	&	3.375	\\
							11	&	2.56	&	28.13	&	3.625	\\
		\rowcolor{rowcolour}12	&	2.67	&	32.00	&	3.875	\\
							13	&	2.78	&	36.13	&	4.125	\\
		\rowcolor{rowcolour}14	&	2.89	&	40.50	&	4.375	\\
							15	&	3.01	&	45.13	&	4.625	\\
		\rowcolor{rowcolour}16	&	3.13	&	50.00	&	4.875	\\
							17	&	3.24	&	55.13	&	5.125	\\
		\rowcolor{rowcolour}18	&	3.36	&	60.50	&	5.375	\\
							19	&	3.48	&	66.13	&	5.625	\\
		\rowcolor{rowcolour}20	&	3.60	&	72.00	&	5.875	\\	\hline
	\end{tabu}
	\end{center}
	\begin{center}
	\begin{tikzpicture}[background color=figurebkgdcolour,use background]
		\begin{axis}[
		axis line style=thick,
		every tick label/.append style={font=\footnotesize},
		ymajorgrids,
		grid style={dotted},
		every node near coord/.append style={font=\scriptsize},
		xticklabel style={rotate=90,anchor=east,/pgf/number format/1000 sep=},
		scaled y ticks=false,
		yticklabel style={/pgf/number format/fixed,/pgf/number format/1000 sep = \thinspace},
		xmin=0,xmax=30,ymin=0,ymax=7,
		y=1cm/1.1,
		x=1cm/3,
		x label style={at={(axis description cs:0.5,-0.05)},anchor=north},
		xlabel={Quantity},
		ylabel={},
		]
		\addplot[atccolour,ultra thick,domain=1:9] table {
			X	Y
			1	3.125
			2	2.25
			3	2.04
			4	2.00
			5	2.03
			6	2.08
			7	2.16
			8	2.25
			9	2.35
			10	2.45
			11	2.56
			12	2.67
			13	2.78
			14	2.89
			15	3.01
			16	3.13
			17	3.24
			18	3.36
			19	3.48
			20	3.60
		};
		\addlegendentry{$ATC$}
		\addplot[datasetcolourthree,ultra thick,domain=1:9] table {
			X	Y
			1	3.125
			2	1.38
			3	1.625
			4	1.875
			5	2.125
			6	2.375
			7	2.625
			8	2.875
			9	3.125
			10	3.375
			11	3.625
			12	3.875
			13	4.125
			14	4.375
			15	4.625
			16	4.875
			17	5.125
			18	5.375
			19	5.625
			20	5.875
		};
		\addlegendentry{$MC$}
		\end{axis}
	\end{tikzpicture}
	\end{center}
\end{sol}
\end{ex}

\begin{ex}\label{ex:ch8ex13}
Suppose you are told that a firm has a long run average total cost that is defined by the following relationship: $LATC=4+48/q$.
\begin{enumerate}
	\item	Plot this curve for several values of $q$ in the range 1\dots 24.
	\item	What kind of returns to scale does this firm never experience?
	\item	By examining your graph of the $LATC$ curve, what will be the numerical value of the $ATC$ as output becomes very large?
	\item	Can you guess what the form of the $LMC$ curve is?
\end{enumerate}
\begin{sol}
\begin{enumerate}
	\item	See graphic below.
	\item	Decreasing returns to scale.
	\item	As $q$ becomes infinitely large the second term tends to zero, hence $ATC$ tends to \$.
	\item	$LMC=4$.
\end{enumerate}
\begin{center}
	\begin{tikzpicture}[background color=figurebkgdcolour,use background]
	\begin{axis}[
	axis line style=thick,
	every tick label/.append style={font=\footnotesize},
	ymajorgrids,
	grid style={dotted},
	every node near coord/.append style={font=\scriptsize},
	xticklabel style={rotate=90,anchor=east,/pgf/number format/1000 sep=},
	scaled y ticks=false,
	yticklabel style={/pgf/number format/fixed,/pgf/number format/1000 sep = \thinspace},
	xmin=0,xmax=20,ymin=0,ymax=60,
	y=1cm/9,
	x=1cm/2.2,
	x label style={at={(axis description cs:0.5,-0.05)},anchor=north},
	xlabel={Quantity},
	ylabel={LR $ATC$},
	]
	\addplot[atccolour,ultra thick,domain=1:20] {4+48/x};
	\end{axis}
	\end{tikzpicture}
\end{center}
\end{sol}
\end{ex}

% Closes solutions file for this chapter
\Closesolutionfile{solutions}

\end{enumialphparenastyle}