
\documentclass[handout]{beamer}
%%%%%%%%%%%%%%%%%%%%%%%%%%%%%%%%%%%%%%%%%%%%%%%%%%%%%%%%%%%%%%%%%%%%%%%%%%%%%%%%%%%%%%%%%%%%%%%%%%%%%%%%%%%%%%%%%%%%%%%%%%%%%%%%%%%%%%%%%%%%%%%%%%%%%%%%%%%%%%%%%%%%%%%%%%%%%%%%%%%%%%%%%%%%%%%%%%%%%%%%%%%%%%%%%%%%%%%%%%%%%%%%%%%%%%%%%%%%%%%%%%%%%%%%%%%%
\usepackage{mathpazo}
\usepackage{hyperref}
\usepackage{multimedia}

%TCIDATA{OutputFilter=LATEX.DLL}
%TCIDATA{Version=5.00.0.2552}
%TCIDATA{<META NAME="SaveForMode" CONTENT="1">}
%TCIDATA{Created=Thursday, November 29, 2007 11:33:44}
%TCIDATA{LastRevised=Friday, November 30, 2007 13:56:25}
%TCIDATA{<META NAME="GraphicsSave" CONTENT="32">}
%TCIDATA{<META NAME="DocumentShell" CONTENT="Other Documents\SW\Slides - Beamer">}
%TCIDATA{CSTFile=beamer.cst}

\newenvironment{stepenumerate}{\begin{enumerate}[<+->]}{\end{enumerate}}
\newenvironment{stepitemize}{\begin{itemize}[<+->]}{\end{itemize} }
\newenvironment{stepenumeratewithalert}{\begin{enumerate}[<+-| alert@+>]}{\end{enumerate}}
\newenvironment{stepitemizewithalert}{\begin{itemize}[<+-| alert@+>]}{\end{itemize} }
\usetheme{Madrid}
\input{tcilatex}

\begin{document}


%TCIMACRO{\TeXButton{BeginFrame}{\begin{frame}}}%
%BeginExpansion
\begin{frame}%
%EndExpansion

\frametitle{\textbf{Week VIII material: Reading/material/exercises}}

\textbf{Imperfect/Incomplete information I: Static Games, Bayesian Nash
Equilibrium; Adverse Selection}

\begin{itemize}
\item Random events and incomplete information:\ Watson, ch 24

\begin{itemize}
\item Ch 24, exercise 1
\end{itemize}

\item Bayesian Nash Equilibrium (BNE)\ {\small and Bayesian
Rationalizability:\ }Watson, ch. 26

\begin{itemize}
\item Ch 26, exercise 1, 2, 6
\end{itemize}

\item Trade with Incomplete Information:

\begin{itemize}
\item Watson ch. 27 \textquotedblleft Markets and Lemons" (Not Auctions)

\item Ch 27, Exercise 1

\item Article:\ Akerlof, George. \textquotedblleft The Market for
Lemons\textquotedblright\ 
\end{itemize}
\end{itemize}

%TCIMACRO{\TeXButton{EndFrame}{\end{frame}}}%
%BeginExpansion
\end{frame}%
%EndExpansion

%TCIMACRO{\TeXButton{BeginFrame}{\begin{frame}}}%
%BeginExpansion
\begin{frame}%
%EndExpansion

\frametitle{\textbf{Answers to last week's exercises:}}

\textbf{Chapter 22, Exercise 1:}\bigskip

\bigskip Repeated stage game, 2 periods, no discounting:

\begin{tabular}{|l|l|l|l|}
\hline
$_{1}^{\text{ \ \ \ \ }2}$ & \textbf{L} & \textbf{M} & \textbf{R} \\ \hline
\textbf{U} & 8,8 & 0,9 & 0,0 \\ \hline
\textbf{C} & 9,0 & 0,0 & 3,1 \\ \hline
\textbf{D} & 0,0 & 1,3 & 3,3 \\ \hline
\end{tabular}

\bigskip

%TCIMACRO{\TeXButton{Pause}{\pause}}%
%BeginExpansion
\pause%
%EndExpansion
Fully describe a SPNE in which players select (U,L) in the first period.

\textquotedblleft (U, L) can be supported as follows. If player 2 defects
((U,M) is played) in the first period, then the players coordinate on (C, R)
in the second period. If player 2 defects ((C, L) is played) in the first
period, then the players play (D, M) in the second period. Otherwise, the
players play (D, R) in the second period.\textquotedblright

%TCIMACRO{\TeXButton{EndFrame}{\end{frame}}}%
%BeginExpansion
\end{frame}%
%EndExpansion

%TCIMACRO{\TeXButton{BeginFrame}{\begin{frame}}}%
%BeginExpansion
\begin{frame}%
%EndExpansion

\textbf{Ch. 22, Exercise 2:}

{\footnotesize Find conditions on the discount factor under which
cooperation can be supported in the infinitely repeated games with the
following stage games (use the grim-trigger strategy profile):}

{\footnotesize \bigskip }

\begin{tabular}{|l|l|l|}
\hline
$_{1}^{\text{ \ \ \ \ \ \ }2}$ & {\footnotesize C} & {\footnotesize D} \\ 
\hline
{\footnotesize C} & {\footnotesize 2,2} & {\footnotesize 0,4} \\ \hline
{\footnotesize D} & {\footnotesize 4,0} & {\footnotesize 1,1} \\ \hline
\end{tabular}

%TCIMACRO{\TeXButton{Pause}{\pause}}%
%BeginExpansion
\pause%
%EndExpansion
{\footnotesize (a) To support cooperation, }$\delta ${\footnotesize \ must
be such that 2/(1 -}$\delta ${\footnotesize ) }$\geq ${\footnotesize \ 4+}$%
\delta ${\footnotesize /(1 -}$\delta ${\footnotesize ).}

{\footnotesize Solving for }$\delta ${\footnotesize , we see that
cooperation requires }$\delta ${\footnotesize \ }$\geq ${\footnotesize \ 2/3.%
}

{\footnotesize \bigskip }

%TCIMACRO{\TeXButton{Pause}{\pause}}%
%BeginExpansion
\pause%
%EndExpansion

\begin{tabular}{|l|l|l|}
\hline
$_{1}^{\text{ \ \ \ \ \ \ }2}$ & {\footnotesize C} & {\footnotesize D} \\ 
\hline
{\footnotesize C} & {\footnotesize 3,4} & {\footnotesize 0,7} \\ \hline
{\footnotesize D} & {\footnotesize 5,0} & {\footnotesize 1,2} \\ \hline
\end{tabular}

{\footnotesize (b) To support cooperation by player 1, it must be that }$%
\delta ${\footnotesize \ }$\geq ${\footnotesize \ 1/2. To support
cooperation by player 2, it must be that }$\delta ${\footnotesize \ }$\geq $%
{\footnotesize \ 3/5. Thus, we need }$\delta ${\footnotesize \ }$\geq $%
{\footnotesize \ 3/5.}

{\footnotesize \bigskip }

%TCIMACRO{\TeXButton{EndFrame}{\end{frame}}}%
%BeginExpansion
\end{frame}%
%EndExpansion

%TCIMACRO{\TeXButton{BeginFrame}{\begin{frame}}}%
%BeginExpansion
\begin{frame}%
%EndExpansion

\begin{tabular}{|l|l|l|}
\hline
$_{1}^{\text{ \ \ \ \ \ \ }2}$ & {\footnotesize C} & {\footnotesize D} \\ 
\hline
{\footnotesize C} & {\footnotesize 3,2} & {\footnotesize 7,0} \\ \hline
{\footnotesize D} & {\footnotesize 7,0} & {\footnotesize 2,1} \\ \hline
\end{tabular}

{\footnotesize (c) Cooperation by player 1 requires }$\delta ${\footnotesize %
\ }$\geq ${\footnotesize \ 4/5. Player 2 has no incentive to deviate in the
short run. Thus, it must be that }$\delta ${\footnotesize \ }$\geq $%
{\footnotesize \ 4/5.}

%TCIMACRO{\TeXButton{EndFrame}{\end{frame}}}%
%BeginExpansion
\end{frame}%
%EndExpansion

%TCIMACRO{\TeXButton{BeginFrame}{\begin{frame}}}%
%BeginExpansion
\begin{frame}%
%EndExpansion

\textbf{Ch 23, exercise 1:}

{\footnotesize Betrand oligopoly, n firms, demand Q=110-}$\underline{p}$%
{\footnotesize \ where }$\underline{p}=\min \{p_{1},...,p_{n}\}$%
{\footnotesize . \ If multiple firms set }$\underline{p}${\footnotesize \
then they equally share the demand. \ }$C(q)-10q${\footnotesize .
Infinitely/indefinitely repeated game with discount factor }$\delta $%
{\footnotesize .}

{\footnotesize \bigskip }

%TCIMACRO{\TeXButton{Pause}{\pause}}%
%BeginExpansion
\pause%
%EndExpansion
{\footnotesize (a) What strategies might support a collusive arrangement in
which all firms set the monopoly price }$p=60${\footnotesize \ each period?}

{\footnotesize Consider all players selecting }$p_{i}${\footnotesize \ }$%
=p=60${\footnotesize , until and unless someone defects . If someone
defects, then everyone chooses }$p_{i}${\footnotesize \ }$=p=10$%
{\footnotesize \ thereafter.\bigskip }

%TCIMACRO{\TeXButton{EndFrame}{\end{frame}}}%
%BeginExpansion
\end{frame}%
%EndExpansion

%TCIMACRO{\TeXButton{BeginFrame}{\begin{frame}}}%
%BeginExpansion
\begin{frame}%
%EndExpansion

%TCIMACRO{\TeXButton{Pause}{\pause}}%
%BeginExpansion
\pause%
%EndExpansion
{\footnotesize (b) Derive a condition on }$n${\footnotesize \ and }$\delta $%
{\footnotesize \ that guarantees (allows that)\ collusion can be
sustained:\bigskip }

{\footnotesize The quantity of each firm when they collude is }$%
qc=(110-60)/n=50/n${\footnotesize .}

{\footnotesize The profit of each firm under collusion is }$%
(50/n)60-10(50/n)=2500/n.${\footnotesize \bigskip }

%TCIMACRO{\TeXButton{Pause}{\pause}}%
%BeginExpansion
\pause%
%EndExpansion
{\footnotesize The profit under the Nash equilibrium of the stage game is 0.
If player i defects, she does so by setting }$p_{i}=60-\varepsilon $%
{\footnotesize ,}

{\footnotesize where }$\varepsilon ${\footnotesize \ is arbitrarily small.
Thus, the stage game payoff of defecting can be made arbitrarily close to }$%
2500.$

{\footnotesize \bigskip To support collusion, it must be that }$%
[2500/n][1/(1-\delta )]\geq 2500+0${\footnotesize , which simplifies to }$%
\delta \geq 1-1/n$

{\footnotesize (c) Collusion is \textquotedblleft easier\textquotedblright\
with fewer firms.}

%TCIMACRO{\TeXButton{EndFrame}{\end{frame}}}%
%BeginExpansion
\end{frame}%
%EndExpansion

%TCIMACRO{\TeXButton{BeginFrame}{\begin{frame}}}%
%BeginExpansion
\begin{frame}%
%EndExpansion

\frametitle{\textbf{"Incomplete information"}}

\textbf{Chance nodes}:\ nature chooses \textquotedblleft
types\textquotedblright , i.e., payoffs.

Fig 24.1: the gift game [Board]

\bigskip 

{\footnotesize If we think of nature as a \textquotedblleft
player\textquotedblright , this is technically equivalent to imperfect
information, but analysis is a bit different..}

{\footnotesize In this course we will examine games where nature moves first.%
}

%TCIMACRO{\TeXButton{EndFrame}{\end{frame}}}%
%BeginExpansion
\end{frame}%
%EndExpansion

%TCIMACRO{\TeXButton{BeginFrame}{\begin{frame}}}%
%BeginExpansion
\begin{frame}%
%EndExpansion

In this game, only the player himself knows his own type. [Explain payoffs.]

\bigskip

%TCIMACRO{\TeXButton{Pause}{\pause}}%
%BeginExpansion
\pause%
%EndExpansion
{\footnotesize \textquotedblleft Although a player's private information
often concerns his own payoffs, it sometimes has to do with the payoffs of
another player.\textquotedblright }

\bigskip

{\footnotesize \textquotedblleft Rational play will require a player who
knows his own type to think about what he would have done had he been
another type\textquotedblright\ Since other guy doesn't know which type you
are, you have to imagine what he thinks you might do if you were the other
type.}

\bigskip 
%TCIMACRO{\TeXButton{EndFrame}{\end{frame}}}%
%BeginExpansion
\end{frame}%
%EndExpansion

%TCIMACRO{\TeXButton{BeginFrame}{\begin{frame}}}%
%BeginExpansion
\begin{frame}%
%EndExpansion

\textbf{(Bayesian)\ Nash Equilibrium (BNE): }Action of each \textit{type of
player} is optimal (in EU)\ given actions chosen by all other types. \ 

%TCIMACRO{\TeXButton{EndFrame}{\end{frame}}}%
%BeginExpansion
\end{frame}%
%EndExpansion

%TCIMACRO{\TeXButton{BeginFrame}{\begin{frame}}}%
%BeginExpansion
\begin{frame}%
%EndExpansion

\frametitle{\textbf{Bayesian Normal Form}}

Compute payoffs by averaging over the random events. \ 

\bigskip 
%TCIMACRO{\TeXButton{Pause}{\pause}}%
%BeginExpansion
\pause%
%EndExpansion

Example: gift game in Bayesian normal form: {\footnotesize [Explain
construction, board]}

\bigskip 

\begin{tabular}{lll}
$_{1}$ \ \ \ $^{2}$ & \textbf{A} & \textbf{R} \\ 
\textbf{G}$^{F}$\textbf{G}$^{E}$ & 1,2p-1 & -1,0 \\ 
\textbf{G}$^{F}$\textbf{N}$^{E}$ & p,p & -p,0 \\ 
\textbf{N}$^{F}$\textbf{G}$^{E}$ & 1-p,p-1 & p-1,0 \\ 
\textbf{N}$^{F}$\textbf{N}$^{E}$ & 0,0 & 0,0%
\end{tabular}%
\bigskip 

%TCIMACRO{\TeXButton{EndFrame}{\end{frame}}}%
%BeginExpansion
\end{frame}%
%EndExpansion

%TCIMACRO{\TeXButton{BeginFrame}{\begin{frame}}}%
%BeginExpansion
\begin{frame}%
%EndExpansion

{\footnotesize Notice a strategy assigns an action for each type of player
1, while player 2 cannot condition on 1's type since she doesn't know
it.\bigskip }

{\footnotesize Once we construct this Bayesian normal form we can solve for
(Bayesian)\ NE, rationalizability, etc., in a standard way.\bigskip }

%TCIMACRO{\TeXButton{Pause}{\pause}}%
%BeginExpansion
\pause%
%EndExpansion
{\footnotesize Note that expected utility is additive and the actual choice
made as one type does not affect the utility if the other type is realised.
\ Thus, the problem can be stated equivalently as each type optimizing, or
as a player optimizing over his utility in expectation of type. \ A player
who optimizes for each type is the same as a player who \ optimizes in
expectation of types. \ That is, a player who optimizes both }$%
{\footnotesize U}^{F}${\footnotesize \ and }${\footnotesize U}^{E}$%
{\footnotesize \ separately is doing the same as a player who is choosing }$%
s^{F}${\footnotesize \ and s}$^{E}${\footnotesize \ to optimize }$%
pU^{F}+(1-p)U^{E}$.

%TCIMACRO{\TeXButton{EndFrame}{\end{frame}}}%
%BeginExpansion
\end{frame}%
%EndExpansion

%TCIMACRO{\TeXButton{BeginFrame}{\begin{frame}}}%
%BeginExpansion
\begin{frame}%
%EndExpansion

\textbf{Alternate depiction:\ Figure 24.3, 24.4}

[explain]\bigskip

%TCIMACRO{\TeXButton{EndFrame}{\end{frame}}}%
%BeginExpansion
\end{frame}%
%EndExpansion

%TCIMACRO{\TeXButton{BeginFrame}{\begin{frame}}}%
%BeginExpansion
\begin{frame}%
%EndExpansion

\textbf{Exercise: Fight or yield (from Osborne)}

\textbf{Players:}\ $\{1,2\}$

\textbf{States:}\ $\{Strong,Weak\}$

\textbf{Actions:} $(\{Fight,Yield\},\{Fight,Yield\})$

%TCIMACRO{\TeXButton{Pause}{\pause}}%
%BeginExpansion
\pause%
%EndExpansion
\textbf{Information structure: }Player 2 can see whether we are in a strong
or weak state (i.e., whether player 2 himself is strong or weak), player 1
can not.\medskip

%TCIMACRO{\TeXButton{Pause}{\pause}}%
%BeginExpansion
\pause%
%EndExpansion
\textbf{Payoffs (to actions, types):}%
\begin{eqnarray*}
\lbrack U_{1}(F,F),strong)),U_{2}(F,F),strong))] &\equiv &(\mathbf{U}%
((F,F),strong))=[-1,1] \\
\mathbf{U}((F,F),weak)) &=&[1,-1] \\
\mathbf{U}((F,Y),\omega )) &=&[1,0] \\
\mathbf{U}((Y,F),\omega )) &=&[0,1] \\
\mathbf{U}((Y,Y),\omega )) &=&[0,0]
\end{eqnarray*}

{\footnotesize Where }$\omega ${\footnotesize \ can represent either state.
\ I.,e, }$\omega \in \{weak,strong\}.$

{\footnotesize Note that it does not matter which state we are in if one or
both players yield.}

%TCIMACRO{\TeXButton{EndFrame}{\end{frame}}}%
%BeginExpansion
\end{frame}%
%EndExpansion

%TCIMACRO{\TeXButton{BeginFrame}{\begin{frame}}}%
%BeginExpansion
\begin{frame}%
%EndExpansion

\textbf{EU to types}

\textit{Outcomes:}

If strong (occurs with, say, probability $\alpha $): 
\begin{tabular}{|c|c|c|}
\hline
& \textbf{F} & \textbf{Y} \\ \hline
\textbf{F} & $-1,1$ & $1,0$ \\ \hline
\textbf{Y} & $0,1$ & $0,0$ \\ \hline
\end{tabular}

\bigskip

If weak (occurs with, say, probability $1-\alpha $): 
\begin{tabular}{|c|c|c|}
\hline
& \textbf{F} & \textbf{Y} \\ \hline
\textbf{F} & $1,-1$ & $1,0$ \\ \hline
\textbf{Y} & $0,1$ & $0,0$ \\ \hline
\end{tabular}%
\bigskip

%TCIMACRO{\TeXButton{EndFrame}{\end{frame}}}%
%BeginExpansion
\end{frame}%
%EndExpansion

%TCIMACRO{\TeXButton{BeginFrame}{\begin{frame}}}%
%BeginExpansion
\begin{frame}%
%EndExpansion

\textbf{How do we find (all)\ BNE?}

We can construct the `Bayesian normal form' and solve as usual. \ Or, we can
look for the best response function of each `player', here of each
player-type, and see where these BR functions meet.\medskip

%TCIMACRO{\TeXButton{Pause}{\pause}}%
%BeginExpansion
\pause%
%EndExpansion
Each type of player 2's BR are underlined below.\medskip

If strong (probability $\alpha $): 
\begin{tabular}{|c|c|c|}
\hline
& \textbf{F} & \textbf{Y} \\ \hline
\textbf{F} & -1,\underline{1} & 1,0 \\ \hline
\textbf{Y} & 0,\underline{1} & 0,0 \\ \hline
\end{tabular}

\medskip

If weak (probability $1-\alpha $): 
\begin{tabular}{|c|c|c|}
\hline
& \textbf{F} & \textbf{Y} \\ \hline
\textbf{F} & 1,-1 & 1,\underline{0} \\ \hline
\textbf{Y} & 0,\underline{1} & 0,0 \\ \hline
\end{tabular}%
\textbf{\medskip }

Let's call these response functions (for both types) $BR_{2}(\tau _{2},A_{1})
$ where $A_{1}$ represent's player 1's action and $\tau _{2}$ her type. \
Any BNE must involve these best responses!

%TCIMACRO{\TeXButton{EndFrame}{\end{frame}}}%
%BeginExpansion
\end{frame}%
%EndExpansion

%TCIMACRO{\TeXButton{BeginFrame}{\begin{frame}}}%
%BeginExpansion
\begin{frame}%
%EndExpansion

\textit{For player 1}

{\footnotesize Note, player 1 only gets one kind of signal here. \ Thus
there is only one `type' of player 1.}\bigskip

{\footnotesize Her move must be a best response in EU terms given the
strategies of the two types of player 2.}

{\footnotesize 1 cannot condition her strategy on 2's type.}\bigskip

{\footnotesize Thus we \textbf{don't} look for 1's BR in the previous
matrices above! \ We need to look for a BR in terms of expected values.}

%TCIMACRO{\TeXButton{EndFrame}{\end{frame}}}%
%BeginExpansion
\end{frame}%
%EndExpansion

%TCIMACRO{\TeXButton{BeginFrame}{\begin{frame}}}%
%BeginExpansion
\begin{frame}%
%EndExpansion

\textbf{Candidate 1:\ }First, let us consider the profile where player 1 is
playing F and 2 is best responding (using the functions on previous
slide).\bigskip 

Given 2's behaviour, player 1's EU to playing F is :

\[
U_{1}(F;BR_{2}(\tau _{2},F))=U_{1}(F;F,Y)=\alpha (-1)+(1-\alpha )1=1-2\alpha 
\]

If 1 deviates to yield, she gets:%
\[
U_{1}(Y;BR_{2}(\tau _{2},F))=U_{1}(Y;F,Y)=\alpha \ast 0+(1-\alpha )0=0 
\]

The BR for player 1 depends on $\alpha $. $\ $

\bigskip 

Thus $(F;F,Y)$ is a BNE$\ $if $U_{1}(F;BR_{2}(\tau _{2},F))\geq
U_{1}(Y;BR_{2}(\tau _{2},F))$, i.e,. if $1-2\alpha \geq 0$ $\mathbf{%
\Longrightarrow ~}\alpha \leq \frac{1}{2}$,

%TCIMACRO{\TeXButton{EndFrame}{\end{frame}}}%
%BeginExpansion
\end{frame}%
%EndExpansion

%TCIMACRO{\TeXButton{BeginFrame}{\begin{frame}}}%
%BeginExpansion
\begin{frame}%
%EndExpansion

\textbf{Candidate 2:\ }Next, let us consider the profile where player 1 is
playing Y and 2 is best responding.

Given 2's behaviour, player 1's EU to playing Y is :

\[
U_{1}(Y;BR_{2}(\tau _{2},Y))=U_{1}(Y;F,F)=\alpha \ast 0+(1-\alpha )\ast 0=0
\]

{\footnotesize Where the first argument of }$U_{1}(Y;F,F)$ {\footnotesize %
represents player 1's actions, and the second and third argument represent
the strong and weak player 2's actions, respectively.}\bigskip 

If 1 deviates to fight, she gets:%
\[
U_{1}(F;BR_{2}(\tau _{2},Y))=U_{1}(F;F,F)=\alpha \ast 1+(1-\alpha )\ast
-1=1-2\alpha 
\]

Again, the BR for player 1 depends on $\alpha $.\bigskip 

Thus $(Y;F,F)$ is a BNE$\ $if $U_{1}(Y;BR_{2}(\tau _{2},Y))\geq
U_{1}(F;BR_{2}(\tau _{2},Y))$, i.e,. if $1-2\alpha \leq 0$ $\mathbf{%
\Longrightarrow ~}\alpha \geq \frac{1}{2}$,

%TCIMACRO{\TeXButton{EndFrame}{\end{frame}}}%
%BeginExpansion
\end{frame}%
%EndExpansion

%TCIMACRO{\TeXButton{BeginFrame}{\begin{frame}}}%
%BeginExpansion
\begin{frame}%
%EndExpansion

Thus if $\alpha \leq \frac{1}{2}$ (player 2 is at least as likely to be weak
as strong), then $\{F;F,Y\}$ will be a BNE.

\bigskip 

%TCIMACRO{\TeXButton{Pause}{\pause}}%
%BeginExpansion
\pause%
%EndExpansion
If $\alpha \geq \frac{1}{2}$ (player 2 is as likely to be strong as weak),
then $\{Y;F,F\}$ will be a BNE.

\bigskip 

%TCIMACRO{\TeXButton{Pause}{\pause}}%
%BeginExpansion
\pause%
%EndExpansion
If $\alpha =\frac{1}{2}$ these will both be NE\ (as will a variety of mixed
strategies)

%TCIMACRO{\TeXButton{EndFrame}{\end{frame}}}%
%BeginExpansion
\end{frame}%
%EndExpansion

%TCIMACRO{\TeXButton{BeginFrame}{\begin{frame}}}%
%BeginExpansion
\begin{frame}%
%EndExpansion

{\footnotesize Note: the fact that 1's utility from deviating from either of
these candidates for BNE yielded 1's utility from the other candidate was a
coincidence. }\bigskip 

{\footnotesize \ In evaluating whether a strategy is a BNE, we need to see
that both players are best responding given the other player's strategy! \
2's strategy here \textit{can not} be conditional on 1's strategy -- this is
a simultaneous game. We check whether a player is better to play a strategy
that is part of a candidate BNE or to deviate unilaterally. \ }\bigskip 

{\footnotesize In dynamic (not simultaneous) games with perfect information
a strategy can be contingent on a previous player's action, but not here!}

%TCIMACRO{\TeXButton{EndFrame}{\end{frame}}}%
%BeginExpansion
\end{frame}%
%EndExpansion

%TCIMACRO{\TeXButton{BeginFrame}{\begin{frame}}}%
%BeginExpansion
\begin{frame}%
%EndExpansion

\textbf{Why would this be a (B)NE?}

For $\{F;F,Y\}$:

For player 2-strong, if player 1 is playing Fight then Fight is a BR.

For player 2-weak, if player 1 is playing Fight then Yield is a BR.

\bigskip

%TCIMACRO{\TeXButton{Pause}{\pause}}%
%BeginExpansion
\pause%
%EndExpansion
For player 1, if player 2-strong is playing Fight and player 2-weak is
playing Yield, player 1 will weigh the odds of facing a strong or weak
player 2. \ If it is more likely she is facing a weak player, it is better
(in expectation) to fight.

%TCIMACRO{\TeXButton{EndFrame}{\end{frame}}}%
%BeginExpansion
\end{frame}%
%EndExpansion

%TCIMACRO{\TeXButton{BeginFrame}{\begin{frame}}}%
%BeginExpansion
\begin{frame}%
%EndExpansion

\textbf{Example 2: Asymmetric information Cournot}

See other sheet (ignore part on signals)

%TCIMACRO{\TeXButton{EndFrame}{\end{frame}}}%
%BeginExpansion
\end{frame}%
%EndExpansion

%TCIMACRO{\TeXButton{BeginFrame}{\begin{frame}}}%
%BeginExpansion
\begin{frame}%
%EndExpansion

\frametitle{\textbf{Trade with Incomplete information: ``The Lemons
Problem''}} {\footnotesize Based on Akerlof (1970)}\medskip

\textbf{Players:} Buyer, Seller, Nature

\medskip 
%TCIMACRO{\TeXButton{Pause}{\pause}}%
%BeginExpansion
\pause%
%EndExpansion

\textbf{Play: }{\footnotesize [Board -- Sketch tree]}

1. Nature chooses quality of car $x$ according to probability distribution:

$x$ $\symbol{126}U(0,1)$ {\footnotesize (standard uniform distribution; go
over this)}\medskip

\medskip 
%TCIMACRO{\TeXButton{Pause}{\pause}}%
%BeginExpansion
\pause%
%EndExpansion
2. Seller proposes a \textit{tioli} price {\footnotesize (could have buyer
propose, or some bargaining process)}.

{\footnotesize Note: a BNE must specify what price each type of Seller would
propose, i.e., need to specify a function }$p^{\ast }(x)${\footnotesize %
.\medskip }

\medskip 
%TCIMACRO{\TeXButton{Pause}{\pause}}%
%BeginExpansion
\pause%
%EndExpansion
3. Buyer decides whether to buy or not.

\medskip 
%TCIMACRO{\TeXButton{Pause}{\pause}}%
%BeginExpansion
\pause%
%EndExpansion
{\footnotesize Note: An equilibrium profile must specify a complete strategy
for Buyer, i.e., a function }$\ {\footnotesize B}^{\ast }{\footnotesize %
(p)\in \{0,1\}}${\footnotesize . \ It is wlog to assume that such a strategy
will be \textquotedblleft Buy if }$p\leq \overline{p}${\footnotesize %
\textquotedblright\ for some maximum price }$\overline{p}${\footnotesize %
.\medskip }

%TCIMACRO{\TeXButton{EndFrame}{\end{frame}}}%
%BeginExpansion
\end{frame}%
%EndExpansion

%TCIMACRO{\TeXButton{BeginFrame}{\begin{frame}}}%
%BeginExpansion
\begin{frame}%
%EndExpansion

\textbf{Payoffs: }

Buyer values the car at $\frac{3}{2}x$;

thus gets payoff $u_{B}=\left\{ 
\begin{array}{c}
\frac{3}{2}x-p\text{ {\footnotesize if buy}} \\ 
0\text{ {\footnotesize otherwise }}%
\end{array}%
\right. $ {\footnotesize (we assume separable utility here) }

\medskip 
%TCIMACRO{\TeXButton{Pause}{\pause}}%
%BeginExpansion
\pause%
%EndExpansion
Seller values the car at $x,$

thus gets payoff $u_{s}=\left\{ 
\begin{array}{c}
p\text{ {\footnotesize if Buyer chooses \textquotedblleft
buy\textquotedblright }} \\ 
x\text{ {\footnotesize otherwise (keeps the car)}}%
\end{array}%
\right. $\medskip

\medskip 
%TCIMACRO{\TeXButton{Pause}{\pause}}%
%BeginExpansion
\pause%
%EndExpansion
\textbf{Efficient outcome:}\ Buyer must buy the car. \ 

{\footnotesize Any price }$p${\footnotesize \ s.t. }$x\leq p\leq \frac{3}{2}%
x ${\footnotesize \ will yield trade and will be a Pareto-improvement over
no trade.}

%TCIMACRO{\TeXButton{EndFrame}{\end{frame}}}%
%BeginExpansion
\end{frame}%
%EndExpansion

%TCIMACRO{\TeXButton{BeginFrame}{\begin{frame}}}%
%BeginExpansion
\begin{frame}%
%EndExpansion

\textbf{Equilibrium outcome:}

Buyer's equilibrium strategy must set $\bar{p}=\frac{3}{2}E(x|p)$. \ 
{\footnotesize I.e., buy if the price is below }$\frac{3}{2}${\footnotesize %
\ of buyer's expectation over the quality of the car -- where that
expectation will depend on the price offered!}\bigskip 
%TCIMACRO{\TeXButton{Pause}{\pause}}%
%BeginExpansion
\pause%
%EndExpansion

What is $E(x|p)$? \ {\footnotesize It will be based on a \textquotedblleft
belief\textquotedblright\ about what seller of each type }$x${\footnotesize %
\ will do, and thus the probability of each }$x${\footnotesize \ given the
observed }$p${\footnotesize \ -- this foreshadows perfect Bayesian
equilibrium (see next week).} 

%TCIMACRO{\TeXButton{EndFrame}{\end{frame}}}%
%BeginExpansion
\end{frame}%
%EndExpansion

%TCIMACRO{\TeXButton{BeginFrame}{\begin{frame}}}%
%BeginExpansion
\begin{frame}%
%EndExpansion

Note Seller will never set $p<x$, since Seller would prefer to keep the car
at such a price. \ Thus, observing $p$ tells Buyer that $x\leq p$.
(Thi\bigskip 

%TCIMACRO{\TeXButton{Pause}{\pause}}%
%BeginExpansion
\pause%
%EndExpansion
{\footnotesize For a standard uniform distribution,} ${\footnotesize %
E(x|x\leq p)=}\frac{1}{2}{\footnotesize p}$. {\footnotesize Thus} $%
{\footnotesize E(x|p)=}\frac{1}{2}{\footnotesize p}$.

%TCIMACRO{\TeXButton{Pause}{\pause}}%
%BeginExpansion
\pause%
%EndExpansion
Thus, $\bar{p}=\frac{3}{2}E(x|p)=\frac{3}{2}\times \frac{1}{2}p=\frac{3}{4}p$%
.\bigskip

%TCIMACRO{\TeXButton{Pause}{\pause}}%
%BeginExpansion
\pause%
%EndExpansion
What does this mean? \ For any price $p$ offered, the Buyer values the
product at $\frac{3}{4}p$. \ Thus, for any price offered, the Buyer will
reject! \ Thus, the only equilibrium\ must involve no trade, and thus be
inefficient!

%TCIMACRO{\TeXButton{EndFrame}{\end{frame}}}%
%BeginExpansion
\end{frame}%
%EndExpansion

%TCIMACRO{\TeXButton{BeginFrame}{\begin{frame}}}%
%BeginExpansion
\begin{frame}%
%EndExpansion

Any profile $\{p(x)=$whatever;reject always$\}$ {\footnotesize (paired with
certain beliefs ... see next week) }is an equilibrium.

\bigskip

We can allow $\{p(x)=whatever,p(0)=0$; accept only if $p=0\}$ ,
but~who~cares?

\bigskip

{\footnotesize Note: See Watson's \textquotedblleft Jerry and
Freddy\textquotedblright\ for a simpler example.}

%TCIMACRO{\TeXButton{EndFrame}{\end{frame}}}%
%BeginExpansion
\end{frame}%
%EndExpansion

\end{document}
