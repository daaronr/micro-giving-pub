\documentclass{article}
%%%%%%%%%%%%%%%%%%%%%%%%%%%%%%%%%%%%%%%%%%%%%%%%%%%%%%%%%%%%%%%%%%%%%%%%%%%%%%%%%%%%%%%%%%%%%%%%%%%%%%%%%%%%%%%%%%%%%%%%%%%%%%%%%%%%%%%%%%%%%%%%%%%%%%%%%%%%%%%%%%%%%%%%%%%%%%%%%%%%%%%%%%%%%%%%%%%%%%%%%%%%%%%%%%%%%%%%%%%%%%%%%%%%%%%%%%%%%%%%%%%%%%%%%%%%
%TCIDATA{OutputFilter=LATEX.DLL}
%TCIDATA{Version=5.00.0.2552}
%TCIDATA{<META NAME="SaveForMode" CONTENT="1">}
%TCIDATA{Created=Wednesday, November 28, 2007 10:37:42}
%TCIDATA{LastRevised=Saturday, December 08, 2007 14:42:26}
%TCIDATA{<META NAME="GraphicsSave" CONTENT="32">}
%TCIDATA{<META NAME="DocumentShell" CONTENT="Standard LaTeX\Blank - Standard LaTeX Article">}
%TCIDATA{CSTFile=40 LaTeX article.cst}

\newtheorem{theorem}{Theorem}
\newtheorem{acknowledgement}[theorem]{Acknowledgement}
\newtheorem{algorithm}[theorem]{Algorithm}
\newtheorem{axiom}[theorem]{Axiom}
\newtheorem{case}[theorem]{Case}
\newtheorem{claim}[theorem]{Claim}
\newtheorem{conclusion}[theorem]{Conclusion}
\newtheorem{condition}[theorem]{Condition}
\newtheorem{conjecture}[theorem]{Conjecture}
\newtheorem{corollary}[theorem]{Corollary}
\newtheorem{criterion}[theorem]{Criterion}
\newtheorem{definition}[theorem]{Definition}
\newtheorem{example}[theorem]{Example}
\newtheorem{exercise}[theorem]{Exercise}
\newtheorem{lemma}[theorem]{Lemma}
\newtheorem{notation}[theorem]{Notation}
\newtheorem{problem}[theorem]{Problem}
\newtheorem{proposition}[theorem]{Proposition}
\newtheorem{remark}[theorem]{Remark}
\newtheorem{solution}[theorem]{Solution}
\newtheorem{summary}[theorem]{Summary}
\newenvironment{proof}[1][Proof]{\noindent\textbf{#1.} }{\ \rule{0.5em}{0.5em}}
\input{tcilatex}

\begin{document}


\section{Week IX material: Reading/material/exercises}

\textbf{Imperfect/Incomplete information II: }Dynamic Games, Perfect
Bayesian Equilibrium, Signaling games

\begin{itemize}
\item Private information and sequential action, Perfect Bayesian
Equilibrium (PBE):\ Watson, ch 28

\begin{itemize}
\item Ch 28, exercise 6
\end{itemize}

\item Signaling:\ \ Watson, Ch 29; through top of page 285

\begin{itemize}
\item Ch 29, exercise 1-3

\item Question at bottom (revised from 2006 exam)
\end{itemize}
\end{itemize}

\bigskip

\section{Outline}

We now (re)-examine games with incomplete information, where nature moves
first, but now allowing sequential moves by the strategic players. \ We also
look at games of imperfect information without a \textquotedblleft
nature\textquotedblright\ player {\footnotesize (this is not in the book
exactly, but learning PBE should be sufficient to apply to such games)}, but
where a player may have a nonsingleton information set that requires her to
form beliefs about \textquotedblleft out-of-equilibrium\textquotedblright\
play. \bigskip

\begin{quote}
As we consider progressively richer games, we progressively strengthen the
equilibrium concept, in order to rule out implausible equilibria in the
richer games that would survive if we applied equilibrium concepts suitable
for simpler games (Gibbons, p. 173)
\end{quote}

Nash Equilibrium (NE)$\ \mathbf{\rightarrow }$ subgame-perfect NE (SPNE) $%
\mathbf{\rightarrow }$ Bayesian NE $\mathbf{\rightarrow }$ Perfect Bayesian
Equilibrium

Static games of perfect information $\mathbf{\rightarrow }$ dynamic games of
perfect (or `some imperfect') information $\mathbf{\rightarrow }$ static
games of incomplete information $\mathbf{\rightarrow }$dynamic games of
incomplete information

\bigskip

\section{Some examples:}

\subsection{\textbf{`Poker'}}

\[
\FRAME{itbphF}{5in}{4in}{0in}{}{}{Figure}{\special{language "Scientific
Word";type "GRAPHIC";maintain-aspect-ratio TRUE;display "USEDEF";valid_file
"T";width 5in;height 4in;depth 0in;original-width 4.6799in;original-height
3.584in;cropleft "0";croptop "1";cropright "1";cropbottom "0";tempfilename
'JSHANV2Y.wmf';tempfile-properties "XPR";}}
\]

What will/should player 2 believe [about \textit{Chance's }move; i.e., about
1's type] if 2 \textquotedblleft has the move\textquotedblright ?

\bigskip

\subsection{Challenger who may be ready or unready and Incumbent}

\begin{equation}
\FRAME{itbphFU}{5.0004in}{4in}{0in}{\Qcb{Fig. 317.1}}{}{Figure}{\special%
{language "Scientific Word";type "GRAPHIC";maintain-aspect-ratio
TRUE;display "USEDEF";valid_file "T";width 5.0004in;height 4in;depth
0in;original-width 5.6886in;original-height 3.5201in;cropleft "0";croptop
"1";cropright "1";cropbottom "0";tempfilename
'JSHANV2Z.wmf';tempfile-properties "XPR";}}
\end{equation}

Is \{Out;Fight\} a reasonable \textquotedblleft equilibrium
prediction\textquotedblright\ here? \ Note: it is a NE and thus a SPNE since
there are no proper subgames!

$\Rightarrow $We need the refinement \textquotedblleft Sequential
rationality.\textquotedblright

\bigskip

\bigskip

\subsection{Signaling game}

\[
\FRAME{itbphF}{5in}{4in}{0in}{}{}{Figure}{\special{language "Scientific
Word";type "GRAPHIC";maintain-aspect-ratio TRUE;display "USEDEF";valid_file
"T";width 5in;height 4in;depth 0in;original-width 7.8006in;original-height
5.7164in;cropleft "0";croptop "1";cropright "1";cropbottom "0";tempfilename
'JSHANV2W.wmf';tempfile-properties "XPR";}}
\]

What will/should the receiver believe [about \textit{Chance's }move; i.e.,
1's type] if observes \textquotedblleft Good\textquotedblright\ signal? \
What if he observes \textquotedblleft Bad\textquotedblright\ signal?

\bigskip

\section{Motivation: Sequential Rationality}

\subsection{Example 322.1: \ entry game, Challenger who may be ready or
unready and Incumbent}

\textbf{Extensive form: }[BOARD]\textbf{\bigskip }

\begin{equation}
\FRAME{itbphFU}{5.0004in}{3.9998in}{0in}{\Qcb{}}{}{Figure}{\special{language
"Scientific Word";type "GRAPHIC";maintain-aspect-ratio TRUE;display
"USEDEF";valid_file "T";width 5.0004in;height 3.9998in;depth
0in;original-width 5.6886in;original-height 3.5201in;cropleft "0";croptop
"1";cropright "1";cropbottom "0";tempfilename
'JSHANV30.wmf';tempfile-properties "XPR";}}
\end{equation}

Note:\ no proper subgames (remember, a subgame must begin at a singleton
information set)

So SPNE is the same as NE here.\bigskip

\bigskip

\bigskip

\textbf{Strategic form:}

\begin{equation}
\begin{tabular}{ccc}
& \textbf{A} & \textbf{F} \\
\textbf{R} & 3,3 & 1,1 \\
\textbf{U} & \underline{\textit{4}},\underline{3} & 0,3 \\
\textbf{O} & 2,4 & \underline{\textit{2,4}}%
\end{tabular}%
\end{equation}

\bigskip

There are two NE here: \{Unready,Acquiesce\} and \{Out,Fight\}?.\

Are they both reasonable?

\bigskip

Subgame perfection doesn't eliminate either (no proper subgames).

\bigskip

But we know it is suboptimal for the incumbent to fight once the challenger
has entered, whether or not the challenger is prepared. \ So `fight'
wouldn't seem to be a good strategy ever, and thus \{Out,Fight\} depends on
an incredible threat.

\bigskip

\bigskip

\bigskip

\textit{We could add a requirement:}

\textbf{Optimal at each information set (\textbf{Sequential Rationality}):}
Each player's strategy is optimal (for at least some node in the information
set) whenever she has to move.

I.e., each player's strategy must be optimal at each of her information sets.

\bigskip

The strategy `Fight' is not optimal at each information set here -- it is
not optimal for the incumbent to fight at the incumbent's information set
(when he has the move, i.e., after some form of entry), niether whether he
is at the node following Unready nor whether he is at the node following
Ready

\bigskip

Thus, only the NE strategy profile \{Unready,Acquiesce\} meets the
requirement of optimality at each information set.

\bigskip

\section{Motivation: Consistency of Beliefs}

\subsection{Variant of this game:}

\textbf{Extensive form: }[BOARD]

\[
\FRAME{itbphFU}{5.0004in}{3.9998in}{0in}{\Qcb{}}{}{Figure}{\special{language
"Scientific Word";type "GRAPHIC";maintain-aspect-ratio TRUE;display
"USEDEF";valid_file "T";width 5.0004in;height 3.9998in;depth
0in;original-width 5.4803in;original-height 3.4396in;cropleft "0";croptop
"1";cropright "1";cropbottom "0";tempfilename
'JSHANV31.wmf';tempfile-properties "XPR";}}
\]

Strategic form: [BOARD]

\[
\begin{tabular}{ccc}
& \textbf{A} & \textbf{F} \\
\textbf{R} & 3,2 & 1,1 \\
\textbf{U} & \underline{\textit{4}}\underline{\textit{,2}} & 0,3 \\
\textbf{O} & 2,4 & \underline{\textit{2,4}}%
\end{tabular}%
\]

There are again 2 pure strategy NE here. \

But are they both reasonable?

\bigskip

Now the incumbent prefers to fight if the challenger enters unready, but
prefers to acquiese if the challenger enters ready. \ So there is no clear
optimal move at this information set -- it depends whether the challenger
prepared.

\bigskip

So, is Out, Fight a reasonable equilibrium?

Perhaps if the incumbent thinks `any challenger who enters will not prepare.'

At this node (following Unready) the action `Fight' is optimal.

Thus the requirement of `optimality at each information set' does not rule
anything out here.

\bigskip

Note that whether fight is `optimal' depends on the incumbent's \textbf{%
beliefs }(about out-of-equilibrium play)!

\bigskip

\section{Definitions: Beliefs, sequential equilibrium, assesments}

\textbf{Rem: }\bigskip

In the static game beliefs were `correct' in equilibrium.

\bigskip

For extensive game this is less clearly defined -- some nodes are never
reached, so there is no obvious `correct' belief if those nodes were reached!

\bigskip

$\mathbf{\Longrightarrow }$\textit{We will define equilibrium in terms of
pairs (assesments): Strategy profiles \& Collections of beliefs}

\bigskip

\textbf{Beliefs }

At an information set with more than one history (element/node) the player
who has the move `forms a belief over the histories' (previous moves;
including moves by nature).

At each information set a \textit{belief }assigns a probability to each
possible history,

...or assigns a probability distribution over all the possible histories (in
an infinite strategy set game)..

\bigskip

\textbf{Belief System}

A \textit{belief system }is a collection of \textit{beliefs}, one for each
information set of each player.

\bigskip

\textbf{Assessment}

An \textit{assessment} pairs a profile of strategies and a belief system.

\bigskip

\bigskip

\section{Equilibrium Concept:\ Perfect Bayesian Equilibrium -- PBE}

\textbf{(also see Weak Sequential Equilibrium )}

\bigskip

\textbf{Perfect Bayesian Equilibrium \footnote{%
There are small difference between PBE\ and WSE but I will not discuss them
here.}}

An assesment is a \textit{PBE }if it meets two requirements

\bigskip

\textit{Requirement 1}\textbf{\ -- Sequentially rationality:} Each strategy
is optimal for each player when he has the move given his belief (and the
other players' subsequent prescribed `continuation' strategies)

\bigskip

\textit{Requirement 2 -- }\textbf{\ Consistency of Beliefs with Strategies: }%
$\ $Each player's belief is consistent with the strategy profile (this will
be made more specific later)..

\bigskip

\textit{Sequential rationality} rules out off-the-equilibrium-path behavior
that is `suboptimal' (given consistent beliefs and given the other player's
subsequent prescribed strategies)

\bigskip

\subsection{Example:}

\begin{equation}
\FRAME{itbphFU}{5.0004in}{4in}{0in}{\Qcb{Fig. 326.1}}{}{Figure}{\special%
{language "Scientific Word";type "GRAPHIC";maintain-aspect-ratio
TRUE;display "USEDEF";valid_file "T";width 5.0004in;height 4in;depth
0in;original-width 5.3599in;original-height 4.4317in;cropleft "0";croptop
"1";cropright "1";cropbottom "0";tempfilename
'JSHANV32.wmf';tempfile-properties "XPR";}}
\end{equation}

\begin{itemize}
\item Above, suppose that player 1 plays the strategy highlighted in blue
(player 2's potential moves are in red)

\item Given player 1's strategy what should 2 suppose if he `gets the move'
(i.e., arrives at information set \{C,D\}?

\item Let's say (we are trying this out), that 2 supposes that if 1 decides
to deviate from his strategy (E) 1 would plays C with probability $\frac{2}{3%
}$ and D with probability $\frac{2}{3}$.

\item I.e., 2's \textit{beliefs} at this information set assign the
probabilities $\frac{2}{3}$ and $\frac{1}{3}$ to C and D, respectively, as
shown in brackets.

\item We assume that 2 will believe that, even after making this `mistake'
or `deviation' 1 will subsequently play rationally, i.e., play J if she gets
to the node following C,F.
\end{itemize}

\bigskip

\textit{Given the assumptions above, }\textbf{what is 2's `best'
(sequentially rational)\ action at the information set \{C,D\}?}

Let's consider the EU\ payoffs to each move:

\begin{itemize}
\item If 1 played C, 2's best move is G, yielding payoff 1. \ If 1 played D,
G yields 0.

\item If 1 played D, 2's best move is F, yielding payoff 1. If 1 played C, F
yields 0.

\item So which action is better depends on the probability 1 played C versus
D

\item (2 believes that) C occured with probability $\frac{2}{3}$ and D with
probability $\frac{1}{3}$

\item Thus 2's EU payoffs to playing G are:
\end{itemize}

\begin{eqnarray*}
EU_{2}(G) &=&\Pr (C)\times U_{2}(\{C;G\})+\Pr (D)\times U_{2}(\{D;G\}) \\
&=&\frac{2}{3}\times 1+\frac{1}{3}\times 0=\frac{2}{3}
\end{eqnarray*}

\begin{itemize}
\item On the other hand, 2's EU payoffs to playing F are:
\end{itemize}

\begin{eqnarray*}
EU_{2}(F) &=&\Pr (C)\times U_{2}(\{C;F\})+\Pr (D)\times U_{2}(\{D;F\}) \\
&=&\frac{2}{3}\times 0+\frac{1}{3}\times 1=\frac{1}{3}
\end{eqnarray*}

So, given the assumptions above, given 2's beliefs and 1's optimal strategy
(for the rest of the game), if 2 is at the information set \{C,D\}, his best
move is G. \ G is the `sequentially rational' action given these beliefs and
1's strategy.

\bigskip

\textbf{Hold on, you! If 1 is choosing E\ anyway, why do we need to worry
about this?}

Ok, so we are only `considering' that 1 may play E. \

1 must ask \textit{`is playing E the best?'} \ `\textit{Will playing C or D
do better (in EU terms)?'}

To get the answer to this question, 2 needs to ask `\textit{if I\ play C (or
D), what will player 2 do?}'

\bigskip

But what player 2 does at \{C,D\} will depend on what player 2 \textit{%
believes }1 played -- C or D.

So, in order to decide the the EU payoffs to playing C (or D), to compare
this to playing E, 1 needs to know (or assign) what player 2's \textit{%
beliefs }are at \{C,D\}.

{\footnotesize Remember that if 1 choosing (E,J) is a complete strategy, but
this tells us nothing about what 2 is to believe if he \textquotedblleft
gets the move\textquotedblright !}

\bigskip

\section{Consistency of Beliefs with Strategies (Requirement 2)}

`In a steady state, each player's belief must be correct' -- Osborne

\bigskip

\textbf{Consistency of beliefs at information sets}\textit{:}

At each information set beliefs are determined by Bayes' rule and the
players' equilibrium strategies \textit{where possible'.}

However, \textquotedblleft off the equilibrium path,\textquotedblright\
e.g., at a surprise event, any updated belief is allowed.\bigskip

\textbf{Bayes Rule:}

$\Pr (X|Y_{1})=\frac{\Pr (X,Y_{1})}{\Pr (Y_{1})}$

\bigskip

$Pr(Y_{1}|X)=\frac{\Pr (X,Y_{1})}{\Pr (X)}=\frac{\Pr (X|Y_{1})\Pr (Y_{1})}{%
\Pr (X)}$

$=\frac{\Pr (X|Y)\Pr (Y_{1})}{\Pr (X|Y_{1})+\Pr (X|Y_{2})+...+\Pr (X|Y_{N})}$

\bigskip

If $Y$ drawn from a discrete set of N possible events $\mathbf{Y=}%
\{Y_{1},Y_{2},...,Y_{N}\}$

\bigskip

\textbf{Bayes rule} for conditional probabilities in the present case:

\[
\Pr (h^{\ast }|I_{i}\text{ reached, strategy }\beta )=\frac{\Pr (h^{\ast
}|\beta )}{\sum\limits_{h\in I_{i}}\Pr (h|\beta )}
\]

A player finds herself at information set $I_{i}$. \ Knowing all players'
strategies $\beta $. \

What probability should she assign that $h^{\ast }$ has been played? \

She should take the ratio of

...the (unconditional)\ probability that $h^{\ast }$ is played under the
strategy $\beta $ \

...to the probability we \textit{get }to the information set $I_{i}$ (which
is the sum of the probabilities of each history that leads us to $I_{i}$...
under strategy $\beta $). \ {\footnotesize This is only an interesting
calculation if we have a mixed or \textquotedblleft
behavioral\textquotedblright\ strategy (similar to a mixed strategy but
mixing at a particular information set/node). \ Otherwise the probability
will generally be 1 or will be undefined.}

\bigskip

\subsection{Example:\ Consistency of beliefs at information sets
\textquotedblleft on the equilibrium path:\textquotedblright}

\[
\FRAME{itbphFU}{5.0004in}{4in}{0in}{\Qcb{Fig. 317.1}}{}{Figure}{\special%
{language "Scientific Word";type "GRAPHIC";maintain-aspect-ratio
TRUE;display "USEDEF";valid_file "T";width 5.0004in;height 4in;depth
0in;original-width 5.6886in;original-height 3.5201in;cropleft "0";croptop
"1";cropright "1";cropbottom "0";tempfilename
'JSHANV2X.wmf';tempfile-properties "XPR";}}
\]

In fig. 317.1 above, we decided that \{Unready,Acquiesce\} and \{Out,Fight\}
are both SPNE, although only \{Unready,Acquiesce\} is optimal at all
information sets (`continuation games').

\bigskip

Let's consider the (candidate) equilibrium \textbf{\{Unready,Acquiesce\}},
which is a \textit{SPNE}, is \textit{optimal at each information set,}

and potentially a \textit{WSE}.

\bigskip

The incumbent's information set \{Ready, Unready\} is thus \textbf{on} the
equilibrium path. \

Given the challenger's equilibrium strategy (if this is an equilibrium),

the only belief that is \textit{consistent} is $\Pr (Ready)=0,\Pr
(Unready)=1.$

\bigskip

...And given this belief, the incumbent is playing the sequentially rational
strategy `acquiesce.'

\bigskip

\bigskip

\subsection{Example II:\ }

\[
\FRAME{itbphFU}{5.0004in}{4in}{0in}{\Qcb{Fig 326.1}}{}{Figure}{\special%
{language "Scientific Word";type "GRAPHIC";maintain-aspect-ratio
TRUE;display "USEDEF";valid_file "T";width 5.0004in;height 4in;depth
0in;original-width 5.1515in;original-height 4.8161in;cropleft "0";croptop
"1";cropright "1";cropbottom "0";tempfilename
'JSHANV33.wmf';tempfile-properties "XPR";}}
\]

\textbf{1. }Let's examine the pairing of beliefs and strategies depicted
above. \

Player 1's strategy (E,J)\ is depicted in blue.

Player 2's strategy (G)\ is depicted in red. \

We already noted that each player's strategy is sequentially rational
(requirement 1).

Do player 2's beliefs meet the  requirement (requirement II)?

Yes, they do trivially, because player 2's information set is off the
equilibrium path, so any beliefs are permitted.

So the assessment that pairs the profile of strategies $\{E,J;G\}$ with
player 2's beliefs \ -- $\Pr (C)=\frac{2}{3}$,$\Pr (D)=\frac{1}{3}$ is a PBE.

\bigskip

\bigskip

\textbf{2. }What about the profile $\{D,J;G\}$ (see below)-- This is a NE,
and sequentially rational for the previously discussed belief system -- $\Pr
(C)=\frac{2}{3}$,$\Pr (D)=\frac{1}{3}$ .

But given 1's strategy this belief system is not consistent! \

Can this be a WSE for some consistent belief system?

\[
\FRAME{itbphFU}{5.0004in}{4in}{0in}{\Qcb{fig 326.1}}{}{Figure}{\special%
{language "Scientific Word";type "GRAPHIC";maintain-aspect-ratio
TRUE;display "USEDEF";valid_file "T";width 5.0004in;height 4in;depth
0in;original-width 5.3358in;original-height 4.6484in;cropleft "0";croptop
"1";cropright "1";cropbottom "0";tempfilename
'JSHANV34.wmf';tempfile-properties "XPR";}}
\]

Given 1's strategy (D,J), the only consistent belief at information set
\{C,D\} is $\Pr (D)=1$.

But if $\Pr (D)=1$,
\begin{eqnarray*}
EU_{2}(G) &=&0 \\
EU_{2}(F) &=&1
\end{eqnarray*}

So playing G is not sequentially rational!

\bigskip

\bigskip

\subsubsection{\textbf{Note:}}

\textbf{Remember}:\ PBE\ is a \textit{refinement }of SPNE, which was a
refinement of NE. \ So, any PBE must be a SPNE and a NE. \

Let's examine a \textit{subset} of the strategic form, leaving out
strategies that include k (which is not subgame perfect).

\begin{tabular}{ccc}
& \textbf{F} & \textbf{G} \\
\textbf{C,J} & \underline{3},0 & 0,\underline{1} \\
\textbf{D,J} & 1,\underline{1} & \underline{2},0 \\
\textbf{E,J} & 2,\underline{0} & \underline{2},\underline{0}%
\end{tabular}

Since $\{E,J;G\}$ was the only SPNE in pure strategies we already knew it
(paired with appropriate beliefs) \ was the only PBE in pure strategies.

\bigskip

\section{PBE\ with chance moves: Some examples}

\subsection{Poker example:\ BNE, check for PBE}

Now let us examine a case where chance plays a role:\bigskip

\begin{equation}
\FRAME{itbphF}{5in}{4in}{0in}{}{}{Figure}{\special{language "Scientific
Word";type "GRAPHIC";maintain-aspect-ratio TRUE;display "USEDEF";valid_file
"T";width 5in;height 4in;depth 0in;original-width 4.6799in;original-height
3.584in;cropleft "0";croptop "1";cropright "1";cropbottom "0";tempfilename
'JSHANV35.wmf';tempfile-properties "XPR";}}
\end{equation}

\bigskip

\subsubsection{Checking BNE}

Bayesian normal form:\ Payoffs expressed as EU payoffs (over both types)

\begin{tabular}{ccc}
& \textbf{Pass} & \textbf{Meet} \\
\textbf{R,R} & \underline{1},-1 & 0,\underline{0} \\
\textbf{R,S} & 0,\underline{0} & \underline{$\frac{1}{2}$}$,-\frac{1}{2}$ \\
\textbf{S,R} & \underline{1},-1 & -$\frac{1}{2},$\underline{$\frac{1}{2}$}
\\
\textbf{S,S} & 0,\underline{0} & 0,\underline{0}%
\end{tabular}

\bigskip

S,S dominated by $P(R,R)\otimes \frac{1}{2}+P(R,S)\otimes \frac{1}{2}$

$\mathbf{\Longrightarrow }S,S$ not used in any NE

\bigskip
\begin{tabular}{ccc}
& \textbf{Pass} & \textbf{Meet} \\
\textbf{R,R} & \underline{1},-1 & 0,\underline{0} \\
\textbf{R,S} & 0,\underline{0} & \underline{$\frac{1}{2}$}$,-\frac{1}{2}$ \\
\textbf{S,R} & \underline{1},-1 & $-\frac{1}{2},$\underline{$\frac{1}{2}$}%
\end{tabular}

No equilibrium in pure strategies.

$\bigskip $

$\mathbf{\Longrightarrow }\Pr (meet)>0$

$\mathbf{\Longrightarrow }S,R$ not a BR

\bigskip

--If Meet than R,R better

\bigskip

--If Pass than R,R as good

$\bigskip $

$\mathbf{\Longrightarrow }$So don't put any probability on $S,R$; would do
better by moving this to $R,R$

$\bigskip $

$\mathbf{\Longrightarrow }$Mix between R,R and R,S

\bigskip

Which I claim was the intuitive result. \

\bigskip

You will always raise if you have the high cards (there is no way to do
better).

You will sometimes raise if you have the low cards (i.e., bluff).

\bigskip

BNE:\ $(P(R,R)\otimes \frac{1}{3}+P(R,S)\otimes \frac{2}{3},P(p)\otimes
\frac{1}{3}+P_{m}\otimes \frac{2}{3})$

Bluffs with probability $\frac{1}{3}$

`always calls bluff' with probability $\frac{2}{3}$

\bigskip

\subsubsection{Checking PBE}

Is this a PBE (if paired with consistent beliefs?) \ Let us check our 2
requirements; we need only check these for player 2 given player 1's
strategy, since for player 1 we need not about beliefs or sequential
rationality of his strategy.

\bigskip

\textit{Consistency of beliefs:}

The only consistent beliefs at 2's information set is that

\bigskip

\textit{Bayes rule again: }

$Pr(Y_{1}|X)=\frac{\Pr (X,Y_{1})}{\Pr (X)}=\frac{\Pr (X|Y_{1})\Pr (Y_{1})}{%
\Pr (X)}$

$=\frac{\Pr (X|Y)\Pr (Y_{1})}{\Pr (X|Y_{1})+\Pr (X|Y_{2})+...+\Pr (X|Y_{N})}$

\bigskip

Applying this:

Pr(high\TEXTsymbol{\vert}2 has move)=$\frac{pr(raise|high)\ast pr(high)}{%
pr(raise|high)\ast pr(high)+pr(raise|low)\ast pr(low)}=\frac{1\ast \frac{1}{2%
}}{1\ast \frac{1}{2}+\frac{1}{2}\ast \frac{1}{3}}=\allowbreak \frac{3}{4}$

\textit{Sequential rationality:\ }

If 2 plays p gets $-1$ (for sure)

If 2 plays m gets $-2\ast \frac{3}{4}+2\ast \frac{1}{4}=\allowbreak -1$

\bigskip

So 2 is indeed neutral between these 2 strategies and thus mixing (in the
proportions determined before) is a best response at this information set.

\bigskip

Hence, $(P(R,R)\otimes \frac{1}{3}+P(R,S)\otimes \frac{2}{3},P(p)\otimes
\frac{1}{3}+P_{m}\otimes \frac{2}{3})$ paired with beliefs Pr(high%
\TEXTsymbol{\vert}2 has move)=$\frac{3}{4}$ is a PBE.

\bigskip

\bigskip

\subsection{Simple 2x2 signaling game (education and labor market)}

\textbf{Players: }Sender, Reciever, Nature

\bigskip

\textbf{Actions (and order):}

1. Nature: Good ability, Bad ability (picks with probabilities P$_{G}$ \& 1-P%
$_{G}$)

2. Sender: Signal Good (e.g, School), Signal Bad (e.g., Drink)

3. Reciever: Hire, Not hire

\bigskip

\textbf{Information structure:}

Sender observes ability

Reciever does not observe ability (Nature's move).

Receiver observes Sender's signal.

\bigskip

\textbf{Payoffs}

In general

U$_{Sender}$=Salary - cost of signal (function of ability)

U$_{Reciever}$= $\left\{
\begin{array}{c}
Production(function\text{ }of\text{ }ability)-Salary=\Pi \ \text{\ if hire}
\\
0\text{ if not hire}%
\end{array}%
\right. $

-- In general, the production function and wages are such that the reciever
will want to hire a good worker and not hire a bad worker.

-- However, if the receiver doesn't know a worker's ability, it would be
more profitable to never hire than to always hire.\bigskip

\begin{equation}
\FRAME{itbphF}{5in}{4in}{0in}{}{}{Figure}{\special{language "Scientific
Word";type "GRAPHIC";maintain-aspect-ratio TRUE;display "USEDEF";valid_file
"T";width 5in;height 4in;depth 0in;original-width 7.3172in;original-height
4.9078in;cropleft "0";croptop "1";cropright "1";cropbottom "0";tempfilename
'JSHANV36.wmf';tempfile-properties "XPR";}}
\end{equation}

\subsubsection{Numerical Example I:}

In a continuous-action example (see text) this will the worker will choose
the level of effort, the firm the level of pay for the good and bad jobs, and

\bigskip

Suppose the salary is set (exogenously)\ at $S=4$

The payoff to of not being hired is $2$ (second best job).

The production of a bad worker is $3$, so the firm `nets' -1 from hiring
such a worker.

The productivity of a good worker is $6$, so the firm `nets' 2 from hiring
such a worker.

The cost of sending a good signal (e.g., going to school)\ is 1 for the good
worker and 3 for the bad worker.

\bigskip

Finally, the probability of nature selecting a good worker is 0.1. \

This implies the firm would rather hire a good worker but not hire a bad
worker.

However, the firm would rather hire no one than hiring everyone.

If education means the difference between a job or no job, in this case,
only the good worker would choose to go to school (send a `good' signal).

However if education will not make a difference (say, you are hired in
either case), then neither type would go to school, as it is costly.

This yields the following game tree:

\begin{equation}
\FRAME{itbphF}{5in}{4in}{0in}{}{}{Figure}{\special{language "Scientific
Word";type "GRAPHIC";maintain-aspect-ratio TRUE;display "USEDEF";valid_file
"T";width 5in;height 4in;depth 0in;original-width 7.8006in;original-height
5.7164in;cropleft "0";croptop "1";cropright "1";cropbottom "0";tempfilename
'JSHANW37.wmf';tempfile-properties "XPR";}}
\end{equation}

\textit{For example .... (check each outcome)}

If a worker is Good, goes to School and gets Hired, he will get a payoff of
3 (4 minus 1) and the firm gets a payoff of 2 \ This is the outcome in the
upper right corner.

\bigskip

To the left of this, if a worker is Good, goes to School and does Not get
hired, he will get a payoff of 1 (2 minus 1) and the firm gets a payoff of 2
\

\bigskip

To take another example, in the lower left outcome, if a worker is Bad,
chooses to Drink, and does Not get hired, he gets 2, and the firm gets zero.
\

Looking one to the left of this, if a worker is Bad, chooses to Drink, but
still gets Hired, he gets 4, and the firm gets -1. \

\bigskip

Let us check a potential `separating' equilibrium.

Suppose the sender plays (Drink,School) -- i.e., Drink if Bad, get an
education if Good.

Then suppose the reciever plays (Not, Hire) -- i.e., don't hire if Drink,
hire if School.

\bigskip

The only beliefs by a sender in such a situation that are \textit{consistent}
with this strategy is to believe Nature picked `Good' if he receives a Good
signal (School),

and Nature picked `Bad' if he receives a Bad signal (Drink). \

\bigskip

Given these beliefs and strategies, are these best responses at each
information set?

\bigskip

Compare EU's:

\[
EU_{S}(Drink,School;(Not,Hire))=0.1(4)+0.9(2)=0.2
\]

Note this EU is \textit{additive} in the EU of each `type' (good or bad). \
\ Thus for this to be a BR it has to be a BR for each type of sender. \ Is
it?

\bigskip

Note that given the receiver's strategy (hire if and only if educated)\ the
decision to get an education is \textit{pivotal}. \ And as we set it up,
only the good type wants to get an education if it is pivotal. \ So each
type of sender is behaving optimally (i.e., the sender is behaving optimally
in each contingency)

\bigskip

Given the sender's strategy (get education if and only if Good), is the
reciever behaving optimally given his beliefs? \ Of course she is. \

As we set it up, the receiver prefers to hire the Good, and not hire the
bad. \

So this $\{Drink,School;Not,Hire\}$ is a PBE (sequential rational under the
beliefs mentioned above, which are of course consistent).

\bigskip

But this may not be the only PBE for this numerical example -- there may be
others. \

For example, pooling on no school if the employer will not hire for either
signal is a PBE.

I.e., \{$Drink,Drink;Not,Not$\} paired with the belief ($%
Pr(Good|Drink)=Pr(Good|School)=0.1$) is a PBE.

\textbf{Exercise: }Check this.

\bigskip

\textit{Note:\ there will always be a pooling on bad signal equilibrium in
such games!}\bigskip

\subsubsection{Numerical Example II}

Let's change the payoffs.

Suppose the salary is set (exogenously)\ at $S=5$

The payoff to of not being hired is $0.$

The production of a bad worker is 4, so the firm `nets' -1 from hiring such
a worker.

The productivity of a good worker is 7, so the firm `nets' 2 from hiring
such a worker.

The cost of sending a good signal (e.g., going to school)\ is 1 for the good
worker and 2 for the bad worker.

\bigskip

Finally, the probability of nature selecting a good worker is 0.1. \

This implies (as before) the firm would rather hire a good worker but not
hire a bad worker.

However, (as before)\ the firm would rather hire no one than hiring everyone.

\bigskip

\textit{Notice:\ (Unlike before ) }If education means the difference between
a job or no job (i.e., is `pivotal'), both types would choose to go to
school (send a `good' signal).

However (as before) if education will not make a difference (say, you are
hired in either case), then neither type would go to school, as it is costly.

This yields the following game tree:

\begin{equation}
\FRAME{itbphF}{5in}{4in}{0in}{}{}{Figure}{\special{language "Scientific
Word";type "GRAPHIC";maintain-aspect-ratio TRUE;display "USEDEF";valid_file
"T";width 5in;height 4in;depth 0in;original-width 7.2834in;original-height
5.7086in;cropleft "0";croptop "1";cropright "1";cropbottom "0";tempfilename
'JSHANW38.wmf';tempfile-properties "XPR";}}
\end{equation}

For example ....

\bigskip

\textbf{Exercise}: check each outcome

\bigskip

Let us check a potential `separating' equilibrium.

Suppose the sender plays (Drink,School) -- i.e., don't get an education if
bad, get an education if good.

Then suppose the reciever plays (Not, Hire) -- i.e., don't hire if Drink,
hire if education.

\bigskip

The only beliefs by a sender in such a situation that are \textit{consistent}
with this strategy is to believe Nature picked `Good' if he receives a Good
signal (education),

and Nature picked `Bad' if he receives a Bad signal (Drink). \

\bigskip

Given these beliefs and strategies, are these best responses at each
information set?\bigskip

Note that given the receiver's strategy (hire if and only if educated)\ the
decision to get an education is \textit{pivotal}. \ And as we set it up,
each type of sender wants to get an education if it is pivotal. \ So the
sender is not behaving optimally in the contingency where he is Bad (i.e.,
for this type). \ The bad type should \textit{also} want to pretend to be a
good type and get an education. .So this is not a PBE!

\bigskip

We can look for \textit{pooling} PBE in this case -- there seem to be no
separating PBE\ (in pure strategies) here.

Note that the firm prefers to hire no one than hire everyone (given the
setup). \ So, if both types do the same thing, the firm cannot tell which is
which, and will thus hire no one. \ So for any pooling equilibrium here, the
firm will hire no one. \ For example, we can look at \{$Drink,Drink;Not,Not$%
\}.

\bigskip

However, here the firm could assign \textit{any} beliefs if it observes a
good signal (education) -- such a signal is \textit{off the equilibrium
path. \ }It gets complicated now. \ One possible belief that the firm might
hold is that any player who signals \textit{Good} (gets an education) is a
Good worker. \ But if this were the beliefs, the firm would not be behaving
optimally -- it should thus \textit{hire} such a worker!

\bigskip

Another possible belief is that if a worker signals `Good' it has only a $.1$
probability of being a Good worker. \ I.e., the conditional probability
(after observing such off-path play) is the same as the unconditional
probability.This belief will make the strategy profile \{$%
Drink,Drink;Not,Not $\} sequentially rational. \ Thus \{$Drink,Drink;Not,Not$%
\} paired with the belief ($Pr(Good|Drink)=Pr(Good|School)=0.1$) is a PBE.

\bigskip

We can also look for other possibilities. \ We may find a separating (or
pooling) NE\ where the receiver plays \textit{behavioral strategies}, hiring
only some proportion of the time when she gets a good signal. \ In such a
case, the EU benefit to the sender may be such that getting an education is
only pivotal for the Good \ \ This is too advanced and complicated to cover
now.

\bigskip

\bigskip

\subsection{Question from last year's final}

{\Large Question 4 [50]}

Imagine the following `signaling game':

ONE\ WORKER, 2 types

A worker's type is $\theta \in \{1,2\}\ $where the probability that $\theta
=2$\ is $\frac{1}{2}$. \

The productivity of a worker in a job is $\theta $. \ The worker observes
this, the firm does not.

Each worker chooses an education level $y=1$ or $y=0$. \ The cost of
obtaining this education level is $\frac{2}{\theta }$.

Suppose the wage at the firm is $\frac{3}{2}$. \ \ Assume that if the worker
is not hired his earnings are $0$.

The firm observes the level of education chosen by the worker.

Assume that the workers payoff is equal to his earnings minus the cost of
any education he attains, and the `good' firm's payoff is its net profit
(production less wages)

\bigskip

\textbf{a. Define a `Perfect Bayesian Equilibrium' (remember, there a PBE
has at least two requirements). \ What will be required for a PBE\ in this
game? [12]}

\bigskip

\textbf{b. \ Depict this\ interaction between the worker and the `good' firm
as a game tree (extensive form). [10]}

\textbf{c. Solve for and give an assessment (strategies and beliefs) that
yield a pooling (perfect Bayesian) equilibrium in pure strategies for this
game. [13]}

\bigskip

\textbf{(4) Solve for and give an assessment (strategies and beliefs) that
yield a separating (perfect Bayesian) equilibrium (in pure strategies) for
this game. \ Briefly explain why it is a PBE. [15]}

\bigskip

\end{document}
